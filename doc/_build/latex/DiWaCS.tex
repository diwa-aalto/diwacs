% Generated by Sphinx.
\def\sphinxdocclass{report}
\documentclass[letterpaper,10pt,english]{sphinxmanual}
\usepackage[utf8]{inputenc}
\DeclareUnicodeCharacter{00A0}{\nobreakspace}
\usepackage{cmap}
\usepackage[T1]{fontenc}
\usepackage{babel}
\usepackage{times}
\usepackage[Bjarne]{fncychap}
\usepackage{longtable}
\usepackage{sphinx}
\usepackage{multirow}


\title{DiWaCS Documentation}
\date{July 15, 2013}
\release{0.9.3.0}
\author{Nick Eriksson}
\newcommand{\sphinxlogo}{}
\renewcommand{\releasename}{Release}
\makeindex

\makeatletter
\def\PYG@reset{\let\PYG@it=\relax \let\PYG@bf=\relax%
    \let\PYG@ul=\relax \let\PYG@tc=\relax%
    \let\PYG@bc=\relax \let\PYG@ff=\relax}
\def\PYG@tok#1{\csname PYG@tok@#1\endcsname}
\def\PYG@toks#1+{\ifx\relax#1\empty\else%
    \PYG@tok{#1}\expandafter\PYG@toks\fi}
\def\PYG@do#1{\PYG@bc{\PYG@tc{\PYG@ul{%
    \PYG@it{\PYG@bf{\PYG@ff{#1}}}}}}}
\def\PYG#1#2{\PYG@reset\PYG@toks#1+\relax+\PYG@do{#2}}

\expandafter\def\csname PYG@tok@gd\endcsname{\def\PYG@tc##1{\textcolor[rgb]{0.63,0.00,0.00}{##1}}}
\expandafter\def\csname PYG@tok@gu\endcsname{\let\PYG@bf=\textbf\def\PYG@tc##1{\textcolor[rgb]{0.50,0.00,0.50}{##1}}}
\expandafter\def\csname PYG@tok@gt\endcsname{\def\PYG@tc##1{\textcolor[rgb]{0.00,0.27,0.87}{##1}}}
\expandafter\def\csname PYG@tok@gs\endcsname{\let\PYG@bf=\textbf}
\expandafter\def\csname PYG@tok@gr\endcsname{\def\PYG@tc##1{\textcolor[rgb]{1.00,0.00,0.00}{##1}}}
\expandafter\def\csname PYG@tok@cm\endcsname{\let\PYG@it=\textit\def\PYG@tc##1{\textcolor[rgb]{0.25,0.50,0.56}{##1}}}
\expandafter\def\csname PYG@tok@vg\endcsname{\def\PYG@tc##1{\textcolor[rgb]{0.73,0.38,0.84}{##1}}}
\expandafter\def\csname PYG@tok@m\endcsname{\def\PYG@tc##1{\textcolor[rgb]{0.13,0.50,0.31}{##1}}}
\expandafter\def\csname PYG@tok@mh\endcsname{\def\PYG@tc##1{\textcolor[rgb]{0.13,0.50,0.31}{##1}}}
\expandafter\def\csname PYG@tok@cs\endcsname{\def\PYG@tc##1{\textcolor[rgb]{0.25,0.50,0.56}{##1}}\def\PYG@bc##1{\setlength{\fboxsep}{0pt}\colorbox[rgb]{1.00,0.94,0.94}{\strut ##1}}}
\expandafter\def\csname PYG@tok@ge\endcsname{\let\PYG@it=\textit}
\expandafter\def\csname PYG@tok@vc\endcsname{\def\PYG@tc##1{\textcolor[rgb]{0.73,0.38,0.84}{##1}}}
\expandafter\def\csname PYG@tok@il\endcsname{\def\PYG@tc##1{\textcolor[rgb]{0.13,0.50,0.31}{##1}}}
\expandafter\def\csname PYG@tok@go\endcsname{\def\PYG@tc##1{\textcolor[rgb]{0.20,0.20,0.20}{##1}}}
\expandafter\def\csname PYG@tok@cp\endcsname{\def\PYG@tc##1{\textcolor[rgb]{0.00,0.44,0.13}{##1}}}
\expandafter\def\csname PYG@tok@gi\endcsname{\def\PYG@tc##1{\textcolor[rgb]{0.00,0.63,0.00}{##1}}}
\expandafter\def\csname PYG@tok@gh\endcsname{\let\PYG@bf=\textbf\def\PYG@tc##1{\textcolor[rgb]{0.00,0.00,0.50}{##1}}}
\expandafter\def\csname PYG@tok@ni\endcsname{\let\PYG@bf=\textbf\def\PYG@tc##1{\textcolor[rgb]{0.84,0.33,0.22}{##1}}}
\expandafter\def\csname PYG@tok@nl\endcsname{\let\PYG@bf=\textbf\def\PYG@tc##1{\textcolor[rgb]{0.00,0.13,0.44}{##1}}}
\expandafter\def\csname PYG@tok@nn\endcsname{\let\PYG@bf=\textbf\def\PYG@tc##1{\textcolor[rgb]{0.05,0.52,0.71}{##1}}}
\expandafter\def\csname PYG@tok@no\endcsname{\def\PYG@tc##1{\textcolor[rgb]{0.38,0.68,0.84}{##1}}}
\expandafter\def\csname PYG@tok@na\endcsname{\def\PYG@tc##1{\textcolor[rgb]{0.25,0.44,0.63}{##1}}}
\expandafter\def\csname PYG@tok@nb\endcsname{\def\PYG@tc##1{\textcolor[rgb]{0.00,0.44,0.13}{##1}}}
\expandafter\def\csname PYG@tok@nc\endcsname{\let\PYG@bf=\textbf\def\PYG@tc##1{\textcolor[rgb]{0.05,0.52,0.71}{##1}}}
\expandafter\def\csname PYG@tok@nd\endcsname{\let\PYG@bf=\textbf\def\PYG@tc##1{\textcolor[rgb]{0.33,0.33,0.33}{##1}}}
\expandafter\def\csname PYG@tok@ne\endcsname{\def\PYG@tc##1{\textcolor[rgb]{0.00,0.44,0.13}{##1}}}
\expandafter\def\csname PYG@tok@nf\endcsname{\def\PYG@tc##1{\textcolor[rgb]{0.02,0.16,0.49}{##1}}}
\expandafter\def\csname PYG@tok@si\endcsname{\let\PYG@it=\textit\def\PYG@tc##1{\textcolor[rgb]{0.44,0.63,0.82}{##1}}}
\expandafter\def\csname PYG@tok@s2\endcsname{\def\PYG@tc##1{\textcolor[rgb]{0.25,0.44,0.63}{##1}}}
\expandafter\def\csname PYG@tok@vi\endcsname{\def\PYG@tc##1{\textcolor[rgb]{0.73,0.38,0.84}{##1}}}
\expandafter\def\csname PYG@tok@nt\endcsname{\let\PYG@bf=\textbf\def\PYG@tc##1{\textcolor[rgb]{0.02,0.16,0.45}{##1}}}
\expandafter\def\csname PYG@tok@nv\endcsname{\def\PYG@tc##1{\textcolor[rgb]{0.73,0.38,0.84}{##1}}}
\expandafter\def\csname PYG@tok@s1\endcsname{\def\PYG@tc##1{\textcolor[rgb]{0.25,0.44,0.63}{##1}}}
\expandafter\def\csname PYG@tok@gp\endcsname{\let\PYG@bf=\textbf\def\PYG@tc##1{\textcolor[rgb]{0.78,0.36,0.04}{##1}}}
\expandafter\def\csname PYG@tok@sh\endcsname{\def\PYG@tc##1{\textcolor[rgb]{0.25,0.44,0.63}{##1}}}
\expandafter\def\csname PYG@tok@ow\endcsname{\let\PYG@bf=\textbf\def\PYG@tc##1{\textcolor[rgb]{0.00,0.44,0.13}{##1}}}
\expandafter\def\csname PYG@tok@sx\endcsname{\def\PYG@tc##1{\textcolor[rgb]{0.78,0.36,0.04}{##1}}}
\expandafter\def\csname PYG@tok@bp\endcsname{\def\PYG@tc##1{\textcolor[rgb]{0.00,0.44,0.13}{##1}}}
\expandafter\def\csname PYG@tok@c1\endcsname{\let\PYG@it=\textit\def\PYG@tc##1{\textcolor[rgb]{0.25,0.50,0.56}{##1}}}
\expandafter\def\csname PYG@tok@kc\endcsname{\let\PYG@bf=\textbf\def\PYG@tc##1{\textcolor[rgb]{0.00,0.44,0.13}{##1}}}
\expandafter\def\csname PYG@tok@c\endcsname{\let\PYG@it=\textit\def\PYG@tc##1{\textcolor[rgb]{0.25,0.50,0.56}{##1}}}
\expandafter\def\csname PYG@tok@mf\endcsname{\def\PYG@tc##1{\textcolor[rgb]{0.13,0.50,0.31}{##1}}}
\expandafter\def\csname PYG@tok@err\endcsname{\def\PYG@bc##1{\setlength{\fboxsep}{0pt}\fcolorbox[rgb]{1.00,0.00,0.00}{1,1,1}{\strut ##1}}}
\expandafter\def\csname PYG@tok@kd\endcsname{\let\PYG@bf=\textbf\def\PYG@tc##1{\textcolor[rgb]{0.00,0.44,0.13}{##1}}}
\expandafter\def\csname PYG@tok@ss\endcsname{\def\PYG@tc##1{\textcolor[rgb]{0.32,0.47,0.09}{##1}}}
\expandafter\def\csname PYG@tok@sr\endcsname{\def\PYG@tc##1{\textcolor[rgb]{0.14,0.33,0.53}{##1}}}
\expandafter\def\csname PYG@tok@mo\endcsname{\def\PYG@tc##1{\textcolor[rgb]{0.13,0.50,0.31}{##1}}}
\expandafter\def\csname PYG@tok@mi\endcsname{\def\PYG@tc##1{\textcolor[rgb]{0.13,0.50,0.31}{##1}}}
\expandafter\def\csname PYG@tok@kn\endcsname{\let\PYG@bf=\textbf\def\PYG@tc##1{\textcolor[rgb]{0.00,0.44,0.13}{##1}}}
\expandafter\def\csname PYG@tok@o\endcsname{\def\PYG@tc##1{\textcolor[rgb]{0.40,0.40,0.40}{##1}}}
\expandafter\def\csname PYG@tok@kr\endcsname{\let\PYG@bf=\textbf\def\PYG@tc##1{\textcolor[rgb]{0.00,0.44,0.13}{##1}}}
\expandafter\def\csname PYG@tok@s\endcsname{\def\PYG@tc##1{\textcolor[rgb]{0.25,0.44,0.63}{##1}}}
\expandafter\def\csname PYG@tok@kp\endcsname{\def\PYG@tc##1{\textcolor[rgb]{0.00,0.44,0.13}{##1}}}
\expandafter\def\csname PYG@tok@w\endcsname{\def\PYG@tc##1{\textcolor[rgb]{0.73,0.73,0.73}{##1}}}
\expandafter\def\csname PYG@tok@kt\endcsname{\def\PYG@tc##1{\textcolor[rgb]{0.56,0.13,0.00}{##1}}}
\expandafter\def\csname PYG@tok@sc\endcsname{\def\PYG@tc##1{\textcolor[rgb]{0.25,0.44,0.63}{##1}}}
\expandafter\def\csname PYG@tok@sb\endcsname{\def\PYG@tc##1{\textcolor[rgb]{0.25,0.44,0.63}{##1}}}
\expandafter\def\csname PYG@tok@k\endcsname{\let\PYG@bf=\textbf\def\PYG@tc##1{\textcolor[rgb]{0.00,0.44,0.13}{##1}}}
\expandafter\def\csname PYG@tok@se\endcsname{\let\PYG@bf=\textbf\def\PYG@tc##1{\textcolor[rgb]{0.25,0.44,0.63}{##1}}}
\expandafter\def\csname PYG@tok@sd\endcsname{\let\PYG@it=\textit\def\PYG@tc##1{\textcolor[rgb]{0.25,0.44,0.63}{##1}}}

\def\PYGZbs{\char`\\}
\def\PYGZus{\char`\_}
\def\PYGZob{\char`\{}
\def\PYGZcb{\char`\}}
\def\PYGZca{\char`\^}
\def\PYGZam{\char`\&}
\def\PYGZlt{\char`\<}
\def\PYGZgt{\char`\>}
\def\PYGZsh{\char`\#}
\def\PYGZpc{\char`\%}
\def\PYGZdl{\char`\$}
\def\PYGZhy{\char`\-}
\def\PYGZsq{\char`\'}
\def\PYGZdq{\char`\"}
\def\PYGZti{\char`\~}
% for compatibility with earlier versions
\def\PYGZat{@}
\def\PYGZlb{[}
\def\PYGZrb{]}
\makeatother

\begin{document}

\maketitle
\tableofcontents
\phantomsection\label{index::doc}


DiWaCS is an application developed for \href{https://cse.aalto.fi/research/groups/stratus/research/research-projects/}{DiWa smart space}
and should be used \textbf{only} inside \textbf{Diwaamo}.
DiWaCS connects to address \textbf{239.128.128.1:5555} using \href{http://code.google.com/p/openpgm/}{Pragmatic General Multicast (PGM)}.
DiWaCS is built on \href{http://www.python.org}{Python} and \href{http://www.wxpython.org}{WxPython} is used for UI programming.
Currently, only supported platform is \textbf{Windows 7}.
\begin{description}
\item[{Required python modules for DiWaCS:}] \leavevmode\begin{itemize}
\item {} 
\href{http://www.voidspace.org.uk/python/configobj.html}{Configobj}

\item {} 
\href{http://pypi.python.org/pypi/lxml/}{lxml}

\item {} 
\href{http://www.pythonware.com/products/pil/}{PIL}

\item {} 
\href{http://pypi.python.org/pypi/PyAudio}{PyAudio}

\item {} 
\href{http://pubsub.sourceforge.net/}{Python Pubsub}

\item {} 
\href{http://www.sqlalchemy.org/}{SQLAlchemy}

\item {} 
\href{http://packages.python.org/watchdog/}{Watchdog}

\item {} 
\href{http://pypi.python.org/pypi/WMI/1.4.9/}{WMI}

\item {} 
\href{http://www.wxpython.org}{WxPython}

\item {} 
\href{http://zeromq.org}{ZeroMQ} with \href{http://code.google.com/p/openpgm/}{openpgm} support

\end{itemize}

\end{description}

Contents:


\chapter{Automated Code Documentation}
\label{api:automated-code-documentation}\label{api::doc}\label{api:welcome-to-diwacs-documentation}
Documentation generated on 2013-07-15 at 15:22.


\section{Add file module}
\label{add_file:module-add_file}\label{add_file:add-file-module}\label{add_file::doc}\index{add\_file (module)}
Created on 5.6.2012
\begin{quote}\begin{description}
\item[{platform}] \leavevmode
Windows

\item[{synopsis}] \leavevmode
Used to add a file in the current project.

\item[{warning}] \leavevmode
Requires ZeroMQ.

\item[{author}] \leavevmode
neriksso

\end{description}\end{quote}
\index{main() (in module add\_file)}

\begin{fulllineitems}
\phantomsection\label{add_file:add_file.main}\pysiglinewithargsret{\code{add\_file.}\bfcode{main}}{}{}
Main function of the sub program.

Sub program is meant to be bound to windows explorer context menu.
Context menu allows the user to quickly add files to project without
interacting with DiWaCS directly.

Transmits the add\_file command to DiWaCS via interprocess socket.
\begin{quote}\begin{description}
\item[{Parameters}] \leavevmode
\textbf{filepath} (\emph{String}) -- Path of the file to be added.

\item[{Returns}] \leavevmode
windows success code (0 on success).

\item[{Return type}] \leavevmode
Integer

\end{description}\end{quote}

\end{fulllineitems}



\section{Send file module}
\label{send_file:module-send_file_to}\label{send_file::doc}\label{send_file:send-file-module}\index{send\_file\_to (module)}
Created on 5.6.2012
\begin{quote}\begin{description}
\item[{author}] \leavevmode
neriksso

\item[{requires}] \leavevmode
Requires ZeroMQ

\item[{synopsis}] \leavevmode
Used to send a file to another node.

\end{description}\end{quote}
\index{main() (in module send\_file\_to)}

\begin{fulllineitems}
\phantomsection\label{send_file:send_file_to.main}\pysiglinewithargsret{\code{send\_file\_to.}\bfcode{main}}{}{}
Main function of the sub program.

Sub program is meant to be bound to windows explorer context menu.
Context menu allows the user to quickly send files without interacting
with DiWaCS directly.

Transmits the send\_to command to DiWaCS via interprocess connection.
\begin{quote}\begin{description}
\item[{Parameters}] \leavevmode\begin{itemize}
\item {} 
\textbf{node\_id} (\emph{Integer}) -- ID of the node to send the file to.

\item {} 
\textbf{filepath} (\emph{String}) -- Path of the file to be sent.

\end{itemize}

\item[{Returns}] \leavevmode
windows success code (0 on success).

\item[{Return type}] \leavevmode
Integer

\end{description}\end{quote}

\end{fulllineitems}



\section{Controller package}
\label{controller::doc}\label{controller:controller-package}
Used to control the database.


\subsection{controller.activity module}
\label{controller:module-controller.activity}\label{controller:controller-activity-module}\index{controller.activity (module)}
Created on 28.6.2013
\begin{quote}\begin{description}
\item[{author}] \leavevmode
neriksso

\end{description}\end{quote}
\index{add\_activity() (in module controller.activity)}

\begin{fulllineitems}
\phantomsection\label{controller:controller.activity.add_activity}\pysiglinewithargsret{\code{controller.activity.}\bfcode{add\_activity}}{\emph{project\_id}, \emph{pgm\_group}, \emph{session\_id=None}, \emph{activity\_id=None}}{}
Add activity to database.
\begin{quote}\begin{description}
\item[{Parameters}] \leavevmode\begin{itemize}
\item {} 
\textbf{project\_id} (\emph{Integer}) -- ID of the project Activity is associated with.

\item {} 
\textbf{pgm\_group} (\emph{Integer}) -- The PGM Group number.

\item {} 
\textbf{session\_id} (\emph{Integer}) -- ID of the session Activity is associated with.

\item {} 
\textbf{activity\_id} (\emph{Integer}) -- ID of the activity.

\end{itemize}

\item[{Returns}] \leavevmode
Activity ID of the added activity.

\item[{Return type}] \leavevmode
Integer

\end{description}\end{quote}

\end{fulllineitems}

\index{get\_active\_activity() (in module controller.activity)}

\begin{fulllineitems}
\phantomsection\label{controller:controller.activity.get_active_activity}\pysiglinewithargsret{\code{controller.activity.}\bfcode{get\_active\_activity}}{\emph{pgm\_group}}{}
Get the latest active activity id.
\begin{quote}\begin{description}
\item[{Parameters}] \leavevmode
\textbf{pgm\_group} (\emph{Integer}) -- The PGM Group number.

\item[{Returns}] \leavevmode
Latest active activity ID.

\item[{Return type}] \leavevmode
Integer

\end{description}\end{quote}

\end{fulllineitems}

\index{logger() (in module controller.activity)}

\begin{fulllineitems}
\phantomsection\label{controller:controller.activity.logger}\pysiglinewithargsret{\code{controller.activity.}\bfcode{logger}}{}{}
Return controller logger.

\end{fulllineitems}

\index{unset\_activity() (in module controller.activity)}

\begin{fulllineitems}
\phantomsection\label{controller:controller.activity.unset_activity}\pysiglinewithargsret{\code{controller.activity.}\bfcode{unset\_activity}}{\emph{pgm\_group}}{}
Unsets activity for PGM Group.
\begin{quote}\begin{description}
\item[{Parameters}] \leavevmode
\textbf{pgm\_group} (\emph{Integer}) -- The PGM Group number.

\end{description}\end{quote}

\end{fulllineitems}



\subsection{controller.common module}
\label{controller:module-controller.common}\label{controller:controller-common-module}\index{controller.common (module)}
Created on 28.6.2013
\begin{quote}\begin{description}
\item[{author}] \leavevmode
neriksso

\end{description}\end{quote}
\index{connect\_to\_database() (in module controller.common)}

\begin{fulllineitems}
\phantomsection\label{controller:controller.common.connect_to_database}\pysiglinewithargsret{\code{controller.common.}\bfcode{connect\_to\_database}}{\emph{expire=False}}{}
Connect to the database and return a Session object.
\begin{quote}\begin{description}
\item[{Parameters}] \leavevmode
\textbf{expire} (\emph{Boolean}) -- Parameter passed to session maker as expire\_on\_commit.

\item[{Returns}] \leavevmode
Session.

\item[{Return type}] \leavevmode
\code{sqlalchemy.orm.session.Session}

\end{description}\end{quote}

\end{fulllineitems}

\index{create\_all() (in module controller.common)}

\begin{fulllineitems}
\phantomsection\label{controller:controller.common.create_all}\pysiglinewithargsret{\code{controller.common.}\bfcode{create\_all}}{}{}
Create tables to the database.

\end{fulllineitems}

\index{delete\_record() (in module controller.common)}

\begin{fulllineitems}
\phantomsection\label{controller:controller.common.delete_record}\pysiglinewithargsret{\code{controller.common.}\bfcode{delete\_record}}{\emph{record\_model}, \emph{id\_number}}{}
Delete a record from database
\begin{quote}\begin{description}
\item[{Parameters}] \leavevmode\begin{itemize}
\item {} 
\textbf{record\_model} (\code{sqlalchemy.ext.declarative.declarative\_base()}) -- The model for which to delete a record.

\item {} 
\textbf{id\_number} (\emph{Integer}) -- Recond id.

\end{itemize}

\item[{Returns}] \leavevmode
Success.

\item[{Return type}] \leavevmode
Boolean

\end{description}\end{quote}

\end{fulllineitems}

\index{get\_action\_id\_by\_name() (in module controller.common)}

\begin{fulllineitems}
\phantomsection\label{controller:controller.common.get_action_id_by_name}\pysiglinewithargsret{\code{controller.common.}\bfcode{get\_action\_id\_by\_name}}{\emph{action\_name}}{}
Get the static ID of action name.

\end{fulllineitems}

\index{get\_or\_create() (in module controller.common)}

\begin{fulllineitems}
\phantomsection\label{controller:controller.common.get_or_create}\pysiglinewithargsret{\code{controller.common.}\bfcode{get\_or\_create}}{\emph{database}, \emph{model}, \emph{**kwargs}}{}
Fetches or creates a instance.
\begin{quote}\begin{description}
\item[{Parameters}] \leavevmode\begin{itemize}
\item {} 
\textbf{database} (\code{sqlalchemy.orm.session.Session}) -- a related database.

\item {} 
\textbf{model} (\code{sqlalchemy.ext.declarative.declarative\_base}) -- The model of which an instance is wanted.

\end{itemize}

\item[{Returns}] \leavevmode
An object of the desired model.

\end{description}\end{quote}

\end{fulllineitems}

\index{set\_node\_name() (in module controller.common)}

\begin{fulllineitems}
\phantomsection\label{controller:controller.common.set_node_name}\pysiglinewithargsret{\code{controller.common.}\bfcode{set\_node\_name}}{\emph{name}}{}
Set the stored node name for own swnp node as global.
\begin{quote}\begin{description}
\item[{Warning }] \leavevmode
This should be removed in the future as globals are bad.

\end{description}\end{quote}

\end{fulllineitems}

\index{set\_node\_screens() (in module controller.common)}

\begin{fulllineitems}
\phantomsection\label{controller:controller.common.set_node_screens}\pysiglinewithargsret{\code{controller.common.}\bfcode{set\_node\_screens}}{\emph{screens}}{}
Set the stored node screens settings for own swnp node as global.
\begin{quote}\begin{description}
\item[{Warning }] \leavevmode
This should be removed in the future as globals are bad.

\end{description}\end{quote}

\end{fulllineitems}

\index{test\_connection() (in module controller.common)}

\begin{fulllineitems}
\phantomsection\label{controller:controller.common.test_connection}\pysiglinewithargsret{\code{controller.common.}\bfcode{test\_connection}}{}{}
Test the connection to database.
\begin{quote}\begin{description}
\item[{Returns}] \leavevmode
Does the software have access to the database at this time.

\item[{Return type}] \leavevmode
Boolean

\end{description}\end{quote}

\end{fulllineitems}

\index{update\_database() (in module controller.common)}

\begin{fulllineitems}
\phantomsection\label{controller:controller.common.update_database}\pysiglinewithargsret{\code{controller.common.}\bfcode{update\_database}}{}{}
Update the database connection engine.

\begin{notice}{note}{Note:}
This only works when DB\_STRING is completely defined by
the log reader.
\end{notice}

\end{fulllineitems}



\subsection{controller.computer module}
\label{controller:module-controller.computer}\label{controller:controller-computer-module}\index{controller.computer (module)}
Created on 28.6.2013
\begin{quote}\begin{description}
\item[{author}] \leavevmode
neriksso

\end{description}\end{quote}
\index{add\_computer() (in module controller.computer)}

\begin{fulllineitems}
\phantomsection\label{controller:controller.computer.add_computer}\pysiglinewithargsret{\code{controller.computer.}\bfcode{add\_computer}}{\emph{name}, \emph{pc\_ip}, \emph{wos\_id}}{}
Add a new computer to the database.
\begin{quote}\begin{description}
\item[{Parameters}] \leavevmode\begin{itemize}
\item {} 
\textbf{name} (\emph{String}) -- Name of the computer.

\item {} 
\textbf{pc\_ip} (\emph{String}) -- IP address of the computer.

\item {} 
\textbf{wos\_id} (\emph{Integer}) -- Node ID of the computer (usually the last part of IP).

\end{itemize}

\item[{Returns}] \leavevmode
The added computer

\item[{Return type}] \leavevmode
{\hyperref[models:models.Computer]{\code{models.Computer}}}

\end{description}\end{quote}

\end{fulllineitems}

\index{add\_computer\_to\_session() (in module controller.computer)}

\begin{fulllineitems}
\phantomsection\label{controller:controller.computer.add_computer_to_session}\pysiglinewithargsret{\code{controller.computer.}\bfcode{add\_computer\_to\_session}}{\emph{session}, \emph{name}, \emph{pc\_ip}, \emph{wos\_id}}{}
Adds a computer to a session.
\begin{quote}\begin{description}
\item[{Parameters}] \leavevmode\begin{itemize}
\item {} 
\textbf{session} ({\hyperref[models:models.Session]{\code{models.Session}}}) -- A current session.

\item {} 
\textbf{name} (\emph{String}) -- A name of the computer.

\item {} 
\textbf{pc\_ip} (\emph{Integer}) -- Computers IP address.

\item {} 
\textbf{wos\_id} (\emph{Integer}) -- Wos id of the computer.

\end{itemize}

\end{description}\end{quote}

\end{fulllineitems}

\index{get\_active\_computers() (in module controller.computer)}

\begin{fulllineitems}
\phantomsection\label{controller:controller.computer.get_active_computers}\pysiglinewithargsret{\code{controller.computer.}\bfcode{get\_active\_computers}}{\emph{timeout}}{}
Get all the active computers from database.
\begin{quote}\begin{description}
\item[{Parameters}] \leavevmode
\textbf{timeout} (\emph{Integer}) -- The number of seconds an ``active'' computer may have been idle while
still being considered active.

\item[{Returns}] \leavevmode
A list of active computers.

\item[{Return type}] \leavevmode
List of {\hyperref[models:models.Computer]{\code{models.Computer}}}

\end{description}\end{quote}

\end{fulllineitems}

\index{get\_active\_responsive\_nodes() (in module controller.computer)}

\begin{fulllineitems}
\phantomsection\label{controller:controller.computer.get_active_responsive_nodes}\pysiglinewithargsret{\code{controller.computer.}\bfcode{get\_active\_responsive\_nodes}}{\emph{pgm\_group}}{}
Return the wos\_id fields of all active responsive nodes.
\begin{quote}\begin{description}
\item[{Parameters}] \leavevmode
\textbf{pgm\_group} (\emph{Integer}) -- The responsive group we want.

\item[{Returns}] \leavevmode
A list of node IDs that are both active and responsive.

\item[{Return type}] \leavevmode
A list of Integer

\end{description}\end{quote}

\end{fulllineitems}

\index{last\_active\_computer() (in module controller.computer)}

\begin{fulllineitems}
\phantomsection\label{controller:controller.computer.last_active_computer}\pysiglinewithargsret{\code{controller.computer.}\bfcode{last\_active\_computer}}{}{}
Is the current node last active computer.

\begin{notice}{note}{Note:}
This uses 10 seconds as timeout for definition ``not active''.
\end{notice}
\begin{quote}\begin{description}
\item[{Return type}] \leavevmode
Boolean

\end{description}\end{quote}

\end{fulllineitems}

\index{logger() (in module controller.computer)}

\begin{fulllineitems}
\phantomsection\label{controller:controller.computer.logger}\pysiglinewithargsret{\code{controller.computer.}\bfcode{logger}}{}{}
Return controller logger.

\end{fulllineitems}

\index{refresh\_computer() (in module controller.computer)}

\begin{fulllineitems}
\phantomsection\label{controller:controller.computer.refresh_computer}\pysiglinewithargsret{\code{controller.computer.}\bfcode{refresh\_computer}}{\emph{computer}}{}
Refresh the computer in database.
\begin{quote}\begin{description}
\item[{Parameters}] \leavevmode
\textbf{computer} ({\hyperref[models:models.Computer]{\code{models.Computer}}}) -- The computer to refresh.

\item[{Returns}] \leavevmode
The Refreshed computer.

\item[{Return type}] \leavevmode
{\hyperref[models:models.Computer]{\code{models.Computer}}}

\end{description}\end{quote}

\end{fulllineitems}

\index{refresh\_computer\_by\_wos\_id() (in module controller.computer)}

\begin{fulllineitems}
\phantomsection\label{controller:controller.computer.refresh_computer_by_wos_id}\pysiglinewithargsret{\code{controller.computer.}\bfcode{refresh\_computer\_by\_wos\_id}}{\emph{wos\_id}, \emph{new\_name=None}, \emph{new\_screens=None}, \emph{new\_responsive=None}}{}
Refresh the computer by node id and give it optionally new configurations.
\begin{quote}\begin{description}
\item[{Parameters}] \leavevmode\begin{itemize}
\item {} 
\textbf{wos\_id} (\emph{Integer}) -- The ID of the node to refresh.

\item {} 
\textbf{new\_name} (\emph{String}) -- Optional new name for the node.

\item {} 
\textbf{new\_screens} (\emph{Integer}) -- Optional new screens configuration for the node.

\item {} 
\textbf{new\_responsive} (\emph{Integer}) -- Optional new responsive setting for the node.

\end{itemize}

\item[{Returns}] \leavevmode
Success

\item[{Return type}] \leavevmode
Boolean

\end{description}\end{quote}

\end{fulllineitems}



\subsection{controller.handlers module}
\label{controller:module-controller.handlers}\label{controller:controller-handlers-module}\index{controller.handlers (module)}
Created on 28.6.2013
\begin{quote}\begin{description}
\item[{author}] \leavevmode
neriksso

\end{description}\end{quote}
\index{PROJECT\_FILE\_EVENT\_HANDLER (class in controller.handlers)}

\begin{fulllineitems}
\phantomsection\label{controller:controller.handlers.PROJECT_FILE_EVENT_HANDLER}\pysiglinewithargsret{\strong{class }\code{controller.handlers.}\bfcode{PROJECT\_FILE\_EVENT\_HANDLER}}{\emph{project\_id}}{}
Handler for FileSystem events on project folder.
\begin{quote}\begin{description}
\item[{Parameters}] \leavevmode
\textbf{project\_id} (\emph{Integer}) -- Project id from database.

\end{description}\end{quote}
\index{on\_created() (controller.handlers.PROJECT\_FILE\_EVENT\_HANDLER method)}

\begin{fulllineitems}
\phantomsection\label{controller:controller.handlers.PROJECT_FILE_EVENT_HANDLER.on_created}\pysiglinewithargsret{\bfcode{on\_created}}{\emph{event}}{}
On\_created event handler. Logs to database.
\begin{quote}\begin{description}
\item[{Parameters}] \leavevmode
\textbf{event} (an instance of \code{watchdog.events.FileSystemEvent}) -- The event.

\end{description}\end{quote}

\end{fulllineitems}

\index{on\_deleted() (controller.handlers.PROJECT\_FILE\_EVENT\_HANDLER method)}

\begin{fulllineitems}
\phantomsection\label{controller:controller.handlers.PROJECT_FILE_EVENT_HANDLER.on_deleted}\pysiglinewithargsret{\bfcode{on\_deleted}}{\emph{event}}{}
On\_deleted event handler. Logs to database.
\begin{quote}\begin{description}
\item[{Parameters}] \leavevmode
\textbf{event} (an instance of \code{watchdog.events.FileSystemEvent}) -- The event.

\end{description}\end{quote}

\end{fulllineitems}

\index{on\_modified() (controller.handlers.PROJECT\_FILE\_EVENT\_HANDLER method)}

\begin{fulllineitems}
\phantomsection\label{controller:controller.handlers.PROJECT_FILE_EVENT_HANDLER.on_modified}\pysiglinewithargsret{\bfcode{on\_modified}}{\emph{event}}{}
On\_modified event handler. Logs to database.
\begin{quote}\begin{description}
\item[{Parameters}] \leavevmode
\textbf{event} (an instance of \code{watchdog.events.FileSystemEvent}) -- The event.

\end{description}\end{quote}

\end{fulllineitems}


\end{fulllineitems}

\index{SCAN\_HANDLER (class in controller.handlers)}

\begin{fulllineitems}
\phantomsection\label{controller:controller.handlers.SCAN_HANDLER}\pysiglinewithargsret{\strong{class }\code{controller.handlers.}\bfcode{SCAN\_HANDLER}}{\emph{project\_id}}{}
Handler for FileSystem events on SCANNING folder.
\begin{quote}\begin{description}
\item[{Parameters}] \leavevmode
\textbf{project\_id} (\emph{Integer}) -- Project id from database.

\end{description}\end{quote}
\index{on\_created() (controller.handlers.SCAN\_HANDLER method)}

\begin{fulllineitems}
\phantomsection\label{controller:controller.handlers.SCAN_HANDLER.on_created}\pysiglinewithargsret{\bfcode{on\_created}}{\emph{event}}{}
On\_created event handler. Logs to database.
\begin{quote}\begin{description}
\item[{Parameters}] \leavevmode
\textbf{event} (an instance of \code{watchdog.events.FileSystemEvent}) -- The event.

\end{description}\end{quote}

\end{fulllineitems}


\end{fulllineitems}

\index{logger() (in module controller.handlers)}

\begin{fulllineitems}
\phantomsection\label{controller:controller.handlers.logger}\pysiglinewithargsret{\code{controller.handlers.}\bfcode{logger}}{}{}
Return controller logger.

\end{fulllineitems}



\subsection{controller.project module}
\label{controller:controller-project-module}\label{controller:module-controller.project}\index{controller.project (module)}
Created on 28.6.2013
\begin{quote}\begin{description}
\item[{author}] \leavevmode
neriksso

\end{description}\end{quote}
\index{add\_file\_to\_project() (in module controller.project)}

\begin{fulllineitems}
\phantomsection\label{controller:controller.project.add_file_to_project}\pysiglinewithargsret{\code{controller.project.}\bfcode{add\_file\_to\_project}}{\emph{filepath}, \emph{project\_id}}{}
Add a file to project. Copies it to the folder and adds a record to
database.
\begin{quote}\begin{description}
\item[{Parameters}] \leavevmode\begin{itemize}
\item {} 
\textbf{filepath} (\emph{String}) -- A filepath.

\item {} 
\textbf{project\_id} -- Project id from database.

\end{itemize}

\item[{Returns}] \leavevmode
New filepath.

\item[{Return type}] \leavevmode
String

\end{description}\end{quote}

\end{fulllineitems}

\index{add\_project() (in module controller.project)}

\begin{fulllineitems}
\phantomsection\label{controller:controller.project.add_project}\pysiglinewithargsret{\code{controller.project.}\bfcode{add\_project}}{\emph{data}}{}
Adds a project to database and returns a  project instance
\begin{quote}\begin{description}
\item[{Parameters}] \leavevmode
\textbf{data} (\emph{A dictionary}) -- Project information

\item[{Return type}] \leavevmode
an instance of {\hyperref[models:models.Project]{\code{models.Project}}}

\end{description}\end{quote}

\end{fulllineitems}

\index{check\_password() (in module controller.project)}

\begin{fulllineitems}
\phantomsection\label{controller:controller.project.check_password}\pysiglinewithargsret{\code{controller.project.}\bfcode{check\_password}}{\emph{project\_id}, \emph{password}}{}
Docstring here.

\end{fulllineitems}

\index{create\_file\_action() (in module controller.project)}

\begin{fulllineitems}
\phantomsection\label{controller:controller.project.create_file_action}\pysiglinewithargsret{\code{controller.project.}\bfcode{create\_file\_action}}{\emph{path}, \emph{action\_id}, \emph{session\_id}, \emph{project\_id}}{}
Logs a file action to the database.
\begin{quote}\begin{description}
\item[{Parameters}] \leavevmode\begin{itemize}
\item {} 
\textbf{path} (\emph{String}) -- Filepath.

\item {} 
\textbf{action\_id} (\emph{Integer}) -- File action id.

\item {} 
\textbf{session\_id} (\emph{Integer}) -- Current session id.

\item {} 
\textbf{project\_id} (\emph{Integer}) -- Project id from database.

\end{itemize}

\end{description}\end{quote}

\end{fulllineitems}

\index{edit\_project() (in module controller.project)}

\begin{fulllineitems}
\phantomsection\label{controller:controller.project.edit_project}\pysiglinewithargsret{\code{controller.project.}\bfcode{edit\_project}}{\emph{id\_number}, \emph{row}}{}
Update the project info.
\begin{quote}\begin{description}
\item[{Parameters}] \leavevmode\begin{itemize}
\item {} 
\textbf{id\_number} (\emph{Integer}) -- Database id number of the project.

\item {} 
\textbf{row} (\emph{A dictionary}) -- The new project information.

\end{itemize}

\end{description}\end{quote}

\end{fulllineitems}

\index{get\_active\_project() (in module controller.project)}

\begin{fulllineitems}
\phantomsection\label{controller:controller.project.get_active_project}\pysiglinewithargsret{\code{controller.project.}\bfcode{get\_active\_project}}{\emph{pgm\_group}}{}
Get the active project.
\begin{quote}\begin{description}
\item[{Parameters}] \leavevmode
\textbf{pgm\_group} (\emph{Integer}) -- The PGM Group number.

\item[{Returns}] \leavevmode
Active project ID.

\item[{Return type}] \leavevmode
Integer

\end{description}\end{quote}

\end{fulllineitems}

\index{get\_file\_path() (in module controller.project)}

\begin{fulllineitems}
\phantomsection\label{controller:controller.project.get_file_path}\pysiglinewithargsret{\code{controller.project.}\bfcode{get\_file\_path}}{\emph{project\_id}, \emph{filename}}{}
Returns the filepath for filename.
\begin{quote}\begin{description}
\item[{Returns}] \leavevmode
Filepath.

\item[{Return type}] \leavevmode
String

\end{description}\end{quote}

\end{fulllineitems}

\index{get\_project() (in module controller.project)}

\begin{fulllineitems}
\phantomsection\label{controller:controller.project.get_project}\pysiglinewithargsret{\code{controller.project.}\bfcode{get\_project}}{\emph{project\_id}}{}
Fetches projects by a company.
\begin{quote}\begin{description}
\item[{Parameters}] \leavevmode
\textbf{company\_id} (\emph{Integer}) -- A company id from database.

\end{description}\end{quote}

\end{fulllineitems}

\index{get\_project\_id\_by\_activity() (in module controller.project)}

\begin{fulllineitems}
\phantomsection\label{controller:controller.project.get_project_id_by_activity}\pysiglinewithargsret{\code{controller.project.}\bfcode{get\_project\_id\_by\_activity}}{\emph{activity\_id}}{}
Docstring here.

\end{fulllineitems}

\index{get\_project\_password() (in module controller.project)}

\begin{fulllineitems}
\phantomsection\label{controller:controller.project.get_project_password}\pysiglinewithargsret{\code{controller.project.}\bfcode{get\_project\_password}}{\emph{project\_id}}{}
Returns the project password.
\begin{quote}\begin{description}
\item[{Parameters}] \leavevmode
\textbf{project\_id} (\emph{Integer}) -- ID of the project.

\item[{Return type}] \leavevmode
String

\end{description}\end{quote}

\end{fulllineitems}

\index{get\_project\_path() (in module controller.project)}

\begin{fulllineitems}
\phantomsection\label{controller:controller.project.get_project_path}\pysiglinewithargsret{\code{controller.project.}\bfcode{get\_project\_path}}{\emph{project\_id}}{}
Fetches the project path from database and return it.
\begin{quote}\begin{description}
\item[{Parameters}] \leavevmode
\textbf{project\_id} (\emph{Integer}) -- Project id for database.

\item[{Return type}] \leavevmode
String

\end{description}\end{quote}

\end{fulllineitems}

\index{get\_projects\_by\_company() (in module controller.project)}

\begin{fulllineitems}
\phantomsection\label{controller:controller.project.get_projects_by_company}\pysiglinewithargsret{\code{controller.project.}\bfcode{get\_projects\_by\_company}}{\emph{company\_id}}{}
Fetches projects by a company.
\begin{quote}\begin{description}
\item[{Parameters}] \leavevmode
\textbf{company\_id} (\emph{Integer}) -- A company id from database.

\end{description}\end{quote}

\end{fulllineitems}

\index{get\_recent\_files() (in module controller.project)}

\begin{fulllineitems}
\phantomsection\label{controller:controller.project.get_recent_files}\pysiglinewithargsret{\code{controller.project.}\bfcode{get\_recent\_files}}{\emph{project\_id}, \emph{max\_files\_count=None}}{}
Fetches files accessed recently in the project sessions from the database.

\DUspan{}{New in version 0.9.3.0: }Added a limit parameter, limits the number of returned results.

\begin{notice}{note}{Note:}
Duplicate check has been added at some point in time.
\end{notice}
\begin{quote}\begin{description}
\item[{Parameters}] \leavevmode
\textbf{project\_id} (\emph{Integer}) -- The project id

\item[{Returns}] \leavevmode
The list of filepaths that have recently been used in this project.

\item[{Return type}] \leavevmode
List of String

\end{description}\end{quote}

\end{fulllineitems}

\index{init\_sync\_project\_directory() (in module controller.project)}

\begin{fulllineitems}
\phantomsection\label{controller:controller.project.init_sync_project_directory}\pysiglinewithargsret{\code{controller.project.}\bfcode{init\_sync\_project\_directory}}{\emph{project\_id}}{}
Initial sync of project dir and database.
\begin{quote}\begin{description}
\item[{Parameters}] \leavevmode
\textbf{project\_id} (\emph{Integer}) -- Project id from database.

\end{description}\end{quote}

\end{fulllineitems}

\index{is\_project\_file() (in module controller.project)}

\begin{fulllineitems}
\phantomsection\label{controller:controller.project.is_project_file}\pysiglinewithargsret{\code{controller.project.}\bfcode{is\_project\_file}}{\emph{filename}, \emph{project\_id}}{}
Checks, if a file belongs to a project. Checks both project folder
and database.
\begin{quote}\begin{description}
\item[{Parameters}] \leavevmode\begin{itemize}
\item {} 
\textbf{filename} (\emph{String}) -- a filepath.

\item {} 
\textbf{project\_id} (\emph{Integer}) -- Project id from database.

\end{itemize}

\item[{Return type}] \leavevmode
Boolean

\end{description}\end{quote}

\end{fulllineitems}

\index{logger() (in module controller.project)}

\begin{fulllineitems}
\phantomsection\label{controller:controller.project.logger}\pysiglinewithargsret{\code{controller.project.}\bfcode{logger}}{}{}
Return controller logger.

\end{fulllineitems}



\subsection{controller.session module}
\label{controller:module-controller.session}\label{controller:controller-session-module}\index{controller.session (module)}
Created on 28.6.2013
\begin{quote}\begin{description}
\item[{author}] \leavevmode
neriksso

\end{description}\end{quote}
\index{add\_event() (in module controller.session)}

\begin{fulllineitems}
\phantomsection\label{controller:controller.session.add_event}\pysiglinewithargsret{\code{controller.session.}\bfcode{add\_event}}{\emph{session\_id}, \emph{title}, \emph{description}}{}
Adds an event to the database.
\begin{quote}\begin{description}
\item[{Parameters}] \leavevmode\begin{itemize}
\item {} 
\textbf{session} ({\hyperref[models:models.Session]{\code{models.Session}}}) -- The current session.

\item {} 
\textbf{description} (\emph{String}) -- Description of the event.

\end{itemize}

\end{description}\end{quote}

\end{fulllineitems}

\index{end\_session() (in module controller.session)}

\begin{fulllineitems}
\phantomsection\label{controller:controller.session.end_session}\pysiglinewithargsret{\code{controller.session.}\bfcode{end\_session}}{\emph{session\_id}}{}
Ends a session, sets its endtime to database.
Ends file scanner.
\begin{quote}\begin{description}
\item[{Parameters}] \leavevmode
\textbf{session} ({\hyperref[models:models.Session]{\code{models.Session}}}) -- Current session.

\end{description}\end{quote}

\end{fulllineitems}

\index{get\_active\_session() (in module controller.session)}

\begin{fulllineitems}
\phantomsection\label{controller:controller.session.get_active_session}\pysiglinewithargsret{\code{controller.session.}\bfcode{get\_active\_session}}{\emph{pgm\_group}}{}
Get the active session.
\begin{quote}\begin{description}
\item[{Parameters}] \leavevmode
\textbf{pgm\_group} (\emph{Integer}) -- The PGM Group number.

\item[{Returns}] \leavevmode
The active session ID.

\item[{Return type}] \leavevmode
Integer

\end{description}\end{quote}

\end{fulllineitems}

\index{get\_latest\_event() (in module controller.session)}

\begin{fulllineitems}
\phantomsection\label{controller:controller.session.get_latest_event}\pysiglinewithargsret{\code{controller.session.}\bfcode{get\_latest\_event}}{}{}
Get the latest event id.
\begin{quote}\begin{description}
\item[{Returns}] \leavevmode
The ID of latest event.

\item[{Return type}] \leavevmode
Integer

\end{description}\end{quote}

\end{fulllineitems}

\index{get\_session\_id\_by\_activity() (in module controller.session)}

\begin{fulllineitems}
\phantomsection\label{controller:controller.session.get_session_id_by_activity}\pysiglinewithargsret{\code{controller.session.}\bfcode{get\_session\_id\_by\_activity}}{\emph{activity\_id}}{}
Docstring here.

\end{fulllineitems}

\index{get\_sessions\_by\_project() (in module controller.session)}

\begin{fulllineitems}
\phantomsection\label{controller:controller.session.get_sessions_by_project}\pysiglinewithargsret{\code{controller.session.}\bfcode{get\_sessions\_by\_project}}{\emph{project\_id}}{}
Fetches sessions for a project.
\begin{quote}\begin{description}
\item[{Parameters}] \leavevmode
\textbf{project\_id} (\emph{Integer}) -- Project id from database.

\end{description}\end{quote}

\end{fulllineitems}

\index{logger() (in module controller.session)}

\begin{fulllineitems}
\phantomsection\label{controller:controller.session.logger}\pysiglinewithargsret{\code{controller.session.}\bfcode{logger}}{}{}
Return controller logger.

\end{fulllineitems}

\index{start\_new\_session() (in module controller.session)}

\begin{fulllineitems}
\phantomsection\label{controller:controller.session.start_new_session}\pysiglinewithargsret{\code{controller.session.}\bfcode{start\_new\_session}}{\emph{project\_id}, \emph{session\_id=None}, \emph{old\_session\_id=None}}{}
Creates a session to the database and return a session object.
\begin{quote}\begin{description}
\item[{Parameters}] \leavevmode\begin{itemize}
\item {} 
\textbf{project\_id} (\emph{Integer}) -- Project id from database.

\item {} 
\textbf{session\_id} (\emph{Integer}) -- an existing session id from database.

\item {} 
\textbf{old\_session\_id} (\emph{Integer}) -- A session id of a session which will be continued.

\end{itemize}

\end{description}\end{quote}

\end{fulllineitems}



\section{Dialogs module}
\label{dialogs:dialogs-module}\label{dialogs::doc}\label{dialogs:module-dialogs}\index{dialogs (module)}
Created on 4.6.2013
\begin{quote}\begin{description}
\item[{author}] \leavevmode
neriksso

\end{description}\end{quote}
\index{AddProjectDialog (class in dialogs)}

\begin{fulllineitems}
\phantomsection\label{dialogs:dialogs.AddProjectDialog}\pysiglinewithargsret{\strong{class }\code{dialogs.}\bfcode{AddProjectDialog}}{\emph{parent}, \emph{title}, \emph{project\_id=None}}{}
A dialog for adding a new project
\begin{quote}\begin{description}
\item[{Parameters}] \leavevmode\begin{itemize}
\item {} 
\textbf{parent} (\code{wx.Frame}) -- Parent frame.

\item {} 
\textbf{title} (\emph{String}) -- A title for the dialog.

\end{itemize}

\end{description}\end{quote}
\index{OnAdd() (dialogs.AddProjectDialog method)}

\begin{fulllineitems}
\phantomsection\label{dialogs:dialogs.AddProjectDialog.OnAdd}\pysiglinewithargsret{\bfcode{OnAdd}}{\emph{event}}{}
Handles the addition of a project to database, when ``Add'' button
is pressed.
\begin{quote}\begin{description}
\item[{Parameters}] \leavevmode
\textbf{event} (\emph{Event}) -- GUI Event.

\end{description}\end{quote}

\end{fulllineitems}

\index{OnClose() (dialogs.AddProjectDialog method)}

\begin{fulllineitems}
\phantomsection\label{dialogs:dialogs.AddProjectDialog.OnClose}\pysiglinewithargsret{\bfcode{OnClose}}{\emph{event}}{}
Handles ``Close'' button presses.
\begin{quote}\begin{description}
\item[{Parameters}] \leavevmode
\textbf{event} (\emph{Event}) -- GUI Event.

\end{description}\end{quote}

\end{fulllineitems}

\index{OnText() (dialogs.AddProjectDialog method)}

\begin{fulllineitems}
\phantomsection\label{dialogs:dialogs.AddProjectDialog.OnText}\pysiglinewithargsret{\bfcode{OnText}}{\emph{event}}{}
Event handler for text changed.

\end{fulllineitems}


\end{fulllineitems}

\index{CloseError}

\begin{fulllineitems}
\phantomsection\label{dialogs:dialogs.CloseError}\pysiglinewithargsret{\strong{exception }\code{dialogs.}\bfcode{CloseError}}{\emph{*args}, \emph{**kwds}}{}
Class describing an error while closing application.

\end{fulllineitems}

\index{ConnectionErrorDialog (class in dialogs)}

\begin{fulllineitems}
\phantomsection\label{dialogs:dialogs.ConnectionErrorDialog}\pysiglinewithargsret{\strong{class }\code{dialogs.}\bfcode{ConnectionErrorDialog}}{\emph{parent}}{}
Create a connection error dialog that informs the user about reconnection
attempts made by the software.

\end{fulllineitems}

\index{DeleteProjectDialog (class in dialogs)}

\begin{fulllineitems}
\phantomsection\label{dialogs:dialogs.DeleteProjectDialog}\pysiglinewithargsret{\strong{class }\code{dialogs.}\bfcode{DeleteProjectDialog}}{\emph{parent}, \emph{title}, \emph{project\_id}}{}
A dialog for deleting project.
\index{OnCancel() (dialogs.DeleteProjectDialog method)}

\begin{fulllineitems}
\phantomsection\label{dialogs:dialogs.DeleteProjectDialog.OnCancel}\pysiglinewithargsret{\bfcode{OnCancel}}{\emph{event}}{}
Event handler for pressing Cancel button.

\end{fulllineitems}

\index{OnOk() (dialogs.DeleteProjectDialog method)}

\begin{fulllineitems}
\phantomsection\label{dialogs:dialogs.DeleteProjectDialog.OnOk}\pysiglinewithargsret{\bfcode{OnOk}}{\emph{event}}{}
Event handler for pressing OK button.

\end{fulllineitems}


\end{fulllineitems}

\index{ErrorDialog (class in dialogs)}

\begin{fulllineitems}
\phantomsection\label{dialogs:dialogs.ErrorDialog}\pysiglinewithargsret{\strong{class }\code{dialogs.}\bfcode{ErrorDialog}}{\emph{parent}, \emph{message}}{}
Error dialog.

\end{fulllineitems}

\index{PreferencesDialog (class in dialogs)}

\begin{fulllineitems}
\phantomsection\label{dialogs:dialogs.PreferencesDialog}\pysiglinewithargsret{\strong{class }\code{dialogs.}\bfcode{PreferencesDialog}}{\emph{parent}, \emph{config\_object}}{}
Creates and displays a preferences dialog that allows the user to
change some settings.
\begin{quote}\begin{description}
\item[{Parameters}] \leavevmode
\textbf{config\_object} (\code{configobj.ConfigObj}) -- a Config object

\end{description}\end{quote}
\index{LoadPreferences() (dialogs.PreferencesDialog method)}

\begin{fulllineitems}
\phantomsection\label{dialogs:dialogs.PreferencesDialog.LoadPreferences}\pysiglinewithargsret{\bfcode{LoadPreferences}}{}{}
Load the current preferences and fills the text controls.

\end{fulllineitems}

\index{OnCancel() (dialogs.PreferencesDialog method)}

\begin{fulllineitems}
\phantomsection\label{dialogs:dialogs.PreferencesDialog.OnCancel}\pysiglinewithargsret{\bfcode{OnCancel}}{\emph{event}}{}
Closes the dialog without modifications.
\begin{quote}\begin{description}
\item[{Parameters}] \leavevmode
\textbf{event} (\emph{Event}) -- GUI event.

\end{description}\end{quote}

\end{fulllineitems}

\index{OpenConfig() (dialogs.PreferencesDialog method)}

\begin{fulllineitems}
\phantomsection\label{dialogs:dialogs.PreferencesDialog.OpenConfig}\pysiglinewithargsret{\bfcode{OpenConfig}}{\emph{event}}{}
Opens config file.
\begin{quote}\begin{description}
\item[{Parameters}] \leavevmode
\textbf{event} (\emph{Event}) -- GUI event.

\end{description}\end{quote}

\end{fulllineitems}

\index{SavePreferences() (dialogs.PreferencesDialog method)}

\begin{fulllineitems}
\phantomsection\label{dialogs:dialogs.PreferencesDialog.SavePreferences}\pysiglinewithargsret{\bfcode{SavePreferences}}{\emph{event}}{}
Save the preferences.
\begin{quote}\begin{description}
\item[{Parameters}] \leavevmode
\textbf{event} (\emph{Event}) -- GUI Event.

\end{description}\end{quote}

\end{fulllineitems}


\end{fulllineitems}

\index{ProjectAuthenticationDialog (class in dialogs)}

\begin{fulllineitems}
\phantomsection\label{dialogs:dialogs.ProjectAuthenticationDialog}\pysiglinewithargsret{\strong{class }\code{dialogs.}\bfcode{ProjectAuthenticationDialog}}{\emph{parent}, \emph{title}, \emph{project\_id}}{}
A dialog for project authentication.
\index{OnOk() (dialogs.ProjectAuthenticationDialog method)}

\begin{fulllineitems}
\phantomsection\label{dialogs:dialogs.ProjectAuthenticationDialog.OnOk}\pysiglinewithargsret{\bfcode{OnOk}}{\emph{event}}{}
Called on OK button press.

\end{fulllineitems}


\end{fulllineitems}

\index{ProjectSelectDialog (class in dialogs)}

\begin{fulllineitems}
\phantomsection\label{dialogs:dialogs.ProjectSelectDialog}\pysiglinewithargsret{\strong{class }\code{dialogs.}\bfcode{ProjectSelectDialog}}{\emph{parent}}{}
A dialog for selecting a project.
\begin{quote}\begin{description}
\item[{Parameters}] \leavevmode
\textbf{parent} (\code{wx.Frame}) -- Parent frame.

\end{description}\end{quote}
\index{OnCancel() (dialogs.ProjectSelectDialog method)}

\begin{fulllineitems}
\phantomsection\label{dialogs:dialogs.ProjectSelectDialog.OnCancel}\pysiglinewithargsret{\bfcode{OnCancel}}{\emph{event}}{}
Handles ``Cancel'' button presses.
\begin{quote}\begin{description}
\item[{Parameters}] \leavevmode
\textbf{event} (\emph{Event}) -- GUI Event.

\end{description}\end{quote}

\end{fulllineitems}

\index{OnProjectAdd() (dialogs.ProjectSelectDialog method)}

\begin{fulllineitems}
\phantomsection\label{dialogs:dialogs.ProjectSelectDialog.OnProjectAdd}\pysiglinewithargsret{\bfcode{OnProjectAdd}}{\emph{event}}{}
Shows a modal dialog for adding a new project.
\begin{quote}\begin{description}
\item[{Parameters}] \leavevmode
\textbf{event} (\emph{Event}) -- GUI Event.

\end{description}\end{quote}

\end{fulllineitems}

\index{OnProjectDelete() (dialogs.ProjectSelectDialog method)}

\begin{fulllineitems}
\phantomsection\label{dialogs:dialogs.ProjectSelectDialog.OnProjectDelete}\pysiglinewithargsret{\bfcode{OnProjectDelete}}{\emph{event}}{}
Handles the selection of a project.
Starts a \code{wos.CURRENT\_PROJECT}, if necessary.
Shows a dialog of the selected project.
\begin{quote}\begin{description}
\item[{Parameters}] \leavevmode
\textbf{evt} (\emph{Event}) -- GUI Event.

\end{description}\end{quote}

\end{fulllineitems}

\index{OnProjectEdit() (dialogs.ProjectSelectDialog method)}

\begin{fulllineitems}
\phantomsection\label{dialogs:dialogs.ProjectSelectDialog.OnProjectEdit}\pysiglinewithargsret{\bfcode{OnProjectEdit}}{\emph{event}}{}
Shows a modal dialog for adding a new project.
\begin{quote}\begin{description}
\item[{Parameters}] \leavevmode
\textbf{event} (\emph{Event}) -- GUI Event.

\end{description}\end{quote}

\end{fulllineitems}

\index{OnProjectSelect() (dialogs.ProjectSelectDialog method)}

\begin{fulllineitems}
\phantomsection\label{dialogs:dialogs.ProjectSelectDialog.OnProjectSelect}\pysiglinewithargsret{\bfcode{OnProjectSelect}}{\emph{event}}{}
Handles the selection of a project.

Starts a \code{wos.CURRENT\_PROJECT}, if necessary.
Shows a dialog of the selected project.
\begin{quote}\begin{description}
\item[{Parameters}] \leavevmode
\textbf{event} (\emph{Event}) -- GUI Event.

\end{description}\end{quote}

\end{fulllineitems}

\index{OnSelectionChange() (dialogs.ProjectSelectDialog method)}

\begin{fulllineitems}
\phantomsection\label{dialogs:dialogs.ProjectSelectDialog.OnSelectionChange}\pysiglinewithargsret{\bfcode{OnSelectionChange}}{\emph{event}}{}
Event handler for selection change of the listbox.

\end{fulllineitems}

\index{UpdateProjects() (dialogs.ProjectSelectDialog method)}

\begin{fulllineitems}
\phantomsection\label{dialogs:dialogs.ProjectSelectDialog.UpdateProjects}\pysiglinewithargsret{\bfcode{UpdateProjects}}{\emph{company\_id=1}}{}
Fetches all projects from the database, based on the company.
\begin{quote}\begin{description}
\item[{Parameters}] \leavevmode
\textbf{company\_id} (\emph{Integer}) -- A company id, the owner of the projects.

\item[{Returns}] \leavevmode
The total number of projects.

\item[{Type }] \leavevmode
Integer

\end{description}\end{quote}

\end{fulllineitems}


\end{fulllineitems}

\index{ProjectSelectedDialog (class in dialogs)}

\begin{fulllineitems}
\phantomsection\label{dialogs:dialogs.ProjectSelectedDialog}\pysiglinewithargsret{\strong{class }\code{dialogs.}\bfcode{ProjectSelectedDialog}}{\emph{parent}, \emph{title}, \emph{project\_id}}{}
A dialog for project selection confirmation.

\end{fulllineitems}

\index{SendProgressBar (class in dialogs)}

\begin{fulllineitems}
\phantomsection\label{dialogs:dialogs.SendProgressBar}\pysiglinewithargsret{\strong{class }\code{dialogs.}\bfcode{SendProgressBar}}{\emph{parent}, \emph{title}, \emph{ypos}}{}
Implements file send progress bar...

\end{fulllineitems}

\index{UpdateDialog (class in dialogs)}

\begin{fulllineitems}
\phantomsection\label{dialogs:dialogs.UpdateDialog}\pysiglinewithargsret{\strong{class }\code{dialogs.}\bfcode{UpdateDialog}}{\emph{title}, \emph{url}, \emph{*args}, \emph{**kwargs}}{}
A Dialog which notifies about a software update.
Contains the URL which the user can click on.
\begin{quote}\begin{description}
\item[{Parameters}] \leavevmode\begin{itemize}
\item {} 
\textbf{title} (\emph{String}) -- Title of the dialog.

\item {} 
\textbf{url} (\emph{String}) -- URL of the update.

\end{itemize}

\end{description}\end{quote}

\end{fulllineitems}

\index{set\_logger\_level() (in module dialogs)}

\begin{fulllineitems}
\phantomsection\label{dialogs:dialogs.set_logger_level}\pysiglinewithargsret{\code{dialogs.}\bfcode{set\_logger\_level}}{\emph{level}}{}
Sets the logger level for dialogs logger.
\begin{quote}\begin{description}
\item[{Parameters}] \leavevmode
\textbf{level} (\emph{Integer}) -- Level of logging.

\end{description}\end{quote}

\end{fulllineitems}

\index{show\_modal\_and\_destroy() (in module dialogs)}

\begin{fulllineitems}
\phantomsection\label{dialogs:dialogs.show_modal_and_destroy}\pysiglinewithargsret{\code{dialogs.}\bfcode{show\_modal\_and\_destroy}}{\emph{class\_}, \emph{parent}, \emph{params=None}}{}
Used to show modal and destroy afterwards.

\begin{notice}{note}{Note:}
The implementation is kind of ugly, but guarantees a safe execution
of the dialog without memory leaks and with all exceptions logged.
\end{notice}
\begin{quote}\begin{description}
\item[{Parameters}] \leavevmode\begin{itemize}
\item {} 
\textbf{class} (\emph{type}) -- The type of dialog to show.

\item {} 
\textbf{parent} (\code{wx.Window}) -- The parent wx.Window of this object.

\item {} 
\textbf{params} (\emph{Dictionary.}) -- The params to give for \_\_init\_\_ call.

\end{itemize}

\item[{Returns}] \leavevmode
The modal result value.

\item[{Return type}] \leavevmode
Integer

\end{description}\end{quote}

\end{fulllineitems}



\section{DiWaCS module}
\label{diwacs:diwacs-module}\label{diwacs:module-diwacs}\label{diwacs::doc}\index{diwacs (module)}
Created on 8.5.2012
\begin{quote}\begin{description}
\item[{author}] \leavevmode
neriksso

\end{description}\end{quote}
\index{EventList (class in diwacs)}

\begin{fulllineitems}
\phantomsection\label{diwacs:diwacs.EventList}\pysiglinewithargsret{\strong{class }\code{diwacs.}\bfcode{EventList}}{\emph{parent}, \emph{*args}, \emph{**kwargs}}{}
A Frame which displays the possible event titles and handles the event
creation.
\index{CheckVisibility() (diwacs.EventList method)}

\begin{fulllineitems}
\phantomsection\label{diwacs:diwacs.EventList.CheckVisibility}\pysiglinewithargsret{\bfcode{CheckVisibility}}{\emph{selection}}{}
Checks the visibility.

\end{fulllineitems}

\index{HideNow() (diwacs.EventList method)}

\begin{fulllineitems}
\phantomsection\label{diwacs:diwacs.EventList.HideNow}\pysiglinewithargsret{\bfcode{HideNow}}{}{}
Method to hide the event list.

\end{fulllineitems}

\index{OnEnter() (diwacs.EventList method)}

\begin{fulllineitems}
\phantomsection\label{diwacs:diwacs.EventList.OnEnter}\pysiglinewithargsret{\bfcode{OnEnter}}{\emph{event}}{}
Event handler for pressing ENTER button.
\begin{quote}\begin{description}
\item[{Parameters}] \leavevmode
\textbf{event} (\code{wx.Event}) -- The EVT\_ON\_TEXT\_EVENT event.

\end{description}\end{quote}

\end{fulllineitems}

\index{OnFocusLost() (diwacs.EventList method)}

\begin{fulllineitems}
\phantomsection\label{diwacs:diwacs.EventList.OnFocusLost}\pysiglinewithargsret{\bfcode{OnFocusLost}}{\emph{event}}{}
On focus lost event handler.

\end{fulllineitems}

\index{OnSelection() (diwacs.EventList method)}

\begin{fulllineitems}
\phantomsection\label{diwacs:diwacs.EventList.OnSelection}\pysiglinewithargsret{\bfcode{OnSelection}}{\emph{event}}{}
On selection event handler.

\end{fulllineitems}

\index{OnText() (diwacs.EventList method)}

\begin{fulllineitems}
\phantomsection\label{diwacs:diwacs.EventList.OnText}\pysiglinewithargsret{\bfcode{OnText}}{\emph{event}}{}
On text event handler.

\end{fulllineitems}

\index{ShowNow() (diwacs.EventList method)}

\begin{fulllineitems}
\phantomsection\label{diwacs:diwacs.EventList.ShowNow}\pysiglinewithargsret{\bfcode{ShowNow}}{}{}
Method to show the event list.

\end{fulllineitems}


\end{fulllineitems}

\index{GraphicalUserInterface (class in diwacs)}

\begin{fulllineitems}
\phantomsection\label{diwacs:diwacs.GraphicalUserInterface}\pysigline{\strong{class }\code{diwacs.}\bfcode{GraphicalUserInterface}}
WOS Application Frame.
\index{DisableDirectoryButton() (diwacs.GraphicalUserInterface method)}

\begin{fulllineitems}
\phantomsection\label{diwacs:diwacs.GraphicalUserInterface.DisableDirectoryButton}\pysiglinewithargsret{\bfcode{DisableDirectoryButton}}{}{}
Used to disable the project directory button when project has been
unselected.

\begin{notice}{note}{Note:}
There should be no need for this as the software should
always start a new project after the old one ends.
But for the mid state to be legimate this is still
usable.
\end{notice}

\end{fulllineitems}

\index{DisableSessionButton() (diwacs.GraphicalUserInterface method)}

\begin{fulllineitems}
\phantomsection\label{diwacs:diwacs.GraphicalUserInterface.DisableSessionButton}\pysiglinewithargsret{\bfcode{DisableSessionButton}}{}{}
Used to disable the needed buttons after session has been stopped.

\begin{notice}{note}{Note:}
Does not actually disable to session button, only the session
state of the button.
\end{notice}

\end{fulllineitems}

\index{EnableDirectoryButton() (diwacs.GraphicalUserInterface method)}

\begin{fulllineitems}
\phantomsection\label{diwacs:diwacs.GraphicalUserInterface.EnableDirectoryButton}\pysiglinewithargsret{\bfcode{EnableDirectoryButton}}{}{}
Used to enable the project directory button when project has been
selected.

\end{fulllineitems}

\index{EnableSessionButton() (diwacs.GraphicalUserInterface method)}

\begin{fulllineitems}
\phantomsection\label{diwacs:diwacs.GraphicalUserInterface.EnableSessionButton}\pysiglinewithargsret{\bfcode{EnableSessionButton}}{}{}
Used to enable the needed buttons after session has been started.

\end{fulllineitems}

\index{GetNodeByName() (diwacs.GraphicalUserInterface method)}

\begin{fulllineitems}
\phantomsection\label{diwacs:diwacs.GraphicalUserInterface.GetNodeByName}\pysiglinewithargsret{\bfcode{GetNodeByName}}{\emph{name}}{}
From current session nodes, select a node with this name
or return None.
\begin{quote}\begin{description}
\item[{Parameters}] \leavevmode
\textbf{name} (\emph{String}) -- Name of the desired node.

\item[{Returns}] \leavevmode
The desired node if one exists.

\item[{Return type}] \leavevmode
{\hyperref[swnp:swnp.Node]{\code{swnp.Node}}}

\end{description}\end{quote}

\end{fulllineitems}

\index{InitUICore() (diwacs.GraphicalUserInterface method)}

\begin{fulllineitems}
\phantomsection\label{diwacs:diwacs.GraphicalUserInterface.InitUICore}\pysiglinewithargsret{\bfcode{InitUICore}}{}{}
Inits the Core UI (\code{guitemplates.GUItemplate.InitUI()}) and
binds the functionality.

\end{fulllineitems}

\index{OnAboutBox() (diwacs.GraphicalUserInterface method)}

\begin{fulllineitems}
\phantomsection\label{diwacs:diwacs.GraphicalUserInterface.OnAboutBox}\pysiglinewithargsret{\bfcode{OnAboutBox}}{\emph{event}}{}
About dialog.
\begin{quote}\begin{description}
\item[{Parameters}] \leavevmode
\textbf{e} (\emph{Event}) -- GraphicalUserInterface Event.

\end{description}\end{quote}

\end{fulllineitems}

\index{OnEvtBtn() (diwacs.GraphicalUserInterface method)}

\begin{fulllineitems}
\phantomsection\label{diwacs:diwacs.GraphicalUserInterface.OnEvtBtn}\pysiglinewithargsret{\bfcode{OnEvtBtn}}{\emph{event}}{}
Event Button handler.
\begin{quote}\begin{description}
\item[{Parameters}] \leavevmode
\textbf{event} (\emph{Event}) -- GraphicalUserInterface Event.

\end{description}\end{quote}

\end{fulllineitems}

\index{OnExit() (diwacs.GraphicalUserInterface method)}

\begin{fulllineitems}
\phantomsection\label{diwacs:diwacs.GraphicalUserInterface.OnExit}\pysiglinewithargsret{\bfcode{OnExit}}{\emph{event}}{}
Exits program.
\begin{quote}\begin{description}
\item[{Parameters}] \leavevmode
\textbf{event} (\emph{Event}) -- GraphicalUserInterface Event

\end{description}\end{quote}

\end{fulllineitems}

\index{OnIconify() (diwacs.GraphicalUserInterface method)}

\begin{fulllineitems}
\phantomsection\label{diwacs:diwacs.GraphicalUserInterface.OnIconify}\pysiglinewithargsret{\bfcode{OnIconify}}{\emph{event}}{}
Window minimize event handler.
Should toggle the minimized state of the application.
\begin{quote}\begin{description}
\item[{Parameters}] \leavevmode
\textbf{evt} (\emph{Event}) -- GraphicalUserInterface Event.

\end{description}\end{quote}

\end{fulllineitems}

\index{OnInfoBtn() (diwacs.GraphicalUserInterface method)}

\begin{fulllineitems}
\phantomsection\label{diwacs:diwacs.GraphicalUserInterface.OnInfoBtn}\pysiglinewithargsret{\bfcode{OnInfoBtn}}{\emph{event}}{}
Handles the pressing of Web-information button.

Directs the user to web-storage website/help.

\end{fulllineitems}

\index{OnMBBtn() (diwacs.GraphicalUserInterface method)}

\begin{fulllineitems}
\phantomsection\label{diwacs:diwacs.GraphicalUserInterface.OnMBBtn}\pysiglinewithargsret{\bfcode{OnMBBtn}}{\emph{event}}{}
Handles the pressing of meetings browser button.

Directs the user to web-storage website/mb.

\end{fulllineitems}

\index{OnPreferences() (diwacs.GraphicalUserInterface method)}

\begin{fulllineitems}
\phantomsection\label{diwacs:diwacs.GraphicalUserInterface.OnPreferences}\pysiglinewithargsret{\bfcode{OnPreferences}}{\emph{event}}{}
Preferences dialog event handler.
\begin{quote}\begin{description}
\item[{Parameters}] \leavevmode
\textbf{event} (\emph{Event}) -- GraphicalUserInterface Event.

\end{description}\end{quote}

\end{fulllineitems}

\index{OnProject() (diwacs.GraphicalUserInterface method)}

\begin{fulllineitems}
\phantomsection\label{diwacs:diwacs.GraphicalUserInterface.OnProject}\pysiglinewithargsret{\bfcode{OnProject}}{}{}
Project selected event handler.

\end{fulllineitems}

\index{OnSession() (diwacs.GraphicalUserInterface method)}

\begin{fulllineitems}
\phantomsection\label{diwacs:diwacs.GraphicalUserInterface.OnSession}\pysiglinewithargsret{\bfcode{OnSession}}{\emph{event}}{}
Session button pressed.

The user either desires to start a new session or end
an existing one.
\begin{quote}\begin{description}
\item[{Parameters}] \leavevmode
\textbf{event} (\emph{Event}) -- GraphicalUserInterface Event.

\end{description}\end{quote}

\end{fulllineitems}

\index{OnTaskBarActivate() (diwacs.GraphicalUserInterface method)}

\begin{fulllineitems}
\phantomsection\label{diwacs:diwacs.GraphicalUserInterface.OnTaskBarActivate}\pysiglinewithargsret{\bfcode{OnTaskBarActivate}}{\emph{event}}{}
Taskbar activate event handler.
\begin{quote}\begin{description}
\item[{Parameters}] \leavevmode
\textbf{event} (\emph{Event}) -- GraphicalUserInterface Event.

\end{description}\end{quote}

\end{fulllineitems}

\index{OnTaskBarClose() (diwacs.GraphicalUserInterface method)}

\begin{fulllineitems}
\phantomsection\label{diwacs:diwacs.GraphicalUserInterface.OnTaskBarClose}\pysiglinewithargsret{\bfcode{OnTaskBarClose}}{\emph{unused\_event}}{}
Taskbar close event handler.
\begin{quote}\begin{description}
\item[{Parameters}] \leavevmode
\textbf{evt} (\emph{Event}) -- GraphicalUserInterface Event.

\end{description}\end{quote}

\end{fulllineitems}

\index{OnWABtn() (diwacs.GraphicalUserInterface method)}

\begin{fulllineitems}
\phantomsection\label{diwacs:diwacs.GraphicalUserInterface.OnWABtn}\pysiglinewithargsret{\bfcode{OnWABtn}}{\emph{event}}{}
Handles the pressing of Web-application button.

Directs the user to web-storage website.

\end{fulllineitems}

\index{OpenProjectDir() (diwacs.GraphicalUserInterface method)}

\begin{fulllineitems}
\phantomsection\label{diwacs:diwacs.GraphicalUserInterface.OpenProjectDir}\pysiglinewithargsret{\bfcode{OpenProjectDir}}{\emph{event}}{}
Opens project directory in windows explorer.
\begin{quote}\begin{description}
\item[{Parameters}] \leavevmode
\textbf{event} (\emph{Event}) -- The GraphicalUserInterface event.

\end{description}\end{quote}

\end{fulllineitems}

\index{PaintSelect() (diwacs.GraphicalUserInterface method)}

\begin{fulllineitems}
\phantomsection\label{diwacs:diwacs.GraphicalUserInterface.PaintSelect}\pysiglinewithargsret{\bfcode{PaintSelect}}{\emph{evt}}{}
Paints the selection of a node.

\begin{notice}{note}{Note:}
For future use.
\end{notice}
\begin{quote}\begin{description}
\item[{Parameters}] \leavevmode
\textbf{evt} (\emph{Event}) -- GraphicalUserInterface Event

\end{description}\end{quote}

\end{fulllineitems}

\index{SelectNode() (diwacs.GraphicalUserInterface method)}

\begin{fulllineitems}
\phantomsection\label{diwacs:diwacs.GraphicalUserInterface.SelectNode}\pysiglinewithargsret{\bfcode{SelectNode}}{\emph{event}}{}
Handles the selection of a node, start remote control.

\begin{notice}{note}{Note:}
For future use.
\end{notice}
\begin{quote}\begin{description}
\item[{Parameters}] \leavevmode
\textbf{event} (\emph{Event}) -- GraphicalUserInterface Event

\end{description}\end{quote}

\end{fulllineitems}

\index{SelectProjectDialog() (diwacs.GraphicalUserInterface method)}

\begin{fulllineitems}
\phantomsection\label{diwacs:diwacs.GraphicalUserInterface.SelectProjectDialog}\pysiglinewithargsret{\bfcode{SelectProjectDialog}}{\emph{event}}{}
Select project event handler.
\begin{quote}\begin{description}
\item[{Parameters}] \leavevmode
\textbf{event} (\emph{Event}) -- GraphicalUserInterface Event.

\end{description}\end{quote}

\end{fulllineitems}

\index{SetProjectName() (diwacs.GraphicalUserInterface method)}

\begin{fulllineitems}
\phantomsection\label{diwacs:diwacs.GraphicalUserInterface.SetProjectName}\pysiglinewithargsret{\bfcode{SetProjectName}}{\emph{name}}{}
Set the project text.
For example ``No Project OnSelection''.

\begin{notice}{note}{Note:}
Requires None explicitly when the purpose is to set default label
because writing SetProjectName(None) is more informative than
SetProjectName()
\end{notice}
\begin{quote}\begin{description}
\item[{Parameters}] \leavevmode
\textbf{name} (\emph{String}) -- The name of the project to set as label.

\end{description}\end{quote}

\end{fulllineitems}

\index{Shift() (diwacs.GraphicalUserInterface method)}

\begin{fulllineitems}
\phantomsection\label{diwacs:diwacs.GraphicalUserInterface.Shift}\pysiglinewithargsret{\bfcode{Shift}}{\emph{event}}{}
Caroussel Shift function.
\begin{quote}\begin{description}
\item[{Parameters}] \leavevmode
\textbf{event} (\emph{Event}) -- GraphicalUserInterface Event.

\end{description}\end{quote}

\end{fulllineitems}

\index{UpdateScreens() (diwacs.GraphicalUserInterface method)}

\begin{fulllineitems}
\phantomsection\label{diwacs:diwacs.GraphicalUserInterface.UpdateScreens}\pysiglinewithargsret{\bfcode{UpdateScreens}}{\emph{update}}{}
Called when screens need to be updated and redrawn.
\begin{quote}\begin{description}
\item[{Parameters}] \leavevmode
\textbf{update} (\emph{Boolean}) -- Pubsub needs one param, therefore it is called update.

\end{description}\end{quote}

\end{fulllineitems}


\end{fulllineitems}

\index{main() (in module diwacs)}

\begin{fulllineitems}
\phantomsection\label{diwacs:diwacs.main}\pysiglinewithargsret{\code{diwacs.}\bfcode{main}}{\emph{profile}}{}
Main function.
\begin{quote}\begin{description}
\item[{Warning }] \leavevmode
The profiler has been pre-calibrated using the development machine
so this should be changed for other development environments that
wish to profile the execution of the diwacs system.

THIS ONLY WORKS WHEN diwavars.DEBUG HAS BEEN ENABLED.

Remember to disable it from release binaries.

\item[{Parameters}] \leavevmode
\textbf{profile} (\emph{Boolean}) -- should the call be profiled?

\end{description}\end{quote}

\end{fulllineitems}

\index{set\_logger\_level() (in module diwacs)}

\begin{fulllineitems}
\phantomsection\label{diwacs:diwacs.set_logger_level}\pysiglinewithargsret{\code{diwacs.}\bfcode{set\_logger\_level}}{\emph{level}}{}
Used to set logger level.
\begin{quote}\begin{description}
\item[{Parameters}] \leavevmode
\textbf{level} (\emph{Integer}) -- The level desired.

\end{description}\end{quote}

\end{fulllineitems}



\section{DiWaVars module}
\label{diwavars:module-diwavars}\label{diwavars:diwavars-module}\label{diwavars::doc}\index{diwavars (module)}
DiWaCS Variables
\index{add\_logger\_initializer() (in module diwavars)}

\begin{fulllineitems}
\phantomsection\label{diwavars:diwavars.add_logger_initializer}\pysiglinewithargsret{\code{diwavars.}\bfcode{add\_logger\_initializer}}{\emph{logger\_initializer}}{}
For initializing the loggers from main.
\begin{quote}\begin{description}
\item[{Parameters}] \leavevmode
\textbf{logger\_initializer} (\emph{function}) -- The logger initializer to add to initialize chain.

\end{description}\end{quote}

\end{fulllineitems}

\index{add\_logger\_level\_setter() (in module diwavars)}

\begin{fulllineitems}
\phantomsection\label{diwavars:diwavars.add_logger_level_setter}\pysiglinewithargsret{\code{diwavars.}\bfcode{add\_logger\_level\_setter}}{\emph{logger\_level\_setter}}{}
For setting application logger level globally.
\begin{quote}\begin{description}
\item[{Parameters}] \leavevmode
\textbf{logger\_level\_setter} (\emph{function}) -- The logger level setter to add to level set chain.

\end{description}\end{quote}

\end{fulllineitems}

\index{set\_blank\_cursor() (in module diwavars)}

\begin{fulllineitems}
\phantomsection\label{diwavars:diwavars.set_blank_cursor}\pysiglinewithargsret{\code{diwavars.}\bfcode{set\_blank\_cursor}}{\emph{value}}{}
Set the blank cursor variable.

\end{fulllineitems}

\index{set\_config() (in module diwavars)}

\begin{fulllineitems}
\phantomsection\label{diwavars:diwavars.set_config}\pysiglinewithargsret{\code{diwavars.}\bfcode{set\_config}}{\emph{config}}{}
Set the CONFIG global...

\end{fulllineitems}

\index{set\_default\_cursor() (in module diwavars)}

\begin{fulllineitems}
\phantomsection\label{diwavars:diwavars.set_default_cursor}\pysiglinewithargsret{\code{diwavars.}\bfcode{set\_default\_cursor}}{\emph{value}}{}
Set the default cursor variable.

\end{fulllineitems}

\index{set\_run\_cmd() (in module diwavars)}

\begin{fulllineitems}
\phantomsection\label{diwavars:diwavars.set_run_cmd}\pysiglinewithargsret{\code{diwavars.}\bfcode{set\_run\_cmd}}{\emph{value}}{}
Update the RUN\_CMD setting.
\begin{quote}\begin{description}
\item[{Parameters}] \leavevmode
\textbf{value} (\emph{Boolean}) -- Desired value.

\end{description}\end{quote}

\end{fulllineitems}

\index{set\_running() (in module diwavars)}

\begin{fulllineitems}
\phantomsection\label{diwavars:diwavars.set_running}\pysiglinewithargsret{\code{diwavars.}\bfcode{set\_running}}{}{}
Set the currently running flag as true.

Causes other modules to redirect their stdout and stderr streams
to files.

\end{fulllineitems}

\index{update\_PGM\_group() (in module diwavars)}

\begin{fulllineitems}
\phantomsection\label{diwavars:diwavars.update_PGM_group}\pysiglinewithargsret{\code{diwavars.}\bfcode{update\_PGM\_group}}{\emph{new\_group}}{}
Update the PGM group for this node.

\end{fulllineitems}

\index{update\_audio() (in module diwavars)}

\begin{fulllineitems}
\phantomsection\label{diwavars:diwavars.update_audio}\pysiglinewithargsret{\code{diwavars.}\bfcode{update\_audio}}{\emph{audio}}{}
Docstring here.

\end{fulllineitems}

\index{update\_camera\_vars() (in module diwavars)}

\begin{fulllineitems}
\phantomsection\label{diwavars:diwavars.update_camera_vars}\pysiglinewithargsret{\code{diwavars.}\bfcode{update\_camera\_vars}}{\emph{url}, \emph{user}, \emph{passwd}}{}
Docstring here.

\end{fulllineitems}

\index{update\_database\_vars() (in module diwavars)}

\begin{fulllineitems}
\phantomsection\label{diwavars:diwavars.update_database_vars}\pysiglinewithargsret{\code{diwavars.}\bfcode{update\_database\_vars}}{\emph{address=None}, \emph{name=None}, \emph{type\_=None}, \emph{user=None}, \emph{password=None}}{}
Docstring here.

\end{fulllineitems}

\index{update\_keys() (in module diwavars)}

\begin{fulllineitems}
\phantomsection\label{diwavars:diwavars.update_keys}\pysiglinewithargsret{\code{diwavars.}\bfcode{update\_keys}}{\emph{modifier=91}, \emph{key=27}}{}
Update the key combination to stop remote controlling.
\begin{quote}\begin{description}
\item[{Parameters}] \leavevmode\begin{itemize}
\item {} 
\textbf{modifier} (\emph{Integer}) -- The key to hold.

\item {} 
\textbf{key} (\emph{Integer}) -- The key to press while holding modifier key.

\end{itemize}

\end{description}\end{quote}

\end{fulllineitems}

\index{update\_padfile() (in module diwavars)}

\begin{fulllineitems}
\phantomsection\label{diwavars:diwavars.update_padfile}\pysiglinewithargsret{\code{diwavars.}\bfcode{update\_padfile}}{\emph{padurl}}{}
Set the padfile address.

\end{fulllineitems}

\index{update\_responsive() (in module diwavars)}

\begin{fulllineitems}
\phantomsection\label{diwavars:diwavars.update_responsive}\pysiglinewithargsret{\code{diwavars.}\bfcode{update\_responsive}}{\emph{resp}}{}
Docstring here.

\end{fulllineitems}

\index{update\_storage() (in module diwavars)}

\begin{fulllineitems}
\phantomsection\label{diwavars:diwavars.update_storage}\pysiglinewithargsret{\code{diwavars.}\bfcode{update\_storage}}{\emph{storage}}{}
Update the address of storage.
\begin{quote}\begin{description}
\item[{Parameters}] \leavevmode
\textbf{storage} (\emph{String}) -- The new address of storage.

\end{description}\end{quote}

\end{fulllineitems}

\index{update\_windows\_version() (in module diwavars)}

\begin{fulllineitems}
\phantomsection\label{diwavars:diwavars.update_windows_version}\pysiglinewithargsret{\code{diwavars.}\bfcode{update\_windows\_version}}{}{}~\begin{description}
\item[{Updates the current version information to variables:}] \leavevmode\begin{itemize}
\item {} 
WINDOWS\_MAJOR

\item {} 
WINDOWS\_MINOR

\end{itemize}

\end{description}

\end{fulllineitems}



\section{Filesystem module}
\label{filesystem:module-filesystem}\label{filesystem::doc}\label{filesystem:filesystem-module}\index{filesystem (module)}
Created on 17.5.2013
\begin{quote}\begin{description}
\item[{author}] \leavevmode
neriksso

\end{description}\end{quote}
\index{copy\_file\_to\_project() (in module filesystem)}

\begin{fulllineitems}
\phantomsection\label{filesystem:filesystem.copy_file_to_project}\pysiglinewithargsret{\code{filesystem.}\bfcode{copy\_file\_to\_project}}{\emph{filepath}, \emph{project\_id}}{}
Copy file to project dir and return new filepath in project directory.
\begin{quote}\begin{description}
\item[{Parameters}] \leavevmode\begin{itemize}
\item {} 
\textbf{filepath} (\emph{String}) -- The file path.

\item {} 
\textbf{project\_id} (\emph{Integer}) -- Project id from database.

\end{itemize}

\item[{Returns}] \leavevmode
The path for this file in project directory or empty string.

\item[{Return type}] \leavevmode
String

\end{description}\end{quote}

\end{fulllineitems}

\index{copy\_to\_temporary\_directory() (in module filesystem)}

\begin{fulllineitems}
\phantomsection\label{filesystem:filesystem.copy_to_temporary_directory}\pysiglinewithargsret{\code{filesystem.}\bfcode{copy\_to\_temporary\_directory}}{\emph{filepath}}{}
Copy a file to temporary folder.
\begin{quote}\begin{description}
\item[{Parameters}] \leavevmode
\textbf{filepath} (\emph{String}) -- The file path.

\end{description}\end{quote}

\end{fulllineitems}

\index{create\_project\_directory() (in module filesystem)}

\begin{fulllineitems}
\phantomsection\label{filesystem:filesystem.create_project_directory}\pysiglinewithargsret{\code{filesystem.}\bfcode{create\_project\_directory}}{\emph{dir\_name}}{}
Creates a project directory, if one does not exist in the file system
\begin{quote}\begin{description}
\item[{Parameters}] \leavevmode
\textbf{dir\_name} (\emph{String}) -- Name of the directory

\end{description}\end{quote}

\end{fulllineitems}

\index{delete\_directory() (in module filesystem)}

\begin{fulllineitems}
\phantomsection\label{filesystem:filesystem.delete_directory}\pysiglinewithargsret{\code{filesystem.}\bfcode{delete\_directory}}{\emph{path}}{}
Deletes a directory.
\begin{quote}\begin{description}
\item[{Returns}] \leavevmode
Weather the function was successful or not.

\item[{Return type}] \leavevmode
String

\end{description}\end{quote}

\end{fulllineitems}

\index{file\_to\_base64() (in module filesystem)}

\begin{fulllineitems}
\phantomsection\label{filesystem:filesystem.file_to_base64}\pysiglinewithargsret{\code{filesystem.}\bfcode{file\_to\_base64}}{\emph{filepath}}{}
Transform a file to a binary object.
\begin{quote}\begin{description}
\item[{Parameters}] \leavevmode
\textbf{filepath} (\emph{String}) -- The file path.

\end{description}\end{quote}

\end{fulllineitems}

\index{get\_file\_extension() (in module filesystem)}

\begin{fulllineitems}
\phantomsection\label{filesystem:filesystem.get_file_extension}\pysiglinewithargsret{\code{filesystem.}\bfcode{get\_file\_extension}}{\emph{path}}{}
Returns the file extension of a file
\begin{quote}\begin{description}
\item[{Parameters}] \leavevmode
\textbf{path} (\emph{String}) -- The file path.

\item[{Return type}] \leavevmode
String

\end{description}\end{quote}

\end{fulllineitems}

\index{get\_node\_image() (in module filesystem)}

\begin{fulllineitems}
\phantomsection\label{filesystem:filesystem.get_node_image}\pysiglinewithargsret{\code{filesystem.}\bfcode{get\_node\_image}}{\emph{node}}{}
Searches for a node's image in STORAGE.
\begin{quote}\begin{description}
\item[{Parameters}] \leavevmode
\textbf{node} (\emph{Integer}) -- The node id.

\end{description}\end{quote}

\end{fulllineitems}

\index{is\_subtree() (in module filesystem)}

\begin{fulllineitems}
\phantomsection\label{filesystem:filesystem.is_subtree}\pysiglinewithargsret{\code{filesystem.}\bfcode{is\_subtree}}{\emph{filename}, \emph{parent}, \emph{case\_sensitive=True}}{}
Determines, if filename is inside the parent folder.
\begin{quote}\begin{description}
\item[{Parameters}] \leavevmode\begin{itemize}
\item {} 
\textbf{filename} (\emph{String}) -- The file path.

\item {} 
\textbf{parent} (\emph{String}) -- The parent file path.

\end{itemize}

\end{description}\end{quote}

\end{fulllineitems}

\index{open\_file() (in module filesystem)}

\begin{fulllineitems}
\phantomsection\label{filesystem:filesystem.open_file}\pysiglinewithargsret{\code{filesystem.}\bfcode{open\_file}}{\emph{filepath}}{}
Opens a file path.
\begin{quote}\begin{description}
\item[{Parameters}] \leavevmode
\textbf{filepath} (\emph{String}) -- The file path.

\end{description}\end{quote}

\end{fulllineitems}

\index{save\_screen() (in module filesystem)}

\begin{fulllineitems}
\phantomsection\label{filesystem:filesystem.save_screen}\pysiglinewithargsret{\code{filesystem.}\bfcode{save\_screen}}{\emph{filepath}}{}
Saves the background image of the desktop.
\begin{quote}\begin{description}
\item[{Parameters}] \leavevmode
\textbf{filepath} (\emph{String}) -- The filepath for the saved image.

\end{description}\end{quote}

\end{fulllineitems}

\index{screen\_capture() (in module filesystem)}

\begin{fulllineitems}
\phantomsection\label{filesystem:filesystem.screen_capture}\pysiglinewithargsret{\code{filesystem.}\bfcode{screen\_capture}}{\emph{path}, \emph{node\_id}}{}
Take a screenshot and store it in project folder.
\begin{quote}\begin{description}
\item[{Parameters}] \leavevmode\begin{itemize}
\item {} 
\textbf{path} (\emph{String}) -- Path to the project folder.

\item {} 
\textbf{node\_id} (\emph{Integer}) -- NodeID

\end{itemize}

\end{description}\end{quote}

\end{fulllineitems}

\index{search\_file() (in module filesystem)}

\begin{fulllineitems}
\phantomsection\label{filesystem:filesystem.search_file}\pysiglinewithargsret{\code{filesystem.}\bfcode{search\_file}}{\emph{filename}, \emph{search\_path}, \emph{case\_sensitive=True}}{}
Search file in a given path.
\begin{quote}\begin{description}
\item[{Parameters}] \leavevmode\begin{itemize}
\item {} 
\textbf{filename} (\emph{String}) -- The file name.

\item {} 
\textbf{search\_path} (\emph{String}) -- The search path.

\end{itemize}

\item[{Returns}] \leavevmode
The path to the file.

\item[{Return type}] \leavevmode
String

\end{description}\end{quote}

\end{fulllineitems}

\index{set\_logger\_level() (in module filesystem)}

\begin{fulllineitems}
\phantomsection\label{filesystem:filesystem.set_logger_level}\pysiglinewithargsret{\code{filesystem.}\bfcode{set\_logger\_level}}{\emph{level}}{}
Sets the logger level for filesystem logger.
\begin{quote}\begin{description}
\item[{Parameters}] \leavevmode
\textbf{level} (\emph{Integer}) -- Level of logging.

\end{description}\end{quote}

\end{fulllineitems}

\index{snapshot() (in module filesystem)}

\begin{fulllineitems}
\phantomsection\label{filesystem:filesystem.snapshot}\pysiglinewithargsret{\code{filesystem.}\bfcode{snapshot}}{\emph{path}}{}
Start the worker thread for snapshot.
\begin{quote}\begin{description}
\item[{Parameters}] \leavevmode
\textbf{path} (\emph{String}) -- File path where to store the snapshot.

\end{description}\end{quote}

\end{fulllineitems}

\index{snapshot\_procedure() (in module filesystem)}

\begin{fulllineitems}
\phantomsection\label{filesystem:filesystem.snapshot_procedure}\pysiglinewithargsret{\code{filesystem.}\bfcode{snapshot\_procedure}}{\emph{path}}{}
Worker for storing the snapshot.

\begin{notice}{warning}{Warning:}
This object has a timeout of 1 minute. So consider terminating
the thread on shutdown if it's hanging.
\end{notice}
\begin{quote}\begin{description}
\item[{Parameters}] \leavevmode
\textbf{path} (\emph{String}) -- File path where to store the snapshot.

\end{description}\end{quote}

\end{fulllineitems}

\index{test\_storage\_connection() (in module filesystem)}

\begin{fulllineitems}
\phantomsection\label{filesystem:filesystem.test_storage_connection}\pysiglinewithargsret{\code{filesystem.}\bfcode{test\_storage\_connection}}{}{}
Try to access \textbackslash{}StorageProjects
\begin{quote}\begin{description}
\item[{Returns}] \leavevmode
Does the path exist.

\item[{Return type}] \leavevmode
Boolean

\end{description}\end{quote}

\end{fulllineitems}



\section{Graphical Design module}
\label{graphicaldesign:graphical-design-module}\label{graphicaldesign::doc}\phantomsection\label{graphicaldesign:module-graphicaldesign}\index{graphicaldesign (module)}
Created on 6.6.2013
\begin{quote}\begin{description}
\item[{author}] \leavevmode
neriksso

\item[{synopsis}] \leavevmode
This file represents graphical designs of some GUI elements in DiWaCS.

\end{description}\end{quote}
\index{BlackOverlay (class in graphicaldesign)}

\begin{fulllineitems}
\phantomsection\label{graphicaldesign:graphicaldesign.BlackOverlay}\pysiglinewithargsret{\strong{class }\code{graphicaldesign.}\bfcode{BlackOverlay}}{\emph{pos}, \emph{size}, \emph{parent}, \emph{text}}{}
Represents all black frame without a mouse.
\index{OnFocusLost() (graphicaldesign.BlackOverlay method)}

\begin{fulllineitems}
\phantomsection\label{graphicaldesign:graphicaldesign.BlackOverlay.OnFocusLost}\pysiglinewithargsret{\bfcode{OnFocusLost}}{\emph{evt}}{}
Event handler for focus losing of the window.
\begin{quote}\begin{description}
\item[{Parameters}] \leavevmode
\textbf{evt} (\code{wx.Event}) -- The focus lost event.

\end{description}\end{quote}

\end{fulllineitems}


\end{fulllineitems}

\index{DropTarget (class in graphicaldesign)}

\begin{fulllineitems}
\phantomsection\label{graphicaldesign:graphicaldesign.DropTarget}\pysiglinewithargsret{\strong{class }\code{graphicaldesign.}\bfcode{DropTarget}}{\emph{window}, \emph{parent}, \emph{i}}{}
Implements drop target functionality to receive files, bitmaps and text.
\index{OnData() (graphicaldesign.DropTarget method)}

\begin{fulllineitems}
\phantomsection\label{graphicaldesign:graphicaldesign.DropTarget.OnData}\pysiglinewithargsret{\bfcode{OnData}}{\emph{x}, \emph{y}, \emph{d}}{}
Handles drag/dropping files/text or a bitmap.
\begin{quote}\begin{description}
\item[{Parameters}] \leavevmode\begin{itemize}
\item {} 
\textbf{x} (\emph{Integer}) -- The x coordinate of the drop-location.

\item {} 
\textbf{y} (\emph{Integer}) -- The y coordinate of the drop-location.

\item {} 
\textbf{d} -- The data of drop.

\end{itemize}

\end{description}\end{quote}

\end{fulllineitems}


\end{fulllineitems}

\index{EventListTemplate (class in graphicaldesign)}

\begin{fulllineitems}
\phantomsection\label{graphicaldesign:graphicaldesign.EventListTemplate}\pysiglinewithargsret{\strong{class }\code{graphicaldesign.}\bfcode{EventListTemplate}}{\emph{parent}, \emph{*args}, \emph{**kwargs}}{}
Represents an event list menu.
\index{GetProgramIcon() (graphicaldesign.EventListTemplate method)}

\begin{fulllineitems}
\phantomsection\label{graphicaldesign:graphicaldesign.EventListTemplate.GetProgramIcon}\pysiglinewithargsret{\bfcode{GetProgramIcon}}{\emph{icon}}{}
Fetches gui icons.
\begin{quote}\begin{description}
\item[{Parameters}] \leavevmode
\textbf{icon} (\emph{String}) -- The icon file name.

\item[{Return type}] \leavevmode
\code{wx.Image}

\end{description}\end{quote}

\end{fulllineitems}


\end{fulllineitems}

\index{GUItemplate (class in graphicaldesign)}

\begin{fulllineitems}
\phantomsection\label{graphicaldesign:graphicaldesign.GUItemplate}\pysiglinewithargsret{\strong{class }\code{graphicaldesign.}\bfcode{GUItemplate}}{\emph{*args}, \emph{**kwargs}}{}
Represents the main GUI window graphical template.
\begin{quote}\begin{description}
\item[{Parameters}] \leavevmode\begin{itemize}
\item {} 
\textbf{parent} (\code{wx.Window}) -- Parent frame.

\item {} 
\textbf{id} (\emph{Integer}) -- ID of the new Frame.

\item {} 
\textbf{title} (\emph{String}) -- Title for the frame, default = EmptyString.

\item {} 
\textbf{pos} (\emph{wx.Point}) -- Position of the new frame.

\item {} 
\textbf{size} (\emph{wx.Size}) -- Size of the new frame.

\item {} 
\textbf{style} (\emph{long}) -- Style flags for the new frame.

\item {} 
\textbf{name} (\emph{String}) -- Name of  the new frame.

\end{itemize}

\end{description}\end{quote}
\index{AlignCenterTop() (graphicaldesign.GUItemplate method)}

\begin{fulllineitems}
\phantomsection\label{graphicaldesign:graphicaldesign.GUItemplate.AlignCenterTop}\pysiglinewithargsret{\bfcode{AlignCenterTop}}{}{}
Aligns frame to Horizontal center and vertical top.

\end{fulllineitems}

\index{ClearStatusText() (graphicaldesign.GUItemplate method)}

\begin{fulllineitems}
\phantomsection\label{graphicaldesign:graphicaldesign.GUItemplate.ClearStatusText}\pysiglinewithargsret{\bfcode{ClearStatusText}}{}{}
Sets the status text to EmptyScreen string.

\end{fulllineitems}

\index{ConnectionErrorHandler() (graphicaldesign.GUItemplate method)}

\begin{fulllineitems}
\phantomsection\label{graphicaldesign:graphicaldesign.GUItemplate.ConnectionErrorHandler}\pysiglinewithargsret{\bfcode{ConnectionErrorHandler}}{\emph{error}}{}
Show connection error handler dialog.

\end{fulllineitems}

\index{GetProgramIcon() (graphicaldesign.GUItemplate method)}

\begin{fulllineitems}
\phantomsection\label{graphicaldesign:graphicaldesign.GUItemplate.GetProgramIcon}\pysiglinewithargsret{\bfcode{GetProgramIcon}}{\emph{icon}}{}
Fetches a GUI icon.
\begin{quote}\begin{description}
\item[{Parameters}] \leavevmode
\textbf{icon} (\emph{String}) -- The icon file name.

\item[{Return type}] \leavevmode
\code{wx.Image}

\end{description}\end{quote}

\end{fulllineitems}

\index{HideScreens() (graphicaldesign.GUItemplate method)}

\begin{fulllineitems}
\phantomsection\label{graphicaldesign:graphicaldesign.GUItemplate.HideScreens}\pysiglinewithargsret{\bfcode{HideScreens}}{}{}
Hides all screens.

\end{fulllineitems}

\index{InitScreens() (graphicaldesign.GUItemplate method)}

\begin{fulllineitems}
\phantomsection\label{graphicaldesign:graphicaldesign.GUItemplate.InitScreens}\pysiglinewithargsret{\bfcode{InitScreens}}{}{}
Inits Screens.

\end{fulllineitems}

\index{InitUI() (graphicaldesign.GUItemplate method)}

\begin{fulllineitems}
\phantomsection\label{graphicaldesign:graphicaldesign.GUItemplate.InitUI}\pysiglinewithargsret{\bfcode{InitUI}}{\emph{node\_id}}{}
UI initializing.
\begin{quote}\begin{description}
\item[{Parameters}] \leavevmode
\textbf{node\_id} (\emph{Integer}) -- The id of current swnp node (self).

\end{description}\end{quote}

\end{fulllineitems}

\index{OnExit() (graphicaldesign.GUItemplate method)}

\begin{fulllineitems}
\phantomsection\label{graphicaldesign:graphicaldesign.GUItemplate.OnExit}\pysiglinewithargsret{\bfcode{OnExit}}{\emph{event}}{}
Exits program.
\begin{quote}\begin{description}
\item[{Parameters}] \leavevmode
\textbf{event} (\emph{Event}) -- GUI Event

\end{description}\end{quote}

\end{fulllineitems}

\index{SelectNode() (graphicaldesign.GUItemplate method)}

\begin{fulllineitems}
\phantomsection\label{graphicaldesign:graphicaldesign.GUItemplate.SelectNode}\pysiglinewithargsret{\bfcode{SelectNode}}{\emph{evt}}{}
Handles the selection of a node, prototype.
\begin{quote}\begin{description}
\item[{Parameters}] \leavevmode
\textbf{evt} (\emph{Event}) -- GUI Event

\end{description}\end{quote}

\end{fulllineitems}


\end{fulllineitems}

\index{ImageViewer (class in graphicaldesign)}

\begin{fulllineitems}
\phantomsection\label{graphicaldesign:graphicaldesign.ImageViewer}\pysiglinewithargsret{\strong{class }\code{graphicaldesign.}\bfcode{ImageViewer}}{\emph{parent}, \emph{image}, \emph{*args}, \emph{**kwargs}}{}
Used to show an image.

\end{fulllineitems}

\index{MySplashScreen (class in graphicaldesign)}

\begin{fulllineitems}
\phantomsection\label{graphicaldesign:graphicaldesign.MySplashScreen}\pysiglinewithargsret{\strong{class }\code{graphicaldesign.}\bfcode{MySplashScreen}}{\emph{parent=None}}{}
Create a splash screen widget.

\end{fulllineitems}

\index{NodeScreen (class in graphicaldesign)}

\begin{fulllineitems}
\phantomsection\label{graphicaldesign:graphicaldesign.NodeScreen}\pysiglinewithargsret{\strong{class }\code{graphicaldesign.}\bfcode{NodeScreen}}{\emph{node}, \emph{parent}}{}
Represents a bitmap with node id.
\index{EmptyScreen() (graphicaldesign.NodeScreen method)}

\begin{fulllineitems}
\phantomsection\label{graphicaldesign:graphicaldesign.NodeScreen.EmptyScreen}\pysiglinewithargsret{\bfcode{EmptyScreen}}{}{}
Make this screen EmptyScreen.

\end{fulllineitems}

\index{ReloadAs() (graphicaldesign.NodeScreen method)}

\begin{fulllineitems}
\phantomsection\label{graphicaldesign:graphicaldesign.NodeScreen.ReloadAs}\pysiglinewithargsret{\bfcode{ReloadAs}}{\emph{node}}{}
Reload the content of this bitmap.

\end{fulllineitems}


\end{fulllineitems}

\index{SysTray (class in graphicaldesign)}

\begin{fulllineitems}
\phantomsection\label{graphicaldesign:graphicaldesign.SysTray}\pysiglinewithargsret{\strong{class }\code{graphicaldesign.}\bfcode{SysTray}}{\emph{parent}}{}
Taskbar Icon class.
\begin{quote}\begin{description}
\item[{Parameters}] \leavevmode
\textbf{parent} (\code{wx.Frame}) -- Parent frame

\end{description}\end{quote}
\index{CreateMenu() (graphicaldesign.SysTray method)}

\begin{fulllineitems}
\phantomsection\label{graphicaldesign:graphicaldesign.SysTray.CreateMenu}\pysiglinewithargsret{\bfcode{CreateMenu}}{}{}
Create systray menu.

\end{fulllineitems}

\index{ShowMenu() (graphicaldesign.SysTray method)}

\begin{fulllineitems}
\phantomsection\label{graphicaldesign:graphicaldesign.SysTray.ShowMenu}\pysiglinewithargsret{\bfcode{ShowMenu}}{\emph{event}}{}
Show popup menu.
\begin{quote}\begin{description}
\item[{Parameters}] \leavevmode
\textbf{event} (\emph{Event}) -- GUI event.

\end{description}\end{quote}

\end{fulllineitems}

\index{ShowNotification() (graphicaldesign.SysTray method)}

\begin{fulllineitems}
\phantomsection\label{graphicaldesign:graphicaldesign.SysTray.ShowNotification}\pysiglinewithargsret{\bfcode{ShowNotification}}{\emph{title}, \emph{message}}{}
Start a thread to show the notification.
\begin{quote}\begin{description}
\item[{Parameters}] \leavevmode\begin{itemize}
\item {} 
\textbf{title} (\emph{String}) -- Title to diplay in the balloon.

\item {} 
\textbf{message} (\emph{String}) -- Message to display in the balloong (max 255 chars).

\end{itemize}

\end{description}\end{quote}

\end{fulllineitems}


\end{fulllineitems}



\section{Macro module}
\label{macro:macro-module}\label{macro:module-macro}\label{macro::doc}\index{macro (module)}
macro.py defines a few user input functions.
\index{GetKeydown() (in module macro)}

\begin{fulllineitems}
\phantomsection\label{macro:macro.GetKeydown}\pysiglinewithargsret{\code{macro.}\bfcode{GetKeydown}}{\emph{code}}{}
Docstring here.

\end{fulllineitems}

\index{HardwareInput (class in macro)}

\begin{fulllineitems}
\phantomsection\label{macro:macro.HardwareInput}\pysigline{\strong{class }\code{macro.}\bfcode{HardwareInput}}
Docstring here.

\end{fulllineitems}

\index{Input (class in macro)}

\begin{fulllineitems}
\phantomsection\label{macro:macro.Input}\pysigline{\strong{class }\code{macro.}\bfcode{Input}}
Docstring here.

\end{fulllineitems}

\index{Input\_I (class in macro)}

\begin{fulllineitems}
\phantomsection\label{macro:macro.Input_I}\pysigline{\strong{class }\code{macro.}\bfcode{Input\_I}}
Docstring here.

\end{fulllineitems}

\index{KeyBdInput (class in macro)}

\begin{fulllineitems}
\phantomsection\label{macro:macro.KeyBdInput}\pysigline{\strong{class }\code{macro.}\bfcode{KeyBdInput}}
Docstring here.

\end{fulllineitems}

\index{MacroPoint (class in macro)}

\begin{fulllineitems}
\phantomsection\label{macro:macro.MacroPoint}\pysigline{\strong{class }\code{macro.}\bfcode{MacroPoint}}
Stores the x and y components of coordinates.
\begin{quote}\begin{description}
\item[{Attribute x}] \leavevmode
c\_ulong

\item[{Attribute y}] \leavevmode
c\_ulong

\end{description}\end{quote}

\end{fulllineitems}

\index{MouseInput (class in macro)}

\begin{fulllineitems}
\phantomsection\label{macro:macro.MouseInput}\pysigline{\strong{class }\code{macro.}\bfcode{MouseInput}}
Docstring here.

\end{fulllineitems}

\index{click() (in module macro)}

\begin{fulllineitems}
\phantomsection\label{macro:macro.click}\pysiglinewithargsret{\code{macro.}\bfcode{click}}{}{}
Send a mouse click\_type: LeftButton down, LeftButton up.

\end{fulllineitems}

\index{get\_mouse\_position() (in module macro)}

\begin{fulllineitems}
\phantomsection\label{macro:macro.get_mouse_position}\pysiglinewithargsret{\code{macro.}\bfcode{get\_mouse\_position}}{}{}
Return the current position of the mouse.
\begin{quote}\begin{description}
\item[{Returns}] \leavevmode
The position of the mouse.

\item[{Return type}] \leavevmode
{\hyperref[macro:macro.MacroPoint]{\code{MacroPoint}}}

\end{description}\end{quote}

\end{fulllineitems}

\index{get\_sendkeys() (in module macro)}

\begin{fulllineitems}
\phantomsection\label{macro:macro.get_sendkeys}\pysiglinewithargsret{\code{macro.}\bfcode{get\_sendkeys}}{\emph{code}}{}
Returns a character for a key code.
\begin{quote}\begin{description}
\item[{Parameters}] \leavevmode
\textbf{code} (\emph{Integer}) -- The character code.

\end{description}\end{quote}

\end{fulllineitems}

\index{hold() (in module macro)}

\begin{fulllineitems}
\phantomsection\label{macro:macro.hold}\pysiglinewithargsret{\code{macro.}\bfcode{hold}}{}{}
Send a mouse hold: LeftButton down.

\end{fulllineitems}

\index{key\_press() (in module macro)}

\begin{fulllineitems}
\phantomsection\label{macro:macro.key_press}\pysiglinewithargsret{\code{macro.}\bfcode{key\_press}}{\emph{event}, \emph{kcode}}{}
Used to send a single virtual keycode to the system.
\begin{quote}\begin{description}
\item[{Parameters}] \leavevmode\begin{itemize}
\item {} 
\textbf{event} (\code{wx.Event}) -- Captured key event.

\item {} 
\textbf{kcode} (\emph{Integer}) -- Keycode.

\end{itemize}

\end{description}\end{quote}

\end{fulllineitems}

\index{middle\_click() (in module macro)}

\begin{fulllineitems}
\phantomsection\label{macro:macro.middle_click}\pysiglinewithargsret{\code{macro.}\bfcode{middle\_click}}{}{}
Send a mouse middle click\_type: MiddleButton down, MiddleButton up.

\end{fulllineitems}

\index{middle\_hold() (in module macro)}

\begin{fulllineitems}
\phantomsection\label{macro:macro.middle_hold}\pysiglinewithargsret{\code{macro.}\bfcode{middle\_hold}}{}{}
Send a mouse middle click\_type: MiddleButton down.

\end{fulllineitems}

\index{middle\_release() (in module macro)}

\begin{fulllineitems}
\phantomsection\label{macro:macro.middle_release}\pysiglinewithargsret{\code{macro.}\bfcode{middle\_release}}{}{}
Send a mouse middle click\_type: MiddleButton up.

\end{fulllineitems}

\index{move() (in module macro)}

\begin{fulllineitems}
\phantomsection\label{macro:macro.move}\pysiglinewithargsret{\code{macro.}\bfcode{move}}{\emph{pos\_x}, \emph{pos\_y}}{}
move the cursor for pos\_x amount in horizontal direction and pos\_y amount
in vertical direction.
\begin{quote}\begin{description}
\item[{Parameters}] \leavevmode\begin{itemize}
\item {} 
\textbf{pos\_x} (\emph{Integer}) -- Amount to move in horizontal direction.

\item {} 
\textbf{pos\_y} (\emph{Integer}) -- Amount to move in vertical direction.

\end{itemize}

\end{description}\end{quote}

\end{fulllineitems}

\index{move\_to() (in module macro)}

\begin{fulllineitems}
\phantomsection\label{macro:macro.move_to}\pysiglinewithargsret{\code{macro.}\bfcode{move\_to}}{\emph{pos\_x}, \emph{pos\_y}}{}
move the mouse cursor to point (pos\_x, pos\_y) on screen.
\begin{quote}\begin{description}
\item[{Parameters}] \leavevmode\begin{itemize}
\item {} 
\textbf{pos\_x} (\emph{Integer}) -- X coordinate of the desired position.

\item {} 
\textbf{pos\_y} (\emph{Integer}) -- Y coordinate of the desired position.

\end{itemize}

\end{description}\end{quote}

\end{fulllineitems}

\index{release() (in module macro)}

\begin{fulllineitems}
\phantomsection\label{macro:macro.release}\pysiglinewithargsret{\code{macro.}\bfcode{release}}{}{}
Send a mouse release\_type: LeftButton up.

\end{fulllineitems}

\index{release\_all\_keys() (in module macro)}

\begin{fulllineitems}
\phantomsection\label{macro:macro.release_all_keys}\pysiglinewithargsret{\code{macro.}\bfcode{release\_all\_keys}}{}{}
Reset every keycode state to UP state.

\end{fulllineitems}

\index{right\_click() (in module macro)}

\begin{fulllineitems}
\phantomsection\label{macro:macro.right_click}\pysiglinewithargsret{\code{macro.}\bfcode{right\_click}}{}{}
Send a mouse right click\_type: RightButton down, RightButton up.

\end{fulllineitems}

\index{right\_hold() (in module macro)}

\begin{fulllineitems}
\phantomsection\label{macro:macro.right_hold}\pysiglinewithargsret{\code{macro.}\bfcode{right\_hold}}{}{}
Send a mouse right hold: RightButton down.

\end{fulllineitems}

\index{right\_release() (in module macro)}

\begin{fulllineitems}
\phantomsection\label{macro:macro.right_release}\pysiglinewithargsret{\code{macro.}\bfcode{right\_release}}{}{}
Send a mouse right release\_type: RightButton up.

\end{fulllineitems}

\index{send\_input() (in module macro)}

\begin{fulllineitems}
\phantomsection\label{macro:macro.send_input}\pysiglinewithargsret{\code{macro.}\bfcode{send\_input}}{\emph{intype}, \emph{data}, \emph{flags}, \emph{scan=0}, \emph{mouse\_data=0}}{}
send\_input sends virtual user input.
\begin{quote}\begin{description}
\item[{Parameters}] \leavevmode\begin{itemize}
\item {} 
\textbf{intype} (\emph{String}) -- Input type, either `mouse\_input' for mouse input or `key\_input' for
keyboard input.

\item {} 
\textbf{data} (\emph{Integer or (Integer, Integer)}) -- Input data, keycode to input or a tuple of (x, y) for mouse.

\item {} 
\textbf{flags} (\emph{Integer}) -- Input flags, used to separate keyup and keydown events.

\item {} 
\textbf{scan} (\emph{Integer}) -- Input scancode. More info in: \href{http://en.wikipedia.org/wiki/Scancode}{http://en.wikipedia.org/wiki/Scancode}

\item {} 
\textbf{mouse\_data} (\emph{Integer}) -- Represents additional information about mouse events for example wheel
amount.

\end{itemize}

\end{description}\end{quote}

\end{fulllineitems}

\index{slide() (in module macro)}

\begin{fulllineitems}
\phantomsection\label{macro:macro.slide}\pysiglinewithargsret{\code{macro.}\bfcode{slide}}{\emph{difference\_x}, \emph{difference\_y}}{}
slide the mouse for difference\_x amount in horizontal direction and
difference\_y amount in vertical direction.
\begin{quote}\begin{description}
\item[{Parameters}] \leavevmode\begin{itemize}
\item {} 
\textbf{difference\_x} (\emph{Integer}) -- The amount to slide in horizontal direction.

\item {} 
\textbf{difference\_y} (\emph{Integer}) -- The amount to slide in vertical direction.

\end{itemize}

\end{description}\end{quote}

\end{fulllineitems}

\index{slide\_to() (in module macro)}

\begin{fulllineitems}
\phantomsection\label{macro:macro.slide_to}\pysiglinewithargsret{\code{macro.}\bfcode{slide\_to}}{\emph{target\_x}, \emph{target\_y}, \emph{speed='normal'}}{}
Slides the mouse to point (target\_x, target\_y)
\begin{quote}\begin{description}
\item[{Parameters}] \leavevmode\begin{itemize}
\item {} 
\textbf{target\_x} (\emph{Integer}) -- The target X coordinate.

\item {} 
\textbf{target\_y} (\emph{Integer}) -- The target Y coordinate.

\item {} 
\textbf{speed} (\emph{String}) -- The speed of motion `slow', `normal' or `fast'.

\end{itemize}

\end{description}\end{quote}

\end{fulllineitems}



\section{Models module}
\label{models:module-models}\label{models::doc}\label{models:models-module}\index{models (module)}
Created on 23.5.2012
\begin{quote}\begin{description}
\item[{author}] \leavevmode
neriksso

\item[{warning}] \leavevmode
Requires \code{sqlalchemy} and \code{pywin32}

\item[{synopsis}] \leavevmode
Used to represent the different database structures on DiWa.

\end{description}\end{quote}
\index{Action (class in models)}

\begin{fulllineitems}
\phantomsection\label{models:models.Action}\pysiglinewithargsret{\strong{class }\code{models.}\bfcode{Action}}{\emph{name}}{}
A class representation of a action. A file action uses this to describe
the action.
\begin{description}
\item[{Field:}] \leavevmode\begin{itemize}
\item {} 
\code{id}        (\code{sqlalchemy.schema.Column(sqlalchemy.types.INTEGER)})        - ID of the action, used as primary key in database table.

\item {} 
\code{name}        (\code{sqlalchemy.schema.Column(sqlalchemy.types.String)})        - Name of the action (Max 50 characters).

\end{itemize}

\end{description}
\begin{quote}\begin{description}
\item[{Parameters}] \leavevmode
\textbf{name} (\code{String}) -- Name of the action.

\end{description}\end{quote}

\end{fulllineitems}

\index{Activity (class in models)}

\begin{fulllineitems}
\phantomsection\label{models:models.Activity}\pysiglinewithargsret{\strong{class }\code{models.}\bfcode{Activity}}{\emph{project}, \emph{session=None}}{}
A class representation of an activity.
\begin{description}
\item[{Fields:}] \leavevmode\begin{itemize}
\item {} 
\code{id}        (\code{sqlalchemy.schema.Column(sqlalchemy.types.INTEGER)})        - ID of activity, used as primary key in database table.

\item {} 
\code{session\_id}        (\code{sqlalchemy.schema.Column(sqlalchemy.types.INTEGER)})        - ID of the session activity belongs to.

\item {} 
\code{session} (\code{sqlalchemy.orm.relationship})        - Session relationship.

\item {} 
\code{project\_id}        (\code{sqlalchemy.schema.Column(sqlalchemy.types.INTEGER)})        - ID of the project activity belongs to.

\item {} 
\code{project} (\code{sqlalchemy.orm.relationship})        - Project relationship.

\item {} 
\code{active}        (\code{sqlalchemy.schema.Column(sqlalchemy.types.BOOLEAN)})        - Boolean flag indicating that the project is active.

\end{itemize}

\end{description}
\begin{quote}\begin{description}
\item[{Parameters}] \leavevmode\begin{itemize}
\item {} 
\textbf{project} ({\hyperref[models:models.Project]{\code{models.Project}}}) -- Project activity belongs to.

\item {} 
\textbf{session} ({\hyperref[models:models.Session]{\code{models.Session}}}) -- Optional session activity belongs to.

\end{itemize}

\end{description}\end{quote}

\end{fulllineitems}

\index{Company (class in models)}

\begin{fulllineitems}
\phantomsection\label{models:models.Company}\pysiglinewithargsret{\strong{class }\code{models.}\bfcode{Company}}{\emph{name}}{}
A class representation of a company.
\begin{description}
\item[{Fields:}] \leavevmode\begin{itemize}
\item {} 
\code{id}        (\code{sqlalchemy.schema.Column(sqlalchemy.types.INTEGER)})        - ID of the company, used as primary key in database table.

\item {} 
\code{name}        (\code{sqlalchemy.schema.Column(sqlalchemy.types.String)})        - Name of the company (Max 50 characters).

\end{itemize}

\end{description}
\begin{quote}\begin{description}
\item[{Parameters}] \leavevmode
\textbf{name} (\code{String}) -- The name of the company.

\end{description}\end{quote}

\end{fulllineitems}

\index{Computer (class in models)}

\begin{fulllineitems}
\phantomsection\label{models:models.Computer}\pysiglinewithargsret{\strong{class }\code{models.}\bfcode{Computer}}{\emph{**kwargs}}{}
A class representation of a computer.
\begin{description}
\item[{Fields:}] \leavevmode\begin{itemize}
\item {} 
\code{id}        (\code{sqlalchemy.schema.Column(sqlalchemy.types.INTEGER)})        - ID of computer, used as primary key in database table.

\item {} 
\code{name}        (\code{sqlalchemy.schema.Column(sqlalchemy.types.String)})        - Name of the computer.

\item {} 
\code{ip}        (\code{sqlalchemy.schema.Column(sqlalchemy.dialects.INTEGER)})        - Internet Protocol address of the computer (Defined as unsigned).

\item {} 
\code{mac}        (\code{sqlalchemy.schema.Column(sqlalchemy.types.String)}        - Media Access Control address of the computer.

\item {} 
\code{time}        (\code{sqlalchemy.schema.Column(sqlalchemy.types.DATETIME)})        - Time of the last network activity from the computer.

\item {} 
\code{screens}        (\code{sqlalchemy.schema.Column(sqlalchemy.types.SMALLINT)})        - Number of screens on the computer.

\item {} 
\code{responsive}        (\code{sqlalchemy.schema.Column(sqlalchemy.types.SMALLINT)})        - The responsive value of the computer.

\item {} 
\code{user\_id}        (\code{sqlalchemy.schema.Column(sqlalchemy.types.INTEGER)})        - ID of the user currently using the computer.

\item {} 
\code{user} (\code{sqlalchemy.orm.relationship})        - The current user.

\item {} 
\code{wos\_id}        (\code{sqlalchemy.schema.Column(sqlalchemy.types.INTEGER)})        - \textbf{WOS} ID.

\end{itemize}

\end{description}

\end{fulllineitems}

\index{Event (class in models)}

\begin{fulllineitems}
\phantomsection\label{models:models.Event}\pysiglinewithargsret{\strong{class }\code{models.}\bfcode{Event}}{\emph{**kwargs}}{}
A class representation of Event. A simple note with timestamp during a
session.
\begin{description}
\item[{Fields:}] \leavevmode\begin{itemize}
\item {} 
\code{id}        (\code{sqlalchemy.schema.Column(sqlalchemy.types.INTEGER)})        - ID of the event, used as primary key in database table.

\item {} 
\code{title}        (\code{sqlalchemy.schema.Column(sqlalchemy.types.String)})        - Title of the event (Max 40 characters).

\item {} 
\code{desc}        (\code{sqlalchemy.schema.Column(sqlalchemy.types.String)})        - More in-depth description of the event (Max 500 characters).

\item {} 
\code{time}        (\code{sqlalchemy.schema.Column(sqlalchemy.types.DATETIME)})        - Time the event took place.

\item {} 
\code{session\_id}        (\code{sqlalchemy.schema.Column(sqlalchemy.types.INTEGER)})        - ID of the session this event belongs to.

\item {} 
\code{session} (\code{sqlalchemy.orm.relationship})        - Session this event belongs to.

\end{itemize}

\end{description}

\end{fulllineitems}

\index{File (class in models)}

\begin{fulllineitems}
\phantomsection\label{models:models.File}\pysiglinewithargsret{\strong{class }\code{models.}\bfcode{File}}{\emph{**kwargs}}{}
A class representation of a file.
\begin{description}
\item[{Fields:}] \leavevmode\begin{itemize}
\item {} 
\code{id}        (\code{sqlalchemy.schema.Column(sqlalchemy.types.INTEGER)})        - ID of the file, used as primary key in database table.

\item {} 
\code{path}        (\code{sqlalchemy.schema.Column(sqlalchemy.types.String)})        - Path of the file on DiWa (max 255 chars).

\item {} 
\code{project\_id}        (\code{sqlalchemy.schema.Column(sqlalchemy.types.INTEGER)})        - ID of the project this file belongs to.

\item {} 
\code{project} (\code{sqlalchemy.orm.relationship})        - Project this file belongs to.

\end{itemize}

\end{description}

\end{fulllineitems}

\index{FileAction (class in models)}

\begin{fulllineitems}
\phantomsection\label{models:models.FileAction}\pysiglinewithargsret{\strong{class }\code{models.}\bfcode{FileAction}}{\emph{fileobject}, \emph{action}, \emph{session=None}, \emph{computer=None}, \emph{user=None}}{}
A class representation of a file action.
\begin{description}
\item[{Fields:}] \leavevmode\begin{itemize}
\item {} 
\code{id}        (\code{sqlalchemy.schema.Column(sqlalchemy.types.INTEGER)})        - ID of the FileAction, used as primary key in the database table.

\item {} 
\code{file\_id}        (\code{sqlalchemy.schema.Column(sqlaclhemy.types.INTEGER)})        - ID of the file this FileAction affects.

\item {} 
\code{file} (\code{sqlalchemy.orm.relationship)})        - The file this FileAction affects.

\item {} 
\code{action\_id}        (\code{sqlalchemy.schema.Column(sqlalchemy.types.INTEGER)})        - ID of the action affecting the file.

\item {} 
\code{action} (\code{sqlalchemy.orm.relationship)})        - Action affecting the file.

\item {} 
\code{action\_time}        (\code{sqlalchemy.schema.Column(sqlalchemy.types.DATETIME)})        - Time the action took place on.

\item {} 
\code{user\_id}        (\code{sqlalchemy.schema.Column(sqlalchemy.types.INTEGER)})        - ID of the user performing the action.

\item {} 
\code{user} (\code{sqlalchemy.orm.relationship})        - User peforming the action.

\item {} 
\code{computer\_id}        (\code{sqlalchemy.schema.Column(sqlalchemy.types.INTEGER)})        - ID of the computer user performed the action on.

\item {} 
\code{computer} (\code{sqlalchemy.orm.relationship})        - Computer user performed the action on.

\item {} 
\code{session\_id}        (\code{sqlalchemy.schema.Column(sqlalchemy.types.INTEGER)})        - ID of the session user performed the action in.

\item {} 
\code{session} (\code{sqlalchemy.orm.relationship})        - Session user performed the action in.

\end{itemize}

\end{description}
\begin{quote}\begin{description}
\item[{Parameters}] \leavevmode\begin{itemize}
\item {} 
\textbf{fileobject} ({\hyperref[models:models.File]{\code{models.File}}}) -- The file which is subjected to the action.

\item {} 
\textbf{action} ({\hyperref[models:models.Action]{\code{models.Action}}}) -- The action which is applied to the file.

\item {} 
\textbf{session} ({\hyperref[models:models.Session]{\code{models.Session}}}) -- The session in which the FileAction took place on.

\item {} 
\textbf{computer} ({\hyperref[models:models.Computer]{\code{models.Computer}}}) -- The computer from which the user performed the action.

\item {} 
\textbf{user} ({\hyperref[models:models.User]{\code{models.User}}}) -- The user performing the action.

\end{itemize}

\end{description}\end{quote}

\end{fulllineitems}

\index{Project (class in models)}

\begin{fulllineitems}
\phantomsection\label{models:models.Project}\pysiglinewithargsret{\strong{class }\code{models.}\bfcode{Project}}{\emph{name}, \emph{company}, \emph{password}}{}
A class representation of a project.
\begin{description}
\item[{Fields:}] \leavevmode\begin{itemize}
\item {} 
\code{id}        (\code{sqlalchemy.schema.Column(sqlalchemy.types.INTEGER)})        - ID of project, used as primary key in database table.

\item {} 
\code{name}        (\code{sqlalchemy.schema.Column(sqlalchemy.types.String)})        - Name of the project (Max 50 characters).

\item {} 
\code{company\_id}        (\code{sqlalchemy.schema.Column(sqlalchemy.types.INTEGER)})        - ID of the company that owns the project.

\item {} 
\code{company} (\code{sqlalchemy.orm.relationship})        - The company that owns the project.

\item {} 
\code{dir}        (\code{sqlalchemy.schema.Column(sqlalchemy.types.String)})        - Directory path for the project files (Max 255 characters).

\item {} 
\code{password}        (\code{sqlalchemy.schema.Column(sqlalchemy.types.String)})        - Password for the project (Max 40 characters).

\item {} 
\code{members} (\code{sqlalchemy.orm.relationship})        - The users that work on the project.

\end{itemize}

\end{description}
\begin{quote}\begin{description}
\item[{Parameters}] \leavevmode\begin{itemize}
\item {} 
\textbf{name} (\code{String}) -- Name of the project.

\item {} 
\textbf{company} ({\hyperref[models:models.Company]{\code{models.Company}}}) -- The owner of the project.

\end{itemize}

\end{description}\end{quote}

\end{fulllineitems}

\index{Session (class in models)}

\begin{fulllineitems}
\phantomsection\label{models:models.Session}\pysiglinewithargsret{\strong{class }\code{models.}\bfcode{Session}}{\emph{project}}{}
A class representation of a session.
\begin{description}
\item[{Fields:}] \leavevmode\begin{itemize}
\item {} 
\code{id}        (\code{sqlalchemy.schema.Column(sqlalchemy.types.INTEGER)})        - ID of session, used as primary key in database table.

\item {} 
\code{name}        (\code{sqlalchemy.schema.Column(sqlalchemy.types.String)})        - Name of session (Max 50 characters).

\item {} 
\code{project\_id}        (\code{sqlalchemy.schema.Column(sqlalchemy.types.INTEGER)})        - ID of the project the session belongs to.

\item {} 
\code{project} (\code{sqlalchemy.orm.relationship})        - The project the session belongs to.

\item {} 
\code{starttime}        (\code{sqlalchemy.schema.Column(sqlalchemy.types.DATETIME)})        - Time the session began, defaults to \emph{now()}.

\item {} 
\code{endtime}        (\code{sqlalchemy.schema.Column(sqlalchemy.types.DATETIME)})        - The time session ended.

\item {} 
\code{previous\_session\_id}        (\code{sqlalchemy.schema.Column(sqlalchemy.types.INTEGER)})        - ID of the previous session.

\item {} 
\code{previous\_session}        (\code{sqlalchemy.orm.relationship}) - The previous session.

\item {} 
\code{participants} (\code{sqlalchemy.orm.relationship})        - Users that belong to this session.

\item {} 
\code{computers} (\code{sqlalchemy.orm.relationship})        - Computers that belong to this session.

\end{itemize}

\end{description}
\begin{quote}\begin{description}
\item[{Parameters}] \leavevmode
\textbf{project} ({\hyperref[models:models.Project]{\code{models.Project}}}) -- The project for the session.

\end{description}\end{quote}
\index{AddUser() (models.Session method)}

\begin{fulllineitems}
\phantomsection\label{models:models.Session.AddUser}\pysiglinewithargsret{\bfcode{AddUser}}{\emph{user}}{}
Add users to a session.
\begin{quote}\begin{description}
\item[{Parameters}] \leavevmode
\textbf{user} ({\hyperref[models:models.User]{\code{models.User}}}) -- User to be added into the session.

\end{description}\end{quote}

\end{fulllineitems}

\index{FileRoutine() (models.Session method)}

\begin{fulllineitems}
\phantomsection\label{models:models.Session.FileRoutine}\pysiglinewithargsret{\bfcode{FileRoutine}}{}{}
File checking routine for logging.
\begin{quote}\begin{description}
\item[{Throws IOError}] \leavevmode
When log.txt is not available for write access.

\end{description}\end{quote}

\end{fulllineitems}

\index{GetLastChecked() (models.Session method)}

\begin{fulllineitems}
\phantomsection\label{models:models.Session.GetLastChecked}\pysiglinewithargsret{\bfcode{GetLastChecked}}{}{}
Fetch \code{last\_checked} field.
\begin{quote}\begin{description}
\item[{Returns}] \leavevmode
\code{last\_checked} field (None before
\code{models.Session.start()} is called).

\item[{Return type}] \leavevmode
\code{datetime.datetime} or \code{None}

\end{description}\end{quote}

\end{fulllineitems}

\index{Start() (models.Session method)}

\begin{fulllineitems}
\phantomsection\label{models:models.Session.Start}\pysiglinewithargsret{\bfcode{Start}}{}{}
Start a session.
Set the \code{last\_checked} field to current DateTime.

\end{fulllineitems}


\end{fulllineitems}

\index{User (class in models)}

\begin{fulllineitems}
\phantomsection\label{models:models.User}\pysiglinewithargsret{\strong{class }\code{models.}\bfcode{User}}{\emph{name}, \emph{company}}{}
A class representation of a user.
\begin{description}
\item[{Fields:}] \leavevmode\begin{itemize}
\item {} 
\code{id}        (\code{sqlalchemy.schema.Column(sqlalchemy.types.INTEGER)})        - ID of the user, used as primary key in database table.

\item {} 
\code{name}        (\code{sqlalchemy.schema.Column(sqlalchemy.types.String)})        - Name of the user (Max 50 characters).

\item {} 
\code{email}        (\code{sqlalchemy.schema.Column(sqlalchemy.types.String)})        - Email address of the user (Max 100 characters).

\item {} 
\code{title}        (\code{sqlalchemy.schema.Column(sqlalchemy.types.String)})        - Title of the user in the company (Max 50 characters).

\item {} 
\code{department}        (\code{sqlalchemy.schema.Column(sqlalchemy.types.String)})        - Department of the user in the company (Max 100 characters).

\item {} 
\code{company\_id}        (\code{sqlalchemy.schema.Column(sqlalchemy.types.INTEGER)})        - Company id of the employing company.

\item {} 
\code{company} (\code{sqlalchemy.orm.relationship})        - Company relationship.

\end{itemize}

\end{description}
\begin{quote}\begin{description}
\item[{Parameters}] \leavevmode\begin{itemize}
\item {} 
\textbf{name} (\code{String}) -- Name of the user.

\item {} 
\textbf{company} ({\hyperref[models:models.Company]{\code{models.Company}}}) -- The employer.

\end{itemize}

\end{description}\end{quote}

\end{fulllineitems}



\section{Setup module}
\label{setup:setup-module}\label{setup::doc}
Created on 8.5.2012
\begin{quote}\begin{description}
\item[{author}] \leavevmode
nick26

\item[{synopsis}] \leavevmode
This file is used to compile a DiWaCS.exe file out of the python project
using py2exe and setuptools packages available at:
pypi.python.org/pypi/setuptools

\end{description}\end{quote}


\section{Models state}
\label{state:module-state}\label{state::doc}\label{state:models-state}\index{state (module)}
Created on 4.7.2013

@author: neriksso
\index{State (class in state)}

\begin{fulllineitems}
\phantomsection\label{state:state.State}\pysiglinewithargsret{\strong{class }\code{state.}\bfcode{State}}{\emph{parent}}{}
classdocs
\index{end\_current\_project() (state.State method)}

\begin{fulllineitems}
\phantomsection\label{state:state.State.end_current_project}\pysiglinewithargsret{\bfcode{end\_current\_project}}{}{}
End the current project.

\end{fulllineitems}

\index{end\_current\_session() (state.State method)}

\begin{fulllineitems}
\phantomsection\label{state:state.State.end_current_session}\pysiglinewithargsret{\bfcode{end\_current\_session}}{}{}
End the current session.

\end{fulllineitems}

\index{get\_random\_responsive() (state.State method)}

\begin{fulllineitems}
\phantomsection\label{state:state.State.get_random_responsive}\pysiglinewithargsret{\bfcode{get\_random\_responsive}}{}{}
Get a random node amongst all the responsive nodes.

\end{fulllineitems}

\index{handle\_file\_send() (state.State method)}

\begin{fulllineitems}
\phantomsection\label{state:state.State.handle_file_send}\pysiglinewithargsret{\bfcode{handle\_file\_send}}{\emph{filenames}, \emph{progressdialog=None}}{}
Sends a file link to another node.

First parses all the files and folder structure, then confirms weather
the users wishes to add the items to project before beginning the copy
routine.

The copy routine first creates all the needed subfolders and then sums
up all the file sizes to be copied. Then it will update the dialog
in the beginning/end of every file transaction and whenever there's
been more than 1 second from the last update dialog update. Assuming
the progressdialog parameter has been given.
\begin{description}
\item[{The progress dialog, if supplied, is updated as follows:}] \leavevmode\begin{itemize}
\item {} 
If there's less than DEF\_FILES (\textbf{40}) files the dialog             will not be shown or updated.

\item {} 
If the data size sum is less than DEF\_SIZE (\textbf{2 MB}) the             dialog will not be shown or updated.

\item {} 
Title will contain the total percentage of data transfer.

\item {} 
Message will contain the percentage of current file transfer.

\item {} 
Progress bar is set to percent {[}0, 100{]} of the total data             transfer.

\end{itemize}

\end{description}
\begin{quote}\begin{description}
\item[{Parameters}] \leavevmode\begin{itemize}
\item {} 
\textbf{filenames} (\emph{List of String}) -- All the files/folders to be copied.

\item {} 
\textbf{progressdialog} (\code{wx.ProgressDialog}) -- The progress dialog to update (optional).

\end{itemize}

\end{description}\end{quote}

\end{fulllineitems}

\index{initialize() (state.State method)}

\begin{fulllineitems}
\phantomsection\label{state:state.State.initialize}\pysiglinewithargsret{\bfcode{initialize}}{}{}
Docstring.

\end{fulllineitems}

\index{message\_handler() (state.State method)}

\begin{fulllineitems}
\phantomsection\label{state:state.State.message_handler}\pysiglinewithargsret{\bfcode{message\_handler}}{\emph{message}}{}
Message handler for received messages.
\begin{quote}\begin{description}
\item[{Parameters}] \leavevmode
\textbf{message} (an instance of {\hyperref[swnp:swnp.Message]{\code{swnp.Message}}}) -- Received message.

\end{description}\end{quote}

\end{fulllineitems}

\index{on\_project\_selected() (state.State method)}

\begin{fulllineitems}
\phantomsection\label{state:state.State.on_project_selected}\pysiglinewithargsret{\bfcode{on\_project\_selected}}{}{}
Docstring.

\end{fulllineitems}

\index{remove\_observers() (state.State method)}

\begin{fulllineitems}
\phantomsection\label{state:state.State.remove_observers}\pysiglinewithargsret{\bfcode{remove\_observers}}{}{}
Docstring.

\end{fulllineitems}

\index{set\_current\_project() (state.State method)}

\begin{fulllineitems}
\phantomsection\label{state:state.State.set_current_project}\pysiglinewithargsret{\bfcode{set\_current\_project}}{\emph{project\_id}}{}
Start current project loop.
\begin{quote}\begin{description}
\item[{Parameters}] \leavevmode
\textbf{project\_id} (\emph{Integer}) -- The project id from database.

\end{description}\end{quote}

\end{fulllineitems}

\index{set\_current\_session() (state.State method)}

\begin{fulllineitems}
\phantomsection\label{state:state.State.set_current_session}\pysiglinewithargsret{\bfcode{set\_current\_session}}{\emph{session\_id}}{}
Set current session.
\begin{quote}\begin{description}
\item[{Parameters}] \leavevmode
\textbf{session\_id} (\emph{Integer}) -- a session id from database.

\end{description}\end{quote}

\end{fulllineitems}

\index{set\_observers() (state.State method)}

\begin{fulllineitems}
\phantomsection\label{state:state.State.set_observers}\pysiglinewithargsret{\bfcode{set\_observers}}{}{}
Docstring.

\end{fulllineitems}

\index{set\_project\_observer() (state.State method)}

\begin{fulllineitems}
\phantomsection\label{state:state.State.set_project_observer}\pysiglinewithargsret{\bfcode{set\_project\_observer}}{}{}
Observer for file changes in project directory.

\end{fulllineitems}

\index{set\_responsive() (state.State method)}

\begin{fulllineitems}
\phantomsection\label{state:state.State.set_responsive}\pysiglinewithargsret{\bfcode{set\_responsive}}{}{}
Docstring.

\end{fulllineitems}

\index{set\_scan\_observer() (state.State method)}

\begin{fulllineitems}
\phantomsection\label{state:state.State.set_scan_observer}\pysiglinewithargsret{\bfcode{set\_scan\_observer}}{}{}
Observer for created files in scanned or taken with camera.

\end{fulllineitems}

\index{start\_audio\_recorder() (state.State method)}

\begin{fulllineitems}
\phantomsection\label{state:state.State.start_audio_recorder}\pysiglinewithargsret{\bfcode{start\_audio\_recorder}}{}{}
Starts the audio recorder thread.

\end{fulllineitems}

\index{start\_current\_project() (state.State method)}

\begin{fulllineitems}
\phantomsection\label{state:state.State.start_current_project}\pysiglinewithargsret{\bfcode{start\_current\_project}}{}{}
Start current project loop.

\end{fulllineitems}

\index{start\_current\_session() (state.State method)}

\begin{fulllineitems}
\phantomsection\label{state:state.State.start_current_session}\pysiglinewithargsret{\bfcode{start\_current\_session}}{}{}
Start current project loop.

\end{fulllineitems}

\index{start\_new\_session() (state.State method)}

\begin{fulllineitems}
\phantomsection\label{state:state.State.start_new_session}\pysiglinewithargsret{\bfcode{start\_new\_session}}{}{}
Start a new session.

\end{fulllineitems}

\index{stop\_responsive() (state.State method)}

\begin{fulllineitems}
\phantomsection\label{state:state.State.stop_responsive}\pysiglinewithargsret{\bfcode{stop\_responsive}}{}{}
Docstring.

\end{fulllineitems}

\index{swnp\_send() (state.State method)}

\begin{fulllineitems}
\phantomsection\label{state:state.State.swnp_send}\pysiglinewithargsret{\bfcode{swnp\_send}}{\emph{node}, \emph{message}}{}
Sends a message to the node.
\begin{quote}\begin{description}
\item[{Parameters}] \leavevmode\begin{itemize}
\item {} 
\textbf{node} (\emph{String}) -- The node for which to send a message.

\item {} 
\textbf{message} (\emph{String}) -- The message.

\end{itemize}

\end{description}\end{quote}

\end{fulllineitems}


\end{fulllineitems}

\index{create\_config() (in module state)}

\begin{fulllineitems}
\phantomsection\label{state:state.create_config}\pysiglinewithargsret{\code{state.}\bfcode{create\_config}}{}{}
Creates a config file.

\end{fulllineitems}

\index{initialization\_test() (in module state)}

\begin{fulllineitems}
\phantomsection\label{state:state.initialization_test}\pysiglinewithargsret{\code{state.}\bfcode{initialization\_test}}{}{}
Docstring.

\end{fulllineitems}

\index{load\_config() (in module state)}

\begin{fulllineitems}
\phantomsection\label{state:state.load_config}\pysiglinewithargsret{\code{state.}\bfcode{load\_config}}{}{}
Loads a config file or creates one.

\end{fulllineitems}



\section{SWNP module}
\label{swnp::doc}\label{swnp:swnp-module}\label{swnp:module-swnp}\index{swnp (module)}
Created on 30.4.2012
\begin{quote}\begin{description}
\item[{author}] \leavevmode
neriksso

\end{description}\end{quote}
\index{Message (class in swnp)}

\begin{fulllineitems}
\phantomsection\label{swnp:swnp.Message}\pysiglinewithargsret{\strong{class }\code{swnp.}\bfcode{Message}}{\emph{tag}, \emph{prefix}, \emph{payload}}{}
A class representation of a Message.

Messages are divided into three parts: tag, prefix, payload.
Messages are encoded to json for transmission.
\begin{quote}\begin{description}
\item[{Parameters}] \leavevmode\begin{itemize}
\item {} 
\textbf{tag} (\emph{String}) -- tag of the message.

\item {} 
\textbf{prefix} (\emph{String}) -- prefix of the message.

\item {} 
\textbf{payload} (\emph{String}) -- payload of the message.

\end{itemize}

\end{description}\end{quote}
\index{from\_json() (swnp.Message static method)}

\begin{fulllineitems}
\phantomsection\label{swnp:swnp.Message.from_json}\pysiglinewithargsret{\strong{static }\bfcode{from\_json}}{\emph{json\_dict}}{}
Return a message from json.
\begin{quote}\begin{description}
\item[{Parameters}] \leavevmode
\textbf{json\_dict} (\emph{json}) -- The json.

\item[{Returns}] \leavevmode
Initializes a message from JSON object.

\item[{Return type}] \leavevmode
{\hyperref[swnp:swnp.Message]{\code{swnp.Message}}}.

\end{description}\end{quote}

\end{fulllineitems}

\index{to\_dict() (swnp.Message static method)}

\begin{fulllineitems}
\phantomsection\label{swnp:swnp.Message.to_dict}\pysiglinewithargsret{\strong{static }\bfcode{to\_dict}}{\emph{msg}}{}
Return a message in a dict.
\begin{quote}\begin{description}
\item[{Parameters}] \leavevmode
\textbf{msg} ({\hyperref[swnp:swnp.Message]{\code{swnp.Message}}}) -- The message.

\item[{Returns}] \leavevmode
Dictionary representation of the message.

\item[{Return type}] \leavevmode
Dict

\end{description}\end{quote}

\end{fulllineitems}


\end{fulllineitems}

\index{Node (class in swnp)}

\begin{fulllineitems}
\phantomsection\label{swnp:swnp.Node}\pysiglinewithargsret{\strong{class }\code{swnp.}\bfcode{Node}}{\emph{node\_id}, \emph{screens}, \emph{name=None}, \emph{data=None}}{}
A class representation of a node in the network.
\begin{quote}\begin{description}
\item[{Parameters}] \leavevmode\begin{itemize}
\item {} 
\textbf{node\_id} (\emph{Integer}) -- Node id.

\item {} 
\textbf{screens} (\emph{Integer}) -- Amount of visible screens.

\item {} 
\textbf{name} (\emph{String}) -- The name of the node.

\end{itemize}

\end{description}\end{quote}
\index{get\_age() (swnp.Node method)}

\begin{fulllineitems}
\phantomsection\label{swnp:swnp.Node.get_age}\pysiglinewithargsret{\bfcode{get\_age}}{}{}
Return the elapsed time since last refresh.

\end{fulllineitems}

\index{refresh() (swnp.Node method)}

\begin{fulllineitems}
\phantomsection\label{swnp:swnp.Node.refresh}\pysiglinewithargsret{\bfcode{refresh}}{}{}
Updates the timestamp.

\end{fulllineitems}


\end{fulllineitems}

\index{SWNP (class in swnp)}

\begin{fulllineitems}
\phantomsection\label{swnp:swnp.SWNP}\pysiglinewithargsret{\strong{class }\code{swnp.}\bfcode{SWNP}}{\emph{pgm\_group}, \emph{screens=0}, \emph{name=None}, \emph{node\_id=None}, \emph{context=None}, \emph{error\_handler=None}}{}
The main class of swnp.

This class has the required ZeroMQ bindings and is responsible for
communicating with other instances.

\begin{notice}{warning}{Warning:}
Only one instance per computer
\end{notice}
\begin{quote}\begin{description}
\item[{Parameters}] \leavevmode\begin{itemize}
\item {} 
\textbf{pgm\_group} (\emph{Integer}) -- The Multicast Group this node wants to be a part of.

\item {} 
\textbf{screens} (\emph{Integer}) -- The number of visible screens. Defaults to 0.

\item {} 
\textbf{name} (\emph{String}) -- The name of the instance.

\item {} 
\textbf{node\_id} (\emph{Integer}) -- ID of the current instance.

\item {} 
\textbf{context} (\code{zmq.Context}) -- ZeroMQ context to use.

\item {} 
\textbf{error\_handler} (\code{wos.CONN\_ERR\_TH}) -- Error handler for the init constructor.

\end{itemize}

\end{description}\end{quote}
\index{close() (swnp.SWNP method)}

\begin{fulllineitems}
\phantomsection\label{swnp:swnp.SWNP.close}\pysiglinewithargsret{\bfcode{close}}{}{}
Closes all connections and exits.

\end{fulllineitems}

\index{do\_ping() (swnp.SWNP method)}

\begin{fulllineitems}
\phantomsection\label{swnp:swnp.SWNP.do_ping}\pysiglinewithargsret{\bfcode{do\_ping}}{}{}
Send a PING message to the network.

\end{fulllineitems}

\index{find\_node() (swnp.SWNP method)}

\begin{fulllineitems}
\phantomsection\label{swnp:swnp.SWNP.find_node}\pysiglinewithargsret{\bfcode{find\_node}}{\emph{node\_id}}{}
Search the node list for a specific node.
\begin{quote}\begin{description}
\item[{Parameters}] \leavevmode
\textbf{node\_id} (\emph{Integer}) -- The id of the searched node.

\item[{Return type}] \leavevmode
{\hyperref[swnp:swnp.Node]{\code{swnp.Node}}}

\end{description}\end{quote}

\end{fulllineitems}

\index{get\_buffer() (swnp.SWNP method)}

\begin{fulllineitems}
\phantomsection\label{swnp:swnp.SWNP.get_buffer}\pysiglinewithargsret{\bfcode{get\_buffer}}{}{}
Gets the buffered messages and returns them
\begin{quote}\begin{description}
\item[{Returns}] \leavevmode
JSON formated string.

\item[{Return type}] \leavevmode
String

\end{description}\end{quote}

\end{fulllineitems}

\index{get\_list() (swnp.SWNP method)}

\begin{fulllineitems}
\phantomsection\label{swnp:swnp.SWNP.get_list}\pysiglinewithargsret{\bfcode{get\_list}}{}{}
Returns a list of all nodes
\begin{quote}\begin{description}
\item[{Return type}] \leavevmode
list

\end{description}\end{quote}

\end{fulllineitems}

\index{get\_screen\_list() (swnp.SWNP method)}

\begin{fulllineitems}
\phantomsection\label{swnp:swnp.SWNP.get_screen_list}\pysiglinewithargsret{\bfcode{get\_screen\_list}}{}{}
Returns a list of screens nodes.
\begin{quote}\begin{description}
\item[{Return type}] \leavevmode
list.

\end{description}\end{quote}

\end{fulllineitems}

\index{ping\_handler() (swnp.SWNP method)}

\begin{fulllineitems}
\phantomsection\label{swnp:swnp.SWNP.ping_handler}\pysiglinewithargsret{\bfcode{ping\_handler}}{\emph{payload}}{}
A handler for PING messages. Sends update\_screens, if necessary.
\begin{quote}\begin{description}
\item[{Parameters}] \leavevmode
\textbf{payload} (\emph{String}) -- The payload of a PING message.

\end{description}\end{quote}

\end{fulllineitems}

\index{ping\_routine() (swnp.SWNP method)}

\begin{fulllineitems}
\phantomsection\label{swnp:swnp.SWNP.ping_routine}\pysiglinewithargsret{\bfcode{ping\_routine}}{\emph{error\_handler}}{}
A routine for sending PING messages at regular intervals.

\end{fulllineitems}

\index{send() (swnp.SWNP method)}

\begin{fulllineitems}
\phantomsection\label{swnp:swnp.SWNP.send}\pysiglinewithargsret{\bfcode{send}}{\emph{tag}, \emph{prefix}, \emph{message}}{}
Send a message to the network.
\begin{quote}\begin{description}
\item[{Parameters}] \leavevmode\begin{itemize}
\item {} 
\textbf{tag} (\emph{String}) -- The tag of the message; recipient.

\item {} 
\textbf{prefix} (\emph{String}) -- The prefix of the message.

\item {} 
\textbf{message} (\emph{String}) -- The payload of the message.

\end{itemize}

\end{description}\end{quote}

\end{fulllineitems}

\index{set\_name() (swnp.SWNP method)}

\begin{fulllineitems}
\phantomsection\label{swnp:swnp.SWNP.set_name}\pysiglinewithargsret{\bfcode{set\_name}}{\emph{name}}{}
Sets the name for the instance.
\begin{quote}\begin{description}
\item[{Parameters}] \leavevmode
\textbf{name} (\emph{String}) -- New name of the instance.

\end{description}\end{quote}

\end{fulllineitems}

\index{set\_responsive() (swnp.SWNP method)}

\begin{fulllineitems}
\phantomsection\label{swnp:swnp.SWNP.set_responsive}\pysiglinewithargsret{\bfcode{set\_responsive}}{\emph{responsive}}{}
Sets the responsive flag for the instance.
\begin{quote}\begin{description}
\item[{Parameters}] \leavevmode
\textbf{responsive} (\emph{Integer}) -- New number of screens.

\end{description}\end{quote}

\end{fulllineitems}

\index{set\_screens() (swnp.SWNP method)}

\begin{fulllineitems}
\phantomsection\label{swnp:swnp.SWNP.set_screens}\pysiglinewithargsret{\bfcode{set\_screens}}{\emph{screens}}{}
Sets the number of screens for the instance.
\begin{quote}\begin{description}
\item[{Parameters}] \leavevmode
\textbf{screens} (\emph{Integer}) -- New number of screens.

\end{description}\end{quote}

\end{fulllineitems}

\index{shutdown() (swnp.SWNP method)}

\begin{fulllineitems}
\phantomsection\label{swnp:swnp.SWNP.shutdown}\pysiglinewithargsret{\bfcode{shutdown}}{}{}
Shuts down all connections, no exit.

\end{fulllineitems}

\index{start\_sub\_routine() (swnp.SWNP static method)}

\begin{fulllineitems}
\phantomsection\label{swnp:swnp.SWNP.start_sub_routine}\pysiglinewithargsret{\strong{static }\bfcode{start\_sub\_routine}}{\emph{target}, \emph{routine}, \emph{name}, \emph{args}}{}
A wrapper for starting up subroutine threads.
\begin{quote}\begin{description}
\item[{Parameters}] \leavevmode\begin{itemize}
\item {} 
\textbf{target} (\code{threading.Thread}) -- Variable that contains the current thread for routine.

\item {} 
\textbf{routine} -- The routine to run.

\item {} 
\textbf{name} (\emph{String}) -- Name of the routine.

\item {} 
\textbf{args} (\emph{List}) -- Arguments for the routine.

\end{itemize}

\item[{Returns}] \leavevmode
The thread of subroutine.

\item[{Return type}] \leavevmode
\code{threading.Thread}

\end{description}\end{quote}

\end{fulllineitems}

\index{sub\_routine() (swnp.SWNP method)}

\begin{fulllineitems}
\phantomsection\label{swnp:swnp.SWNP.sub_routine}\pysiglinewithargsret{\bfcode{sub\_routine}}{\emph{sub\_urls}}{}
Subscriber routine for the node ID.
\begin{quote}\begin{description}
\item[{Parameters}] \leavevmode
\textbf{sub\_urls} (\emph{List of Strings}) -- Subscribing URLs.

\end{description}\end{quote}

\end{fulllineitems}

\index{sub\_routine\_sys() (swnp.SWNP method)}

\begin{fulllineitems}
\phantomsection\label{swnp:swnp.SWNP.sub_routine_sys}\pysiglinewithargsret{\bfcode{sub\_routine\_sys}}{\emph{sub\_urls}}{}
Subscriber routine for the node ID.
\begin{quote}\begin{description}
\item[{Parameters}] \leavevmode
\textbf{sub\_urls} (\emph{List of Strings}) -- Subscribing URLs.

\end{description}\end{quote}

\end{fulllineitems}

\index{sys\_handler() (swnp.SWNP method)}

\begin{fulllineitems}
\phantomsection\label{swnp:swnp.SWNP.sys_handler}\pysiglinewithargsret{\bfcode{sys\_handler}}{\emph{msg}}{}
Handler for ``SYS'' messages.
\begin{quote}\begin{description}
\item[{Parameters}] \leavevmode
\textbf{msg} ({\hyperref[swnp:swnp.Message]{\code{swnp.Message}}}) -- The received message.

\end{description}\end{quote}

\end{fulllineitems}

\index{timeout\_routine() (swnp.SWNP method)}

\begin{fulllineitems}
\phantomsection\label{swnp:swnp.SWNP.timeout_routine}\pysiglinewithargsret{\bfcode{timeout\_routine}}{}{}
Routine for checking node list and removing nodes with timeout.

\end{fulllineitems}


\end{fulllineitems}

\index{set\_logger\_level() (in module swnp)}

\begin{fulllineitems}
\phantomsection\label{swnp:swnp.set_logger_level}\pysiglinewithargsret{\code{swnp.}\bfcode{set\_logger\_level}}{\emph{level}}{}
Sets the logger level for swnp logger.
\begin{quote}\begin{description}
\item[{Parameters}] \leavevmode
\textbf{level} (\emph{Integer}) -- Level of logging.

\end{description}\end{quote}

\end{fulllineitems}



\section{Testing module}
\label{testing:module-testing}\label{testing:testing-module}\label{testing::doc}\index{testing (module)}
Created on 20.5.2013
\begin{quote}\begin{description}
\item[{author}] \leavevmode
Kristian

\end{description}\end{quote}


\section{Threads package}
\label{threads:threads-package}\label{threads::doc}
Set of threading functionality.


\subsection{threads.audiorecorder module}
\label{threads:threads-audiorecorder-module}\label{threads:module-threads.audiorecorder}\index{threads.audiorecorder (module)}
Created on 5.6.2013
\begin{quote}\begin{description}
\item[{author}] \leavevmode
neriksso

\end{description}\end{quote}
\index{AudioRecorder (class in threads.audiorecorder)}

\begin{fulllineitems}
\phantomsection\label{threads:threads.audiorecorder.AudioRecorder}\pysiglinewithargsret{\strong{class }\code{threads.audiorecorder.}\bfcode{AudioRecorder}}{\emph{parent}}{}
A thread for capturing audio continuously.
It keeps a buffer that can be saved to a file.
By convention AudioRecorder is usually written in mixed case
even as we prefer upper case for threading types.
\begin{quote}\begin{description}
\item[{Parameters}] \leavevmode
\textbf{parent} (\code{threading.Thread}) -- Parent of the thread.

\end{description}\end{quote}
\index{find\_input\_device() (threads.audiorecorder.AudioRecorder method)}

\begin{fulllineitems}
\phantomsection\label{threads:threads.audiorecorder.AudioRecorder.find_input_device}\pysiglinewithargsret{\bfcode{find\_input\_device}}{}{}
Find the microphone device.

\end{fulllineitems}

\index{open\_mic\_stream() (threads.audiorecorder.AudioRecorder method)}

\begin{fulllineitems}
\phantomsection\label{threads:threads.audiorecorder.AudioRecorder.open_mic_stream}\pysiglinewithargsret{\bfcode{open\_mic\_stream}}{}{}
Opens the stream object for microphone.

\end{fulllineitems}

\index{run() (threads.audiorecorder.AudioRecorder method)}

\begin{fulllineitems}
\phantomsection\label{threads:threads.audiorecorder.AudioRecorder.run}\pysiglinewithargsret{\bfcode{run}}{}{}
Continuously record from the microphone to the buffer.

If the buffer is full, the first frame will be removed and
the new block appended.

\end{fulllineitems}

\index{save() (threads.audiorecorder.AudioRecorder method)}

\begin{fulllineitems}
\phantomsection\label{threads:threads.audiorecorder.AudioRecorder.save}\pysiglinewithargsret{\bfcode{save}}{\emph{ide}, \emph{path}}{}
Save the buffer to a file.

\end{fulllineitems}

\index{stop() (threads.audiorecorder.AudioRecorder method)}

\begin{fulllineitems}
\phantomsection\label{threads:threads.audiorecorder.AudioRecorder.stop}\pysiglinewithargsret{\bfcode{stop}}{}{}
Stop audio recorder.

\end{fulllineitems}


\end{fulllineitems}

\index{logger() (in module threads.audiorecorder)}

\begin{fulllineitems}
\phantomsection\label{threads:threads.audiorecorder.logger}\pysiglinewithargsret{\code{threads.audiorecorder.}\bfcode{logger}}{}{}
Get the common logger.

\end{fulllineitems}



\subsection{threads.checkupdate module}
\label{threads:threads-checkupdate-module}\label{threads:module-threads.checkupdate}\index{threads.checkupdate (module)}
Created on 5.6.2013
\begin{quote}\begin{description}
\item[{author}] \leavevmode
neriksso

\end{description}\end{quote}
\index{CHECK\_UPDATE (class in threads.checkupdate)}

\begin{fulllineitems}
\phantomsection\label{threads:threads.checkupdate.CHECK_UPDATE}\pysigline{\strong{class }\code{threads.checkupdate.}\bfcode{CHECK\_UPDATE}}
Thread for checking version updates.
\index{get\_pad() (threads.checkupdate.CHECK\_UPDATE static method)}

\begin{fulllineitems}
\phantomsection\label{threads:threads.checkupdate.CHECK_UPDATE.get_pad}\pysiglinewithargsret{\strong{static }\bfcode{get\_pad}}{}{}
Returns the padfile object using PAD\_URL setting.
\begin{quote}\begin{description}
\item[{Returns}] \leavevmode
A Filelike object with additional methods geturl(), info() and
getcode().

\end{description}\end{quote}

\end{fulllineitems}

\index{run() (threads.checkupdate.CHECK\_UPDATE method)}

\begin{fulllineitems}
\phantomsection\label{threads:threads.checkupdate.CHECK_UPDATE.run}\pysiglinewithargsret{\bfcode{run}}{}{}
Returns weather the update checking was successful.
\begin{quote}\begin{description}
\item[{Return type}] \leavevmode
Boolean

\end{description}\end{quote}

\end{fulllineitems}

\index{show\_dialog() (threads.checkupdate.CHECK\_UPDATE method)}

\begin{fulllineitems}
\phantomsection\label{threads:threads.checkupdate.CHECK_UPDATE.show_dialog}\pysiglinewithargsret{\bfcode{show\_dialog}}{\emph{url}}{}
Shows the dialog that promps the user to download newer version of
the software.
\begin{quote}\begin{description}
\item[{Parameters}] \leavevmode
\textbf{url} (\emph{String}) -- URL address of the new version.

\end{description}\end{quote}

\end{fulllineitems}


\end{fulllineitems}

\index{logger() (in module threads.checkupdate)}

\begin{fulllineitems}
\phantomsection\label{threads:threads.checkupdate.logger}\pysiglinewithargsret{\code{threads.checkupdate.}\bfcode{logger}}{}{}
Get the common logger.

\end{fulllineitems}



\subsection{threads.common module}
\label{threads:module-threads.common}\label{threads:threads-common-module}\index{threads.common (module)}
Created on 5.6.2013
\begin{quote}\begin{description}
\item[{author}] \leavevmode
neriksso

\end{description}\end{quote}


\subsection{threads.connectionerror module}
\label{threads:threads-connectionerror-module}\label{threads:module-threads.connectionerror}\index{threads.connectionerror (module)}
Created on 5.6.2013
\begin{quote}\begin{description}
\item[{author}] \leavevmode
neriksso

\end{description}\end{quote}
\index{CONNECTION\_ERROR\_THREAD (class in threads.connectionerror)}

\begin{fulllineitems}
\phantomsection\label{threads:threads.connectionerror.CONNECTION_ERROR_THREAD}\pysiglinewithargsret{\strong{class }\code{threads.connectionerror.}\bfcode{CONNECTION\_ERROR\_THREAD}}{\emph{parent}}{}
Thread for checking connection errors.
\begin{quote}\begin{description}
\item[{Parameters}] \leavevmode
\textbf{parent} (\emph{wx.Frame}) -- Parent object.

\end{description}\end{quote}
\index{run() (threads.connectionerror.CONNECTION\_ERROR\_THREAD method)}

\begin{fulllineitems}
\phantomsection\label{threads:threads.connectionerror.CONNECTION_ERROR_THREAD.run}\pysiglinewithargsret{\bfcode{run}}{}{}
Starts the thread.

\end{fulllineitems}


\end{fulllineitems}

\index{logger() (in module threads.connectionerror)}

\begin{fulllineitems}
\phantomsection\label{threads:threads.connectionerror.logger}\pysiglinewithargsret{\code{threads.connectionerror.}\bfcode{logger}}{}{}
Get the common logger.

\end{fulllineitems}



\subsection{threads.contextmenu module}
\label{threads:module-threads.contextmenu}\label{threads:threads-contextmenu-module}\index{threads.contextmenu (module)}
Created on 27.6.2013
\begin{quote}\begin{description}
\item[{author}] \leavevmode
neriksso

\end{description}\end{quote}
\index{ContextMenuFailure (class in threads.contextmenu)}

\begin{fulllineitems}
\phantomsection\label{threads:threads.contextmenu.ContextMenuFailure}\pysiglinewithargsret{\strong{class }\code{threads.contextmenu.}\bfcode{ContextMenuFailure}}{\emph{self}, \emph{EventType type=wxEVT\_NULL}, \emph{int winid=0}}{}
Represents a failure of CMFH initialization.

\end{fulllineitems}

\index{SEND\_FILE\_CONTEX\_MENU\_HANDLER (class in threads.contextmenu)}

\begin{fulllineitems}
\phantomsection\label{threads:threads.contextmenu.SEND_FILE_CONTEX_MENU_HANDLER}\pysiglinewithargsret{\strong{class }\code{threads.contextmenu.}\bfcode{SEND\_FILE\_CONTEX\_MENU\_HANDLER}}{\emph{parent}, \emph{context}, \emph{send\_file}, \emph{handle\_file}}{}
Thread for OS context menu actions like file sending to other node.
\begin{quote}\begin{description}
\item[{Parameters}] \leavevmode\begin{itemize}
\item {} 
\textbf{context} (\code{zmq.Context}) -- ZeroMQ Context for creating sockets.

\item {} 
\textbf{send\_file} (\emph{Function}) -- Sends files.

\item {} 
\textbf{handle\_file} (\emph{Function}) -- Handles files.

\end{itemize}

\end{description}\end{quote}
\index{run() (threads.contextmenu.SEND\_FILE\_CONTEX\_MENU\_HANDLER method)}

\begin{fulllineitems}
\phantomsection\label{threads:threads.contextmenu.SEND_FILE_CONTEX_MENU_HANDLER.run}\pysiglinewithargsret{\bfcode{run}}{}{}
Starts the thread.

\end{fulllineitems}

\index{stop() (threads.contextmenu.SEND\_FILE\_CONTEX\_MENU\_HANDLER method)}

\begin{fulllineitems}
\phantomsection\label{threads:threads.contextmenu.SEND_FILE_CONTEX_MENU_HANDLER.stop}\pysiglinewithargsret{\bfcode{stop}}{}{}
Stops the thread.

\end{fulllineitems}


\end{fulllineitems}

\index{logger() (in module threads.contextmenu)}

\begin{fulllineitems}
\phantomsection\label{threads:threads.contextmenu.logger}\pysiglinewithargsret{\code{threads.contextmenu.}\bfcode{logger}}{}{}
Get the common logger.

\end{fulllineitems}



\subsection{threads.current module}
\label{threads:threads-current-module}\label{threads:module-threads.current}\index{threads.current (module)}
Created on 27.6.2013
\begin{quote}\begin{description}
\item[{author}] \leavevmode
neriksso

\end{description}\end{quote}
\index{CURRENT\_PROJECT (class in threads.current)}

\begin{fulllineitems}
\phantomsection\label{threads:threads.current.CURRENT_PROJECT}\pysiglinewithargsret{\strong{class }\code{threads.current.}\bfcode{CURRENT\_PROJECT}}{\emph{swnp}}{}
Thread for transmitting current project selection.
When user selects a project, an instance is started.
When a new selection is made, by any DiWaCS instance,
the old instance is terminated.
\begin{quote}\begin{description}
\item[{Parameters}] \leavevmode\begin{itemize}
\item {} 
\textbf{project\_id} (\emph{Integer}) -- Project id from the database.

\item {} 
\textbf{swnp} ({\hyperref[swnp:swnp.SWNP]{\code{swnp.SWNP}}}) -- SWNP instance for sending data to the network.

\end{itemize}

\end{description}\end{quote}
\index{run() (threads.current.CURRENT\_PROJECT method)}

\begin{fulllineitems}
\phantomsection\label{threads:threads.current.CURRENT_PROJECT.run}\pysiglinewithargsret{\bfcode{run}}{}{}
Starts the thread.

\end{fulllineitems}


\end{fulllineitems}

\index{CURRENT\_SESSION (class in threads.current)}

\begin{fulllineitems}
\phantomsection\label{threads:threads.current.CURRENT_SESSION}\pysiglinewithargsret{\strong{class }\code{threads.current.}\bfcode{CURRENT\_SESSION}}{\emph{swnp}}{}
Thread for transmitting current session id, when one is started by
the user.  When the session is ended, by any DiWaCS instance, the
instance is terminated.
\begin{quote}\begin{description}
\item[{Parameters}] \leavevmode\begin{itemize}
\item {} 
\textbf{session\_id} (\emph{Integer}) -- Session id from the database.

\item {} 
\textbf{swnp} ({\hyperref[swnp:swnp.SWNP]{\code{swnp.SWNP}}}) -- SWNP instance for sending data to the network.

\end{itemize}

\end{description}\end{quote}
\index{run() (threads.current.CURRENT\_SESSION method)}

\begin{fulllineitems}
\phantomsection\label{threads:threads.current.CURRENT_SESSION.run}\pysiglinewithargsret{\bfcode{run}}{}{}
Starts the thread.

\end{fulllineitems}


\end{fulllineitems}

\index{logger() (in module threads.current)}

\begin{fulllineitems}
\phantomsection\label{threads:threads.current.logger}\pysiglinewithargsret{\code{threads.current.}\bfcode{logger}}{}{}
Get the common logger.

\end{fulllineitems}



\subsection{threads.diwathread module}
\label{threads:module-threads.diwathread}\label{threads:threads-diwathread-module}\index{threads.diwathread (module)}
Created on 5.6.2013
\begin{quote}\begin{description}
\item[{author}] \leavevmode
neriksso

\end{description}\end{quote}
\index{DIWA\_THREAD (class in threads.diwathread)}

\begin{fulllineitems}
\phantomsection\label{threads:threads.diwathread.DIWA_THREAD}\pysiglinewithargsret{\strong{class }\code{threads.diwathread.}\bfcode{DIWA\_THREAD}}{\emph{target=None}, \emph{name=None}, \emph{args=()}, \emph{kwargs=None}}{}
Doc string here.
\index{stop() (threads.diwathread.DIWA\_THREAD method)}

\begin{fulllineitems}
\phantomsection\label{threads:threads.diwathread.DIWA_THREAD.stop}\pysiglinewithargsret{\bfcode{stop}}{}{}
Stop the thread.

\end{fulllineitems}

\index{stop\_all() (threads.diwathread.DIWA\_THREAD static method)}

\begin{fulllineitems}
\phantomsection\label{threads:threads.diwathread.DIWA_THREAD.stop_all}\pysiglinewithargsret{\strong{static }\bfcode{stop\_all}}{}{}
Stop all program threads except the calling one.

\end{fulllineitems}

\index{stop\_is\_set() (threads.diwathread.DIWA\_THREAD method)}

\begin{fulllineitems}
\phantomsection\label{threads:threads.diwathread.DIWA_THREAD.stop_is_set}\pysiglinewithargsret{\bfcode{stop\_is\_set}}{}{}
Is the thread supposed to stop.

\end{fulllineitems}


\end{fulllineitems}

\index{TimeoutException}

\begin{fulllineitems}
\phantomsection\label{threads:threads.diwathread.TimeoutException}\pysiglinewithargsret{\strong{exception }\code{threads.diwathread.}\bfcode{TimeoutException}}{\emph{message}}{}
Represents a thread timeout event.

\end{fulllineitems}

\index{logger() (in module threads.diwathread)}

\begin{fulllineitems}
\phantomsection\label{threads:threads.diwathread.logger}\pysiglinewithargsret{\code{threads.diwathread.}\bfcode{logger}}{}{}
Get the common logger.

\end{fulllineitems}



\subsection{threads.inputcapture module}
\label{threads:module-threads.inputcapture}\label{threads:threads-inputcapture-module}\index{threads.inputcapture (module)}
Created on 5.6.2013
\begin{quote}\begin{description}
\item[{author}] \leavevmode
neriksso

\end{description}\end{quote}
\index{INPUT\_CAPTURE (class in threads.inputcapture)}

\begin{fulllineitems}
\phantomsection\label{threads:threads.inputcapture.INPUT_CAPTURE}\pysiglinewithargsret{\strong{class }\code{threads.inputcapture.}\bfcode{INPUT\_CAPTURE}}{\emph{parent}, \emph{swnp}}{}
Thread for capturing input from mouse/keyboard.
\begin{quote}\begin{description}
\item[{Parameters}] \leavevmode\begin{itemize}
\item {} 
\textbf{parent} (\code{GUI}) -- Parent instance.

\item {} 
\textbf{swnp} ({\hyperref[swnp:swnp.SWNP]{\code{swnp.SWNP}}}) -- SWNP instance for sending data to the network.

\end{itemize}

\end{description}\end{quote}
\index{hook() (threads.inputcapture.INPUT\_CAPTURE method)}

\begin{fulllineitems}
\phantomsection\label{threads:threads.inputcapture.INPUT_CAPTURE.hook}\pysiglinewithargsret{\bfcode{hook}}{}{}
Docstring here.

\end{fulllineitems}

\index{on\_keyboard\_event() (threads.inputcapture.INPUT\_CAPTURE method)}

\begin{fulllineitems}
\phantomsection\label{threads:threads.inputcapture.INPUT_CAPTURE.on_keyboard_event}\pysiglinewithargsret{\bfcode{on\_keyboard\_event}}{\emph{event}}{}
Called when keyboard events are received.

\end{fulllineitems}

\index{on\_mouse\_event() (threads.inputcapture.INPUT\_CAPTURE method)}

\begin{fulllineitems}
\phantomsection\label{threads:threads.inputcapture.INPUT_CAPTURE.on_mouse_event}\pysiglinewithargsret{\bfcode{on\_mouse\_event}}{\emph{event}}{}
Called when mouse events are received.
\begin{itemize}
\item {} 
WM\_MOUSEFIRST = 0x200

\item {} 
WM\_MOUSEMOVE = 0x200

\item {} 
WM\_LBUTTONDOWN = 0x201

\item {} 
WM\_LBUTTONUP = 0x202

\item {} 
WM\_LBUTTONDBLCLK = 0x203

\item {} 
WM\_RBUTTONDOWN = 0x204

\item {} 
WM\_RBUTTONUP = 0x205

\item {} 
WM\_RBUTTONDBLCLK = 0x206

\item {} 
WM\_MBUTTONDOWN = 0x207

\item {} 
WM\_MBUTTONUP = 0x208

\item {} 
WM\_MBUTTONDBLCLK = 0x209

\item {} 
WM\_MOUSEWHEEL = 0x20A

\item {} 
WM\_MOUSEHWHEEL = 0x20E

\end{itemize}

\end{fulllineitems}

\index{reset\_mouse\_events() (threads.inputcapture.INPUT\_CAPTURE method)}

\begin{fulllineitems}
\phantomsection\label{threads:threads.inputcapture.INPUT_CAPTURE.reset_mouse_events}\pysiglinewithargsret{\bfcode{reset\_mouse\_events}}{}{}
Docstring here.

\end{fulllineitems}

\index{run() (threads.inputcapture.INPUT\_CAPTURE method)}

\begin{fulllineitems}
\phantomsection\label{threads:threads.inputcapture.INPUT_CAPTURE.run}\pysiglinewithargsret{\bfcode{run}}{}{}
Starts the thread.

\end{fulllineitems}

\index{stop() (threads.inputcapture.INPUT\_CAPTURE method)}

\begin{fulllineitems}
\phantomsection\label{threads:threads.inputcapture.INPUT_CAPTURE.stop}\pysiglinewithargsret{\bfcode{stop}}{}{}
Stops the thread.

\end{fulllineitems}

\index{unhook() (threads.inputcapture.INPUT\_CAPTURE method)}

\begin{fulllineitems}
\phantomsection\label{threads:threads.inputcapture.INPUT_CAPTURE.unhook}\pysiglinewithargsret{\bfcode{unhook}}{}{}
Docstring here.

\end{fulllineitems}


\end{fulllineitems}

\index{MOUSE\_CAPTURE (class in threads.inputcapture)}

\begin{fulllineitems}
\phantomsection\label{threads:threads.inputcapture.MOUSE_CAPTURE}\pysiglinewithargsret{\strong{class }\code{threads.inputcapture.}\bfcode{MOUSE\_CAPTURE}}{\emph{parent}, \emph{swnp}}{}
Docstring.
\index{parse\_mouse\_events() (threads.inputcapture.MOUSE\_CAPTURE method)}

\begin{fulllineitems}
\phantomsection\label{threads:threads.inputcapture.MOUSE_CAPTURE.parse_mouse_events}\pysiglinewithargsret{\bfcode{parse\_mouse\_events}}{}{}
Docstring here.

\end{fulllineitems}


\end{fulllineitems}

\index{logger() (in module threads.inputcapture)}

\begin{fulllineitems}
\phantomsection\label{threads:threads.inputcapture.logger}\pysiglinewithargsret{\code{threads.inputcapture.}\bfcode{logger}}{}{}
Get the common logger.

\end{fulllineitems}

\index{set\_capture() (in module threads.inputcapture)}

\begin{fulllineitems}
\phantomsection\label{threads:threads.inputcapture.set_capture}\pysiglinewithargsret{\code{threads.inputcapture.}\bfcode{set\_capture}}{\emph{value}}{}
Set's the capture value for threads.
\begin{quote}\begin{description}
\item[{Parameters}] \leavevmode
\textbf{value} (\emph{Boolean}) -- Is the capture on.

\end{description}\end{quote}

\end{fulllineitems}



\subsection{threads.worker module}
\label{threads:threads-worker-module}\label{threads:module-threads.worker}\index{threads.worker (module)}
Created on 27.6.2013
\begin{quote}\begin{description}
\item[{author}] \leavevmode
neriksso

\end{description}\end{quote}
\index{WORKER\_THREAD (class in threads.worker)}

\begin{fulllineitems}
\phantomsection\label{threads:threads.worker.WORKER_THREAD}\pysiglinewithargsret{\strong{class }\code{threads.worker.}\bfcode{WORKER\_THREAD}}{\emph{parent}}{}
Worker thread for non-UI jobs.
\index{add\_project\_registry\_entry() (threads.worker.WORKER\_THREAD static method)}

\begin{fulllineitems}
\phantomsection\label{threads:threads.worker.WORKER_THREAD.add_project_registry_entry}\pysiglinewithargsret{\strong{static }\bfcode{add\_project\_registry\_entry}}{\emph{reg\_type}}{}
Adds ``Add to project'' context menu item to registry. The item
will be added to SoftwareClasses\textless{}reg\_type\textgreater{}, where \textless{}reg\_type\textgreater{}
can be e.g. `*' for all files or `Folder' for folders.
\begin{quote}\begin{description}
\item[{Parameters}] \leavevmode
\textbf{reg\_type} (\emph{String}) -- Registry type.

\end{description}\end{quote}

\end{fulllineitems}

\index{add\_registry\_entry() (threads.worker.WORKER\_THREAD static method)}

\begin{fulllineitems}
\phantomsection\label{threads:threads.worker.WORKER_THREAD.add_registry_entry}\pysiglinewithargsret{\strong{static }\bfcode{add\_registry\_entry}}{\emph{name}, \emph{node\_id}}{}
Adds a node to registry.
\begin{quote}\begin{description}
\item[{Parameters}] \leavevmode\begin{itemize}
\item {} 
\textbf{name} (\emph{String}) -- Node name.

\item {} 
\textbf{id} (\emph{Integer}) -- Node id.

\end{itemize}

\end{description}\end{quote}

\end{fulllineitems}

\index{check\_responsive() (threads.worker.WORKER\_THREAD method)}

\begin{fulllineitems}
\phantomsection\label{threads:threads.worker.WORKER_THREAD.check_responsive}\pysiglinewithargsret{\bfcode{check\_responsive}}{}{}
Docstring here.

\end{fulllineitems}

\index{create\_event() (threads.worker.WORKER\_THREAD method)}

\begin{fulllineitems}
\phantomsection\label{threads:threads.worker.WORKER_THREAD.create_event}\pysiglinewithargsret{\bfcode{create\_event}}{\emph{title}}{}
Docstring here.

\end{fulllineitems}

\index{parse\_config() (threads.worker.WORKER\_THREAD method)}

\begin{fulllineitems}
\phantomsection\label{threads:threads.worker.WORKER_THREAD.parse_config}\pysiglinewithargsret{\bfcode{parse\_config}}{\emph{config\_object}}{}
Handles config file settings.

\end{fulllineitems}

\index{remove\_all\_registry\_entries() (threads.worker.WORKER\_THREAD static method)}

\begin{fulllineitems}
\phantomsection\label{threads:threads.worker.WORKER_THREAD.remove_all_registry_entries}\pysiglinewithargsret{\strong{static }\bfcode{remove\_all\_registry\_entries}}{}{}
Removes all related registry entries.

\end{fulllineitems}

\index{run() (threads.worker.WORKER\_THREAD method)}

\begin{fulllineitems}
\phantomsection\label{threads:threads.worker.WORKER_THREAD.run}\pysiglinewithargsret{\bfcode{run}}{}{}
Run the worker thread.

\end{fulllineitems}


\end{fulllineitems}

\index{logger() (in module threads.worker)}

\begin{fulllineitems}
\phantomsection\label{threads:threads.worker.logger}\pysiglinewithargsret{\code{threads.worker.}\bfcode{logger}}{}{}
Get the common logger.

\end{fulllineitems}



\section{Utils module}
\label{utils:utils-module}\label{utils:module-utils}\label{utils::doc}\index{utils (module)}
Recreated on 17.5.2013
\begin{quote}\begin{description}
\item[{author}] \leavevmode
neriksso

\end{description}\end{quote}
\index{DottedIPToInt() (in module utils)}

\begin{fulllineitems}
\phantomsection\label{utils:utils.DottedIPToInt}\pysiglinewithargsret{\code{utils.}\bfcode{DottedIPToInt}}{\emph{dotted\_ip}}{}
Transforms a dotted IP address to Integer.
\begin{quote}\begin{description}
\item[{Parameters}] \leavevmode
\textbf{dotted\_ip} (\emph{String}) -- The IP address.

\item[{Returns}] \leavevmode
The IP address.

\item[{Return type}] \leavevmode
Integer

\end{description}\end{quote}

\end{fulllineitems}

\index{GetLANMachines() (in module utils)}

\begin{fulllineitems}
\phantomsection\label{utils:utils.GetLANMachines}\pysiglinewithargsret{\code{utils.}\bfcode{GetLANMachines}}{\emph{lan\_ip}}{}~\begin{quote}\begin{description}
\item[{Parameters}] \leavevmode
\textbf{lan\_ip} (\emph{string}) -- Local Area Network IP.

\item[{Returns}] \leavevmode
lan machines

\item[{Return type}] \leavevmode
string{[}{]}

\end{description}\end{quote}

\end{fulllineitems}

\index{GetLocalIPAddress() (in module utils)}

\begin{fulllineitems}
\phantomsection\label{utils:utils.GetLocalIPAddress}\pysiglinewithargsret{\code{utils.}\bfcode{GetLocalIPAddress}}{\emph{target}}{}
Used to get local Internet Protocol address.
\begin{quote}\begin{description}
\item[{Returns}] \leavevmode
The current IP address.

\item[{Return type}] \leavevmode
string

\end{description}\end{quote}

\end{fulllineitems}

\index{GetMacForIp() (in module utils)}

\begin{fulllineitems}
\phantomsection\label{utils:utils.GetMacForIp}\pysiglinewithargsret{\code{utils.}\bfcode{GetMacForIp}}{\emph{ip}}{}
Returns the mac address for an local IP address.
\begin{quote}\begin{description}
\item[{Parameters}] \leavevmode
\textbf{ip} (\emph{String}) -- IP address

\end{description}\end{quote}

\end{fulllineitems}

\index{IntToDottedIP() (in module utils)}

\begin{fulllineitems}
\phantomsection\label{utils:utils.IntToDottedIP}\pysiglinewithargsret{\code{utils.}\bfcode{IntToDottedIP}}{\emph{intip}}{}
Transforms an Integer IP address to dotted representation.
\begin{quote}\begin{description}
\item[{Parameters}] \leavevmode
\textbf{intip} (\emph{Integer}) -- The IP

\item[{Returns}] \leavevmode
The IP

\item[{Return type}] \leavevmode
string

\end{description}\end{quote}

\end{fulllineitems}

\index{IterIsLast() (in module utils)}

\begin{fulllineitems}
\phantomsection\label{utils:utils.IterIsLast}\pysiglinewithargsret{\code{utils.}\bfcode{IterIsLast}}{\emph{iterable}}{{ $\rightarrow$ generates (item, islast) pairs.}}
Generates pairs where the first element is an item from the iterable
source and the second element is a boolean flag indicating if it is the
last item in the sequence.
\begin{quote}\begin{description}
\item[{Parameters}] \leavevmode
\textbf{iterable} (\emph{iterable}) -- The iterable element.

\end{description}\end{quote}

\end{fulllineitems}

\index{MapNetworkShare() (in module utils)}

\begin{fulllineitems}
\phantomsection\label{utils:utils.MapNetworkShare}\pysiglinewithargsret{\code{utils.}\bfcode{MapNetworkShare}}{\emph{letter}, \emph{share=None}}{}
Maps the network share to a letter.
\begin{quote}\begin{description}
\item[{Parameters}] \leavevmode\begin{itemize}
\item {} 
\textbf{letter} (\emph{String}) -- The letter for which to map.

\item {} 
\textbf{share} (\emph{String}) -- The network share, defaults to None which unmaps the letter.

\end{itemize}

\end{description}\end{quote}

\end{fulllineitems}

\index{check\_project\_password() (in module utils)}

\begin{fulllineitems}
\phantomsection\label{utils:utils.check_project_password}\pysiglinewithargsret{\code{utils.}\bfcode{check\_project\_password}}{\emph{project\_id}, \emph{password}}{}
Compares the the provided password with the project password.

\end{fulllineitems}

\index{get\_encrypted\_directory\_name() (in module utils)}

\begin{fulllineitems}
\phantomsection\label{utils:utils.get_encrypted_directory_name}\pysiglinewithargsret{\code{utils.}\bfcode{get\_encrypted\_directory\_name}}{\emph{name}, \emph{hashed\_password}}{}
Returns the encrypted name for project directory.

\end{fulllineitems}

\index{hash\_password() (in module utils)}

\begin{fulllineitems}
\phantomsection\label{utils:utils.hash_password}\pysiglinewithargsret{\code{utils.}\bfcode{hash\_password}}{\emph{password}}{}
Hashes the provided password.

\end{fulllineitems}

\index{set\_logger\_level() (in module utils)}

\begin{fulllineitems}
\phantomsection\label{utils:utils.set_logger_level}\pysiglinewithargsret{\code{utils.}\bfcode{set\_logger\_level}}{\emph{level}}{}
Docstring here.

\end{fulllineitems}



\chapter{Bugs}
\label{bugs::doc}\label{bugs:bugs}
\begin{tabulary}{\linewidth}{|L|L|L|}
\hline
\textbf{\relax 
Bug
} & \textbf{\relax 
Description
} & \textbf{\relax 
Status
}\\\hline

Sample bug
 & 
Description for sample
 & 
Open / Closed / Will not be fixed
\\\hline
\end{tabulary}



\chapter{Features}
\label{features::doc}\label{features:features}
\begin{tabulary}{\linewidth}{|L|L|}
\hline
\textbf{\relax 
Feature
} & \textbf{\relax 
Description
}\\\hline

Project
 & 
User can add, edit and select a project
\\\hline

Session
 & 
User can start, end and continue sessions
\\\hline

Event
 & 
User can tag an interesting event during  a session
\\\hline

File Monitoring
 & 
Users' file actions are monitored during a session. It includes opening files.
\\\hline
\end{tabulary}



\chapter{License}
\label{license::doc}\label{license:license}
European Union Public Licence
\begin{enumerate}
\setcounter{enumi}{21}
\item {} 
1.1

\end{enumerate}

EUPL © the European Community 2007

This European Union Public Licence (the “EUPL”) applies to the Work or Software
(as defined below) which is provided under the terms of this Licence. Any use of the
Work, other than as authorised under this Licence is prohibited (to the extent such use
is covered by a right of the copyright holder of the Work).

The Original Work is provided under the terms of this Licence when the Licensor (as
defined below) has placed the following notice immediately following the copyright
notice for the Original Work:
\begin{quote}

Licensed under the EUPL V.1.1
\end{quote}

or has expressed by any other mean his willingness to license under the EUPL.


\section{1. Definitions}
\label{license:definitions}
In this Licence, the following terms have the following meaning:
\begin{itemize}
\item {} \begin{description}
\item[{The Licence:}] \leavevmode
This Licence.

\end{description}

\item {} \begin{description}
\item[{The Original Work or the Software:}] \leavevmode
The software distributed and/or communicated by the Licensor under this Licence,
available as Source Code and also as Executable Code as the case may be.

\end{description}

\item {} \begin{description}
\item[{Derivative Works:}] \leavevmode
The works or software that could be created by the Licensee,
based upon the Original Work or modifications thereof. This Licence does not define
the extent of modification or dependence on the Original Work required in order to
classify a work as a Derivative Work; this extent is determined by copyright law
applicable in the country mentioned in Article 15.

\end{description}

\item {} \begin{description}
\item[{The Work:}] \leavevmode
The Original Work and/or its Derivative Works.

\end{description}

\item {} \begin{description}
\item[{The Source Code:}] \leavevmode
The human-readable form of the Work which is the most convenient for people to study and modify.

\end{description}

\item {} \begin{description}
\item[{The Executable Code:}] \leavevmode
Any code which has generally been compiled and which is meant to be interpreted by a computer as a program.

\end{description}

\item {} \begin{description}
\item[{The Licensor:}] \leavevmode
The natural or legal person that distributes and/or communicates the Work under the Licence.

\end{description}

\item {} \begin{description}
\item[{Contributor(s):}] \leavevmode
Any natural or legal person who modifies the Work under the Licence, or otherwise contributes to the creation of a Derivative Work.

\end{description}

\item {} \begin{description}
\item[{The Licensee or “You”:}] \leavevmode
Any natural or legal person who makes any usage of the Software under the terms of the Licence.

\end{description}

\item {} \begin{description}
\item[{Distribution and/or Communication:}] \leavevmode
Any act of selling, giving, lending, renting,
distributing, communicating, transmitting, or otherwise making available, on-line or
off-line, copies of the Work or providing access to its essential functionalities at the
disposal of any other natural or legal person.

\end{description}

\end{itemize}


\section{2. Scope of the rights granted by the Licence}
\label{license:scope-of-the-rights-granted-by-the-licence}
The Licensor hereby grants You a world-wide, royalty-free, non-exclusive, sublicensable
licence to do the following, for the duration of copyright vested in the
Original Work:
\begin{itemize}
\item {} 
use the Work in any circumstance and for all usage,

\item {} 
reproduce the Work,

\item {} 
modify the Original Work, and make Derivative Works based upon the Work,

\item {} 
communicate to the public, including the right to make available or display the
Work or copies thereof to the public and perform publicly, as the case may be, the Work,

\item {} 
distribute the Work or copies thereof,

\item {} 
lend and rent the Work or copies thereof,

\item {} 
sub-license rights in the Work or copies thereof.

\end{itemize}

Those rights can be exercised on any media, supports and formats, whether now
known or later invented, as far as the applicable law permits so.

In the countries where moral rights apply, the Licensor waives his right to exercise his
moral right to the extent allowed by law in order to make effective the licence of the
economic rights here above listed.

The Licensor grants to the Licensee royalty-free, non exclusive usage rights to any
patents held by the Licensor, to the extent necessary to make use of the rights granted
on the Work under this Licence.


\section{3. Communication of the Source Code}
\label{license:communication-of-the-source-code}
The Licensor may provide the Work either in its Source Code form, or as Executable
Code. If the Work is provided as Executable Code, the Licensor provides in addition a
machine-readable copy of the Source Code of the Work along with each copy of the
Work that the Licensor distributes or indicates, in a notice following the copyright
notice attached to the Work, a repository where the Source Code is easily and freely
accessible for as long as the Licensor continues to distribute and/or communicate the
Work.


\section{4. Limitations on copyright}
\label{license:limitations-on-copyright}
Nothing in this Licence is intended to deprive the Licensee of the benefits from any
exception or limitation to the exclusive rights of the rights owners in the Original
Work or Software, of the exhaustion of those rights or of other applicable limitations
thereto.


\section{5. Obligations of the Licensee}
\label{license:obligations-of-the-licensee}
The grant of the rights mentioned above is subject to some restrictions and obligations
imposed on the Licensee. Those obligations are the following:

Attribution right: the Licensee shall keep intact all copyright, patent or trademarks
notices and all notices that refer to the Licence and to the disclaimer of warranties.
The Licensee must include a copy of such notices and a copy of the Licence with
every copy of the Work he/she distributes and/or communicates. The Licensee must
cause any Derivative Work to carry prominent notices stating that the Work has been
modified and the date of modification.
\begin{description}
\item[{Copyleft clause:}] \leavevmode
If the Licensee distributes and/or communicates copies of the
Original Works or Derivative Works based upon the Original Work, this Distribution
and/or Communication will be done under the terms of this Licence or of a later
version of this Licence unless the Original Work is expressly distributed only under
this version of the Licence. The Licensee (becoming Licensor) cannot offer or impose
any additional terms or conditions on the Work or Derivative Work that alter or
restrict the terms of the Licence.

\item[{Compatibility clause:}] \leavevmode
If the Licensee Distributes and/or Communicates Derivative
Works or copies thereof based upon both the Original Work and another work
licensed under a Compatible Licence, this Distribution and/or Communication can be
done under the terms of this Compatible Licence. For the sake of this clause,
“Compatible Licence” refers to the licences listed in the appendix attached to this
Licence. Should the Licensee’s obligations under the Compatible Licence conflict
with his/her obligations under this Licence, the obligations of the Compatible Licence
shall prevail.

\item[{Provision of Source Code:}] \leavevmode
When distributing and/or communicating copies of the
Work, the Licensee will provide a machine-readable copy of the Source Code or
indicate a repository where this Source will be easily and freely available for as long
as the Licensee continues to distribute and/or communicate the Work.

\item[{Legal Protection:}] \leavevmode
This Licence does not grant permission to use the trade names,
trademarks, service marks, or names of the Licensor, except as required for
reasonable and customary use in describing the origin of the Work and reproducing
the content of the copyright notice.

\end{description}


\section{6. Chain of Authorship}
\label{license:chain-of-authorship}
The original Licensor warrants that the copyright in the Original Work granted
hereunder is owned by him/her or licensed to him/her and that he/she has the power
and authority to grant the Licence.

Each Contributor warrants that the copyright in the modifications he/she brings to the
Work are owned by him/her or licensed to him/her and that he/she has the power and
authority to grant the Licence.

Each time You accept the Licence, the original Licensor and subsequent Contributors
grant You a licence to their contributions to the Work, under the terms of this
Licence.


\section{7. Disclaimer of Warranty}
\label{license:disclaimer-of-warranty}
The Work is a work in progress, which is continuously improved by numerous
contributors. It is not a finished work and may therefore contain defects or “bugs”
inherent to this type of software development.

For the above reason, the Work is provided under the Licence on an “as is” basis and
without warranties of any kind concerning the Work, including without limitation
merchantability, fitness for a particular purpose, absence of defects or errors,
accuracy, non-infringement of intellectual property rights other than copyright as
stated in Article 6 of this Licence.

This disclaimer of warranty is an essential part of the Licence and a condition for the
grant of any rights to the Work.


\section{8. Disclaimer of Liability}
\label{license:disclaimer-of-liability}
Except in the cases of wilful misconduct or damages directly caused to natural
persons, the Licensor will in no event be liable for any direct or indirect, material or
moral, damages of any kind, arising out of the Licence or of the use of the Work,
including without limitation, damages for loss of goodwill, work stoppage, computer
failure or malfunction, loss of data or any commercial damage, even if the Licensor
has been advised of the possibility of such damage. However, the Licensor will be
liable under statutory product liability laws as far such laws apply to the Work.


\section{9. Additional agreements}
\label{license:additional-agreements}
While distributing the Original Work or Derivative Works, You may choose to
conclude an additional agreement to offer, and charge a fee for, acceptance of support,
warranty, indemnity, or other liability obligations and/or services consistent with this
Licence. However, in accepting such obligations, You may act only on your own
behalf and on your sole responsibility, not on behalf of the original Licensor or any
other Contributor, and only if You agree to indemnify, defend, and hold each
Contributor harmless for any liability incurred by, or claims asserted against such
Contributor by the fact You have accepted any such warranty or additional liability.


\section{10. Acceptance of the Licence}
\label{license:acceptance-of-the-licence}
The provisions of this Licence can be accepted by clicking on an icon “I agree”
placed under the bottom of a window displaying the text of this Licence or by
affirming consent in any other similar way, in accordance with the rules of applicable
law. Clicking on that icon indicates your clear and irrevocable acceptance of this
Licence and all of its terms and conditions.

Similarly, you irrevocably accept this Licence and all of its terms and conditions by
exercising any rights granted to You by Article 2 of this Licence, such as the use of
the Work, the creation by You of a Derivative Work or the Distribution and/or
Communication by You of the Work or copies thereof.


\section{11. Information to the public}
\label{license:information-to-the-public}
In case of any Distribution and/or Communication of the Work by means of electronic
communication by You (for example, by offering to download the Work from a
remote location) the distribution channel or media (for example, a website) must at
least provide to the public the information requested by the applicable law regarding
the Licensor, the Licence and the way it may be accessible, concluded, stored and
reproduced by the Licensee.


\section{12. Termination of the Licence}
\label{license:termination-of-the-licence}
The Licence and the rights granted hereunder will terminate automatically upon any
breach by the Licensee of the terms of the Licence.

Such a termination will not terminate the licences of any person who has received the
Work from the Licensee under the Licence, provided such persons remain in full
compliance with the Licence.


\section{13. Miscellaneous}
\label{license:miscellaneous}
Without prejudice of Article 9 above, the Licence represents the complete agreement
between the Parties as to the Work licensed hereunder.

If any provision of the Licence is invalid or unenforceable under applicable law, this
will not affect the validity or enforceability of the Licence as a whole. Such provision
will be construed and/or reformed so as necessary to make it valid and enforceable.

The European Commission may publish other linguistic versions and/or new versions
of this Licence, so far this is required and reasonable, without reducing the scope of
the rights granted by the Licence. New versions of the Licence will be published with
a unique version number.

All linguistic versions of this Licence, approved by the European Commission, have
identical value. Parties can take advantage of the linguistic version of their choice.


\section{14. Jurisdiction}
\label{license:jurisdiction}
Any litigation resulting from the interpretation of this License, arising between the
European Commission, as a Licensor, and any Licensee, will be subject to the
jurisdiction of the Court of Justice of the European Communities, as laid down in
article 238 of the Treaty establishing the European Community.

Any litigation arising between Parties, other than the European Commission, and
resulting from the interpretation of this License, will be subject to the exclusive
jurisdiction of the competent court where the Licensor resides or conducts its primary
business.


\section{15. Applicable Law}
\label{license:applicable-law}
This Licence shall be governed by the law of the European Union country where the
Licensor resides or has his registered office.
This licence shall be governed by the Belgian law if:
- a litigation arises between the European Commission, as a Licensor, and any Licensee;
- the Licensor, other than the European Commission, has no residence or registered office inside a European Union country.


\section{Appendix}
\label{license:appendix}\begin{description}
\item[{“Compatible Licences” according to article 5 EUPL are:}] \leavevmode\begin{itemize}
\item {} 
GNU General Public License (GNU GPL) v. 2

\item {} 
Open Software License (OSL) v. 2.1, v. 3.0

\item {} 
Common Public License v. 1.0

\item {} 
Eclipse Public License v. 1.0

\item {} 
Cecill v. 2.0

\end{itemize}

\end{description}


\chapter{User Interface}
\label{ui::doc}\label{ui:user-interface}\begin{figure}[htbp]
\centering
\capstart

\includegraphics{screencap.png}
\caption{The UI of DiWaCS; Several icons for different functions.}\end{figure}

The screen icons identify different screens nodes. The user can drop files on to these icons, causing the dropped files to be opened in the specific node. The arrows control the carousel of nodes, and are visble only if more than three nodes are connected. The drop-down list in holds recently viewed files in the selected project.


\begin{threeparttable}
\capstart\caption{Icons explained}

\begin{tabulary}{\linewidth}{|L|L|}
\hline
\textbf{\relax 
Icon
} & \textbf{\relax 
Description
}\\\hline

Briefcase
 & 
Select a project
\\\hline

Clock
 & 
Start / End a session
\\\hline

Folder
 & 
Open project directory
\\\hline

Note
 & 
Create an Event note
\\\hline

Circle
 & 
Hide the application
\\\hline

Cross
 & 
Exit the application
\\\hline
\end{tabulary}

\end{threeparttable}



\chapter{Indices and tables}
\label{index:indices-and-tables}\begin{itemize}
\item {} 
\emph{genindex}

\item {} 
\emph{modindex}

\item {} 
\emph{search}

\end{itemize}


\renewcommand{\indexname}{Python Module Index}
\begin{theindex}
\def\bigletter#1{{\Large\sffamily#1}\nopagebreak\vspace{1mm}}
\bigletter{a}
\item {\texttt{add\_file}}, \pageref{add_file:module-add_file}
\indexspace
\bigletter{c}
\item {\texttt{controller.activity}}, \pageref{controller:module-controller.activity}
\item {\texttt{controller.common}}, \pageref{controller:module-controller.common}
\item {\texttt{controller.computer}}, \pageref{controller:module-controller.computer}
\item {\texttt{controller.handlers}}, \pageref{controller:module-controller.handlers}
\item {\texttt{controller.project}}, \pageref{controller:module-controller.project}
\item {\texttt{controller.session}}, \pageref{controller:module-controller.session}
\indexspace
\bigletter{d}
\item {\texttt{dialogs}}, \pageref{dialogs:module-dialogs}
\item {\texttt{diwacs}}, \pageref{diwacs:module-diwacs}
\item {\texttt{diwavars}}, \pageref{diwavars:module-diwavars}
\indexspace
\bigletter{f}
\item {\texttt{filesystem}}, \pageref{filesystem:module-filesystem}
\indexspace
\bigletter{g}
\item {\texttt{graphicaldesign}}, \pageref{graphicaldesign:module-graphicaldesign}
\indexspace
\bigletter{m}
\item {\texttt{macro}}, \pageref{macro:module-macro}
\item {\texttt{models}}, \pageref{models:module-models}
\indexspace
\bigletter{s}
\item {\texttt{send\_file\_to}}, \pageref{send_file:module-send_file_to}
\item {\texttt{state}}, \pageref{state:module-state}
\item {\texttt{swnp}}, \pageref{swnp:module-swnp}
\indexspace
\bigletter{t}
\item {\texttt{testing}}, \pageref{testing:module-testing}
\item {\texttt{threads.audiorecorder}}, \pageref{threads:module-threads.audiorecorder}
\item {\texttt{threads.checkupdate}}, \pageref{threads:module-threads.checkupdate}
\item {\texttt{threads.common}}, \pageref{threads:module-threads.common}
\item {\texttt{threads.connectionerror}}, \pageref{threads:module-threads.connectionerror}
\item {\texttt{threads.contextmenu}}, \pageref{threads:module-threads.contextmenu}
\item {\texttt{threads.current}}, \pageref{threads:module-threads.current}
\item {\texttt{threads.diwathread}}, \pageref{threads:module-threads.diwathread}
\item {\texttt{threads.inputcapture}}, \pageref{threads:module-threads.inputcapture}
\item {\texttt{threads.worker}}, \pageref{threads:module-threads.worker}
\indexspace
\bigletter{u}
\item {\texttt{utils}}, \pageref{utils:module-utils}
\end{theindex}

\renewcommand{\indexname}{Index}
\printindex
\end{document}
