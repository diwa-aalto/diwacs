% Generated by Sphinx.
\def\sphinxdocclass{report}
\documentclass[letterpaper,10pt,english]{sphinxmanual}
\usepackage[utf8]{inputenc}
\DeclareUnicodeCharacter{00A0}{\nobreakspace}
\usepackage{cmap}
\usepackage[T1]{fontenc}
\usepackage{babel}
\usepackage{times}
\usepackage[Bjarne]{fncychap}
\usepackage{longtable}
\usepackage{sphinx}
\usepackage{multirow}


\title{DiWaCS Documentation}
\date{May 20, 2013}
\release{0.9.2.1}
\author{Nick Eriksson}
\newcommand{\sphinxlogo}{}
\renewcommand{\releasename}{Release}
\makeindex

\makeatletter
\def\PYG@reset{\let\PYG@it=\relax \let\PYG@bf=\relax%
    \let\PYG@ul=\relax \let\PYG@tc=\relax%
    \let\PYG@bc=\relax \let\PYG@ff=\relax}
\def\PYG@tok#1{\csname PYG@tok@#1\endcsname}
\def\PYG@toks#1+{\ifx\relax#1\empty\else%
    \PYG@tok{#1}\expandafter\PYG@toks\fi}
\def\PYG@do#1{\PYG@bc{\PYG@tc{\PYG@ul{%
    \PYG@it{\PYG@bf{\PYG@ff{#1}}}}}}}
\def\PYG#1#2{\PYG@reset\PYG@toks#1+\relax+\PYG@do{#2}}

\expandafter\def\csname PYG@tok@gd\endcsname{\def\PYG@tc##1{\textcolor[rgb]{0.63,0.00,0.00}{##1}}}
\expandafter\def\csname PYG@tok@gu\endcsname{\let\PYG@bf=\textbf\def\PYG@tc##1{\textcolor[rgb]{0.50,0.00,0.50}{##1}}}
\expandafter\def\csname PYG@tok@gt\endcsname{\def\PYG@tc##1{\textcolor[rgb]{0.00,0.27,0.87}{##1}}}
\expandafter\def\csname PYG@tok@gs\endcsname{\let\PYG@bf=\textbf}
\expandafter\def\csname PYG@tok@gr\endcsname{\def\PYG@tc##1{\textcolor[rgb]{1.00,0.00,0.00}{##1}}}
\expandafter\def\csname PYG@tok@cm\endcsname{\let\PYG@it=\textit\def\PYG@tc##1{\textcolor[rgb]{0.25,0.50,0.56}{##1}}}
\expandafter\def\csname PYG@tok@vg\endcsname{\def\PYG@tc##1{\textcolor[rgb]{0.73,0.38,0.84}{##1}}}
\expandafter\def\csname PYG@tok@m\endcsname{\def\PYG@tc##1{\textcolor[rgb]{0.13,0.50,0.31}{##1}}}
\expandafter\def\csname PYG@tok@mh\endcsname{\def\PYG@tc##1{\textcolor[rgb]{0.13,0.50,0.31}{##1}}}
\expandafter\def\csname PYG@tok@cs\endcsname{\def\PYG@tc##1{\textcolor[rgb]{0.25,0.50,0.56}{##1}}\def\PYG@bc##1{\setlength{\fboxsep}{0pt}\colorbox[rgb]{1.00,0.94,0.94}{\strut ##1}}}
\expandafter\def\csname PYG@tok@ge\endcsname{\let\PYG@it=\textit}
\expandafter\def\csname PYG@tok@vc\endcsname{\def\PYG@tc##1{\textcolor[rgb]{0.73,0.38,0.84}{##1}}}
\expandafter\def\csname PYG@tok@il\endcsname{\def\PYG@tc##1{\textcolor[rgb]{0.13,0.50,0.31}{##1}}}
\expandafter\def\csname PYG@tok@go\endcsname{\def\PYG@tc##1{\textcolor[rgb]{0.20,0.20,0.20}{##1}}}
\expandafter\def\csname PYG@tok@cp\endcsname{\def\PYG@tc##1{\textcolor[rgb]{0.00,0.44,0.13}{##1}}}
\expandafter\def\csname PYG@tok@gi\endcsname{\def\PYG@tc##1{\textcolor[rgb]{0.00,0.63,0.00}{##1}}}
\expandafter\def\csname PYG@tok@gh\endcsname{\let\PYG@bf=\textbf\def\PYG@tc##1{\textcolor[rgb]{0.00,0.00,0.50}{##1}}}
\expandafter\def\csname PYG@tok@ni\endcsname{\let\PYG@bf=\textbf\def\PYG@tc##1{\textcolor[rgb]{0.84,0.33,0.22}{##1}}}
\expandafter\def\csname PYG@tok@nl\endcsname{\let\PYG@bf=\textbf\def\PYG@tc##1{\textcolor[rgb]{0.00,0.13,0.44}{##1}}}
\expandafter\def\csname PYG@tok@nn\endcsname{\let\PYG@bf=\textbf\def\PYG@tc##1{\textcolor[rgb]{0.05,0.52,0.71}{##1}}}
\expandafter\def\csname PYG@tok@no\endcsname{\def\PYG@tc##1{\textcolor[rgb]{0.38,0.68,0.84}{##1}}}
\expandafter\def\csname PYG@tok@na\endcsname{\def\PYG@tc##1{\textcolor[rgb]{0.25,0.44,0.63}{##1}}}
\expandafter\def\csname PYG@tok@nb\endcsname{\def\PYG@tc##1{\textcolor[rgb]{0.00,0.44,0.13}{##1}}}
\expandafter\def\csname PYG@tok@nc\endcsname{\let\PYG@bf=\textbf\def\PYG@tc##1{\textcolor[rgb]{0.05,0.52,0.71}{##1}}}
\expandafter\def\csname PYG@tok@nd\endcsname{\let\PYG@bf=\textbf\def\PYG@tc##1{\textcolor[rgb]{0.33,0.33,0.33}{##1}}}
\expandafter\def\csname PYG@tok@ne\endcsname{\def\PYG@tc##1{\textcolor[rgb]{0.00,0.44,0.13}{##1}}}
\expandafter\def\csname PYG@tok@nf\endcsname{\def\PYG@tc##1{\textcolor[rgb]{0.02,0.16,0.49}{##1}}}
\expandafter\def\csname PYG@tok@si\endcsname{\let\PYG@it=\textit\def\PYG@tc##1{\textcolor[rgb]{0.44,0.63,0.82}{##1}}}
\expandafter\def\csname PYG@tok@s2\endcsname{\def\PYG@tc##1{\textcolor[rgb]{0.25,0.44,0.63}{##1}}}
\expandafter\def\csname PYG@tok@vi\endcsname{\def\PYG@tc##1{\textcolor[rgb]{0.73,0.38,0.84}{##1}}}
\expandafter\def\csname PYG@tok@nt\endcsname{\let\PYG@bf=\textbf\def\PYG@tc##1{\textcolor[rgb]{0.02,0.16,0.45}{##1}}}
\expandafter\def\csname PYG@tok@nv\endcsname{\def\PYG@tc##1{\textcolor[rgb]{0.73,0.38,0.84}{##1}}}
\expandafter\def\csname PYG@tok@s1\endcsname{\def\PYG@tc##1{\textcolor[rgb]{0.25,0.44,0.63}{##1}}}
\expandafter\def\csname PYG@tok@gp\endcsname{\let\PYG@bf=\textbf\def\PYG@tc##1{\textcolor[rgb]{0.78,0.36,0.04}{##1}}}
\expandafter\def\csname PYG@tok@sh\endcsname{\def\PYG@tc##1{\textcolor[rgb]{0.25,0.44,0.63}{##1}}}
\expandafter\def\csname PYG@tok@ow\endcsname{\let\PYG@bf=\textbf\def\PYG@tc##1{\textcolor[rgb]{0.00,0.44,0.13}{##1}}}
\expandafter\def\csname PYG@tok@sx\endcsname{\def\PYG@tc##1{\textcolor[rgb]{0.78,0.36,0.04}{##1}}}
\expandafter\def\csname PYG@tok@bp\endcsname{\def\PYG@tc##1{\textcolor[rgb]{0.00,0.44,0.13}{##1}}}
\expandafter\def\csname PYG@tok@c1\endcsname{\let\PYG@it=\textit\def\PYG@tc##1{\textcolor[rgb]{0.25,0.50,0.56}{##1}}}
\expandafter\def\csname PYG@tok@kc\endcsname{\let\PYG@bf=\textbf\def\PYG@tc##1{\textcolor[rgb]{0.00,0.44,0.13}{##1}}}
\expandafter\def\csname PYG@tok@c\endcsname{\let\PYG@it=\textit\def\PYG@tc##1{\textcolor[rgb]{0.25,0.50,0.56}{##1}}}
\expandafter\def\csname PYG@tok@mf\endcsname{\def\PYG@tc##1{\textcolor[rgb]{0.13,0.50,0.31}{##1}}}
\expandafter\def\csname PYG@tok@err\endcsname{\def\PYG@bc##1{\setlength{\fboxsep}{0pt}\fcolorbox[rgb]{1.00,0.00,0.00}{1,1,1}{\strut ##1}}}
\expandafter\def\csname PYG@tok@kd\endcsname{\let\PYG@bf=\textbf\def\PYG@tc##1{\textcolor[rgb]{0.00,0.44,0.13}{##1}}}
\expandafter\def\csname PYG@tok@ss\endcsname{\def\PYG@tc##1{\textcolor[rgb]{0.32,0.47,0.09}{##1}}}
\expandafter\def\csname PYG@tok@sr\endcsname{\def\PYG@tc##1{\textcolor[rgb]{0.14,0.33,0.53}{##1}}}
\expandafter\def\csname PYG@tok@mo\endcsname{\def\PYG@tc##1{\textcolor[rgb]{0.13,0.50,0.31}{##1}}}
\expandafter\def\csname PYG@tok@mi\endcsname{\def\PYG@tc##1{\textcolor[rgb]{0.13,0.50,0.31}{##1}}}
\expandafter\def\csname PYG@tok@kn\endcsname{\let\PYG@bf=\textbf\def\PYG@tc##1{\textcolor[rgb]{0.00,0.44,0.13}{##1}}}
\expandafter\def\csname PYG@tok@o\endcsname{\def\PYG@tc##1{\textcolor[rgb]{0.40,0.40,0.40}{##1}}}
\expandafter\def\csname PYG@tok@kr\endcsname{\let\PYG@bf=\textbf\def\PYG@tc##1{\textcolor[rgb]{0.00,0.44,0.13}{##1}}}
\expandafter\def\csname PYG@tok@s\endcsname{\def\PYG@tc##1{\textcolor[rgb]{0.25,0.44,0.63}{##1}}}
\expandafter\def\csname PYG@tok@kp\endcsname{\def\PYG@tc##1{\textcolor[rgb]{0.00,0.44,0.13}{##1}}}
\expandafter\def\csname PYG@tok@w\endcsname{\def\PYG@tc##1{\textcolor[rgb]{0.73,0.73,0.73}{##1}}}
\expandafter\def\csname PYG@tok@kt\endcsname{\def\PYG@tc##1{\textcolor[rgb]{0.56,0.13,0.00}{##1}}}
\expandafter\def\csname PYG@tok@sc\endcsname{\def\PYG@tc##1{\textcolor[rgb]{0.25,0.44,0.63}{##1}}}
\expandafter\def\csname PYG@tok@sb\endcsname{\def\PYG@tc##1{\textcolor[rgb]{0.25,0.44,0.63}{##1}}}
\expandafter\def\csname PYG@tok@k\endcsname{\let\PYG@bf=\textbf\def\PYG@tc##1{\textcolor[rgb]{0.00,0.44,0.13}{##1}}}
\expandafter\def\csname PYG@tok@se\endcsname{\let\PYG@bf=\textbf\def\PYG@tc##1{\textcolor[rgb]{0.25,0.44,0.63}{##1}}}
\expandafter\def\csname PYG@tok@sd\endcsname{\let\PYG@it=\textit\def\PYG@tc##1{\textcolor[rgb]{0.25,0.44,0.63}{##1}}}

\def\PYGZbs{\char`\\}
\def\PYGZus{\char`\_}
\def\PYGZob{\char`\{}
\def\PYGZcb{\char`\}}
\def\PYGZca{\char`\^}
\def\PYGZam{\char`\&}
\def\PYGZlt{\char`\<}
\def\PYGZgt{\char`\>}
\def\PYGZsh{\char`\#}
\def\PYGZpc{\char`\%}
\def\PYGZdl{\char`\$}
\def\PYGZhy{\char`\-}
\def\PYGZsq{\char`\'}
\def\PYGZdq{\char`\"}
\def\PYGZti{\char`\~}
% for compatibility with earlier versions
\def\PYGZat{@}
\def\PYGZlb{[}
\def\PYGZrb{]}
\makeatother

\begin{document}

\maketitle
\tableofcontents
\phantomsection\label{index::doc}


DiWaCS is an application developed for \href{https://cse.aalto.fi/research/groups/stratus/research/research-projects/}{DiWa smart space} and should be used \textbf{only} inside \textbf{Diwaamo}. DiWaCS connects to address \textbf{239.128.128.1:5555} using \href{http://code.google.com/p/openpgm/}{Pragmatic General Multicast (PGM)}. DiWaCS is built on \href{http://www.python.org}{Python} and \href{http://www.wxpython.org}{WxPython} is used for UI programming. Currently, only supported platform is \textbf{Windows 7}.
\begin{description}
\item[{Required python modules for DiWaCS:}] \leavevmode\begin{itemize}
\item {} 
Configobj \href{http://www.voidspace.org.uk/python/configobj.html}{http://www.voidspace.org.uk/python/configobj.html}

\item {} 
PIL \href{http://www.pythonware.com/products/pil/}{http://www.pythonware.com/products/pil/}

\item {} 
Python Pubsub \href{http://pubsub.sourceforge.net/}{http://pubsub.sourceforge.net/}

\item {} 
SQLAlchemy \href{http://www.sqlalchemy.org/}{http://www.sqlalchemy.org/}

\item {} 
Watchdog  \href{http://packages.python.org/watchdog/}{http://packages.python.org/watchdog/}

\item {} 
WxPython \href{http://www.wxpython.org}{http://www.wxpython.org}

\item {} 
ZeroMQ \href{http://zeromq.org}{http://zeromq.org} with openpgm support \href{http://code.google.com/p/openpgm/}{http://code.google.com/p/openpgm/}

\end{itemize}

\end{description}

Contents:


\chapter{Automated Code Documentation}
\label{api:automated-code-documentation}\label{api::doc}\label{api:welcome-to-diwacs-documentation}
Documentation generated on 2013-05-20 at 13:41.


\section{Add file module}
\label{api:add-file-module}\label{api:module-add_file}\index{add\_file (module)}
Created on 5.6.2012

@author: neriksso

@requires: ZeroMQ
\begin{quote}\begin{description}
\item[{synopsis}] \leavevmode
Used to add a file in the current project.

\end{description}\end{quote}
\index{main() (in module add\_file)}

\begin{fulllineitems}
\phantomsection\label{api:add_file.main}\pysiglinewithargsret{\code{add\_file.}\bfcode{main}}{}{}
Main function of the sub program.

Sub program is meant to be bound to windows explorer context menu.
Context menu allows the user to quickly add files to project without interacting with DiWaCS directly.

Transmits the add\_file command to DiWaCS via interprocess socket.
\begin{quote}\begin{description}
\item[{Parameters}] \leavevmode
\textbf{filepath} (\emph{String}) -- Path of the file to be added.

\item[{Returns}] \leavevmode
windows success code (0 on success).

\item[{Return type}] \leavevmode
Integer

\end{description}\end{quote}

\end{fulllineitems}



\section{Send file module}
\label{api:module-send_file_to}\label{api:send-file-module}\index{send\_file\_to (module)}
Created on 5.6.2012

@author: neriksso

@requires: ZeroMQ
\begin{quote}\begin{description}
\item[{synopsis}] \leavevmode
Used to send a file to another node.

\end{description}\end{quote}
\index{main() (in module send\_file\_to)}

\begin{fulllineitems}
\phantomsection\label{api:send_file_to.main}\pysiglinewithargsret{\code{send\_file\_to.}\bfcode{main}}{}{}
Main function of the sub program.

Sub program is meant to be bound to windows explorer context menu.
Context menu allows the user to quickly send files without interacting with DiWaCS directly.

Transmits the send\_to command to DiWaCS via interprocess connection.
\begin{quote}\begin{description}
\item[{Parameters}] \leavevmode\begin{itemize}
\item {} 
\textbf{node\_id} (\emph{Integer}) -- ID of the node to send the file to.

\item {} 
\textbf{filepath} (\emph{String}) -- Path of the file to be sent.

\end{itemize}

\item[{Returns}] \leavevmode
windows success code (0 on success).

\item[{Return type}] \leavevmode
Integer

\end{description}\end{quote}

\end{fulllineitems}



\section{Controller module}
\label{api:module-controller}\label{api:controller-module}\index{controller (module)}
Created on 28.5.2012

@author: neriksso
\index{AddComputerToSession() (in module controller)}

\begin{fulllineitems}
\phantomsection\label{api:controller.AddComputerToSession}\pysiglinewithargsret{\code{controller.}\bfcode{AddComputerToSession}}{\emph{session}, \emph{name}, \emph{ip}, \emph{wos\_id}}{}
Adds a computer to a session.
\begin{quote}\begin{description}
\item[{Parameters}] \leavevmode\begin{itemize}
\item {} 
\textbf{session} ({\hyperref[api:models.Session]{\code{models.Session}}}) -- A current session.

\item {} 
\textbf{name} (\emph{String.}) -- A name of the computer.

\item {} 
\textbf{ip} (\emph{Integer.}) -- Computers IP address.

\item {} 
\textbf{wos\_id} (\emph{Integer.}) -- Wos id of the computer.

\end{itemize}

\end{description}\end{quote}

\end{fulllineitems}

\index{AddEvent() (in module controller)}

\begin{fulllineitems}
\phantomsection\label{api:controller.AddEvent}\pysiglinewithargsret{\code{controller.}\bfcode{AddEvent}}{\emph{session\_id}, \emph{title}, \emph{desc}}{}
Adds an event to the database.
\begin{quote}\begin{description}
\item[{Parameters}] \leavevmode\begin{itemize}
\item {} 
\textbf{session} ({\hyperref[api:models.Session]{\code{models.Session}}}) -- The current session.

\item {} 
\textbf{desc} (\emph{String.}) -- Description of the event.

\end{itemize}

\end{description}\end{quote}

\end{fulllineitems}

\index{AddFileToProject() (in module controller)}

\begin{fulllineitems}
\phantomsection\label{api:controller.AddFileToProject}\pysiglinewithargsret{\code{controller.}\bfcode{AddFileToProject}}{\emph{file}, \emph{project\_id}}{}
Add a file to project. Copies it to the folder and adds a record to database.
\begin{quote}\begin{description}
\item[{Parameters}] \leavevmode\begin{itemize}
\item {} 
\textbf{file} (\emph{String}) -- A filepath.

\item {} 
\textbf{project\_id} -- Project id from database.

\end{itemize}

\item[{Returns}] \leavevmode
New filepath.

\item[{Return type}] \leavevmode
String

\end{description}\end{quote}

\end{fulllineitems}

\index{AddProject() (in module controller)}

\begin{fulllineitems}
\phantomsection\label{api:controller.AddProject}\pysiglinewithargsret{\code{controller.}\bfcode{AddProject}}{\emph{data}}{}
Adds a project to database and returns a  project instance
\begin{quote}\begin{description}
\item[{Parameters}] \leavevmode
\textbf{data} (\emph{A dictionary}) -- Project information

\item[{Return type}] \leavevmode
an instance of {\hyperref[api:models.Project]{\code{models.Project}}}

\end{description}\end{quote}

\end{fulllineitems}

\index{ConnectToDatabase() (in module controller)}

\begin{fulllineitems}
\phantomsection\label{api:controller.ConnectToDatabase}\pysiglinewithargsret{\code{controller.}\bfcode{ConnectToDatabase}}{\emph{expire=False}}{}
Connect to the database and return a Session object

\end{fulllineitems}

\index{CreateAll() (in module controller)}

\begin{fulllineitems}
\phantomsection\label{api:controller.CreateAll}\pysiglinewithargsret{\code{controller.}\bfcode{CreateAll}}{}{}
Create tables to the database

\end{fulllineitems}

\index{CreateFileaction() (in module controller)}

\begin{fulllineitems}
\phantomsection\label{api:controller.CreateFileaction}\pysiglinewithargsret{\code{controller.}\bfcode{CreateFileaction}}{\emph{path}, \emph{action}, \emph{session\_id}, \emph{project\_id}}{}
Logs a file action to the database.
\begin{quote}\begin{description}
\item[{Parameters}] \leavevmode\begin{itemize}
\item {} 
\textbf{path} (\emph{String.}) -- Filepath.

\item {} 
\textbf{action} (\emph{Integer.}) -- File action id.

\item {} 
\textbf{session\_id} (\emph{Integer.}) -- Current session id.

\item {} 
\textbf{project\_id} (\emph{Integer.}) -- Project id from database.

\end{itemize}

\end{description}\end{quote}

\end{fulllineitems}

\index{DeleteRecord() (in module controller)}

\begin{fulllineitems}
\phantomsection\label{api:controller.DeleteRecord}\pysiglinewithargsret{\code{controller.}\bfcode{DeleteRecord}}{\emph{Model}, \emph{idNum}}{}
Delete a record from database
\begin{quote}\begin{description}
\item[{Parameters}] \leavevmode\begin{itemize}
\item {} 
\textbf{Model} (\code{sqlalchemy.ext.declarative.declarative\_base()}.) -- The model for which to delete a record.

\item {} 
\textbf{idNum} (\emph{Integer.}) -- Recond id.

\end{itemize}

\end{description}\end{quote}

\end{fulllineitems}

\index{EditProject() (in module controller)}

\begin{fulllineitems}
\phantomsection\label{api:controller.EditProject}\pysiglinewithargsret{\code{controller.}\bfcode{EditProject}}{\emph{idNum}, \emph{row}}{}
Update a project info
\begin{quote}\begin{description}
\item[{Parameters}] \leavevmode\begin{itemize}
\item {} 
\textbf{idNum} (\emph{Integer.}) -- Database id number of the project.

\item {} 
\textbf{row} (\emph{A dictionary}) -- The new project information.

\end{itemize}

\end{description}\end{quote}

\end{fulllineitems}

\index{EndSession() (in module controller)}

\begin{fulllineitems}
\phantomsection\label{api:controller.EndSession}\pysiglinewithargsret{\code{controller.}\bfcode{EndSession}}{\emph{session\_id}}{}
Ends a session, sets its endtime to database. Ends file scanner.
\begin{quote}\begin{description}
\item[{Parameters}] \leavevmode
\textbf{session} ({\hyperref[api:models.Session]{\code{models.Session}}}) -- Current session.

\end{description}\end{quote}

\end{fulllineitems}

\index{FILE\_ACTION\_SCANNER (class in controller)}

\begin{fulllineitems}
\phantomsection\label{api:controller.FILE_ACTION_SCANNER}\pysiglinewithargsret{\strong{class }\code{controller.}\bfcode{FILE\_ACTION\_SCANNER}}{\emph{session\_id}, \emph{project\_id}, \emph{path}}{}
A scanner thread for monitoring user actions (Open, Close, Create, etc..) during a session. Utilizes Nirsoft's tools \href{http://www.nirsoft.net/utils/recent\_files\_view.html}{RecentFilesView} and \href{http://www.nirsoft.net/utils/opened\_files\_view.html}{OpenedFilesView} .
\begin{quote}\begin{description}
\item[{Parameters}] \leavevmode\begin{itemize}
\item {} 
\textbf{session\_id} (\emph{Integer.}) -- Current session id from database.

\item {} 
\textbf{project\_id} (\emph{Integer.}) -- Current project id from database.

\item {} 
\textbf{path} (\emph{String.}) -- Filepath of project folder.

\end{itemize}

\end{description}\end{quote}
\index{run() (controller.FILE\_ACTION\_SCANNER method)}

\begin{fulllineitems}
\phantomsection\label{api:controller.FILE_ACTION_SCANNER.run}\pysiglinewithargsret{\bfcode{run}}{}{}
Starts the thread.

\end{fulllineitems}

\index{stop() (controller.FILE\_ACTION\_SCANNER method)}

\begin{fulllineitems}
\phantomsection\label{api:controller.FILE_ACTION_SCANNER.stop}\pysiglinewithargsret{\bfcode{stop}}{}{}
Stops the thread.

\end{fulllineitems}


\end{fulllineitems}

\index{GetOrCreate() (in module controller)}

\begin{fulllineitems}
\phantomsection\label{api:controller.GetOrCreate}\pysiglinewithargsret{\code{controller.}\bfcode{GetOrCreate}}{\emph{session}, \emph{model}, \emph{**kwargs}}{}
Fetches or creates a instance.
\begin{quote}\begin{description}
\item[{Parameters}] \leavevmode\begin{itemize}
\item {} 
\textbf{session} ({\hyperref[api:models.Session]{\code{models.Session}}}) -- a related session

\item {} 
\textbf{model} (\code{sqlalchemy.ext.declarative.declarative\_base()}.) -- The model of which an instance is wanted

\end{itemize}

\end{description}\end{quote}

\end{fulllineitems}

\index{GetProject() (in module controller)}

\begin{fulllineitems}
\phantomsection\label{api:controller.GetProject}\pysiglinewithargsret{\code{controller.}\bfcode{GetProject}}{\emph{project\_id}}{}
Fetches projects by a company.
\begin{quote}\begin{description}
\item[{Parameters}] \leavevmode
\textbf{company\_id} (\emph{Integer.}) -- A company id from database.

\end{description}\end{quote}

\end{fulllineitems}

\index{GetProjectPath() (in module controller)}

\begin{fulllineitems}
\phantomsection\label{api:controller.GetProjectPath}\pysiglinewithargsret{\code{controller.}\bfcode{GetProjectPath}}{\emph{project\_id}}{}
Fetches the project path from database and return it.
\begin{quote}\begin{description}
\item[{Parameters}] \leavevmode
\textbf{project\_id} (\emph{Integer.}) -- Project id for database.

\item[{Return type}] \leavevmode
String.

\end{description}\end{quote}

\end{fulllineitems}

\index{GetProjectsByCompany() (in module controller)}

\begin{fulllineitems}
\phantomsection\label{api:controller.GetProjectsByCompany}\pysiglinewithargsret{\code{controller.}\bfcode{GetProjectsByCompany}}{\emph{company\_id}}{}
Fetches projects by a company.
\begin{quote}\begin{description}
\item[{Parameters}] \leavevmode
\textbf{company\_id} (\emph{Integer.}) -- A company id from database.

\end{description}\end{quote}

\end{fulllineitems}

\index{GetRecentFiles() (in module controller)}

\begin{fulllineitems}
\phantomsection\label{api:controller.GetRecentFiles}\pysiglinewithargsret{\code{controller.}\bfcode{GetRecentFiles}}{\emph{project\_id}}{}
Fetches files accessed recently in the project sessions from the database.

\begin{notice}{note}{Todo}

Add a limit parameter, currently fetches all files.
\end{notice}

\begin{notice}{note}{Todo}

Duplicate check.
\end{notice}
\begin{quote}\begin{description}
\item[{Parameters}] \leavevmode
\textbf{project\_id} (\emph{Integer.}) -- The project id

\item[{Return type}] \leavevmode
a list of files

\end{description}\end{quote}

\end{fulllineitems}

\index{GetSessionsByProject() (in module controller)}

\begin{fulllineitems}
\phantomsection\label{api:controller.GetSessionsByProject}\pysiglinewithargsret{\code{controller.}\bfcode{GetSessionsByProject}}{\emph{project\_id}}{}
Fetches sessions for a project.
\begin{quote}\begin{description}
\item[{Parameters}] \leavevmode
\textbf{project\_id} (\emph{Integer.}) -- Project id from database.

\end{description}\end{quote}

\end{fulllineitems}

\index{InitSyncProjectDir() (in module controller)}

\begin{fulllineitems}
\phantomsection\label{api:controller.InitSyncProjectDir}\pysiglinewithargsret{\code{controller.}\bfcode{InitSyncProjectDir}}{\emph{project\_id}}{}
Initial sync of project dir and database.
\begin{quote}\begin{description}
\item[{Parameters}] \leavevmode
\textbf{project\_id} (\emph{Integer.}) -- Project id from database.

\end{description}\end{quote}

\end{fulllineitems}

\index{IsProjectFile() (in module controller)}

\begin{fulllineitems}
\phantomsection\label{api:controller.IsProjectFile}\pysiglinewithargsret{\code{controller.}\bfcode{IsProjectFile}}{\emph{filename}, \emph{project\_id}}{}
Checks, if a file belongs to a project. Checks both project folder and database.
\begin{quote}\begin{description}
\item[{Parameters}] \leavevmode\begin{itemize}
\item {} 
\textbf{filename} (\emph{String.}) -- a filepath.

\item {} 
\textbf{project\_id} (\emph{Integer.}) -- Project id from database.

\end{itemize}

\item[{Return type}] \leavevmode
Boolean.

\end{description}\end{quote}

\end{fulllineitems}

\index{PROJECT\_FILE\_EVENT\_HANDLER (class in controller)}

\begin{fulllineitems}
\phantomsection\label{api:controller.PROJECT_FILE_EVENT_HANDLER}\pysiglinewithargsret{\strong{class }\code{controller.}\bfcode{PROJECT\_FILE\_EVENT\_HANDLER}}{\emph{project\_id}}{}
Handler for FileSystem events on project folder.
\begin{quote}\begin{description}
\item[{Parameters}] \leavevmode
\textbf{project\_id} (\emph{Integer.}) -- Project id from database.

\end{description}\end{quote}
\index{on\_created() (controller.PROJECT\_FILE\_EVENT\_HANDLER method)}

\begin{fulllineitems}
\phantomsection\label{api:controller.PROJECT_FILE_EVENT_HANDLER.on_created}\pysiglinewithargsret{\bfcode{on\_created}}{\emph{event}}{}
On\_created event handler. Logs to database.
\begin{quote}\begin{description}
\item[{Parameters}] \leavevmode
\textbf{event} (an instance of \code{watchdog.events.FileSystemEvent}) -- The event.

\end{description}\end{quote}

\end{fulllineitems}

\index{on\_deleted() (controller.PROJECT\_FILE\_EVENT\_HANDLER method)}

\begin{fulllineitems}
\phantomsection\label{api:controller.PROJECT_FILE_EVENT_HANDLER.on_deleted}\pysiglinewithargsret{\bfcode{on\_deleted}}{\emph{event}}{}
On\_deleted event handler. Logs to database.
\begin{quote}\begin{description}
\item[{Parameters}] \leavevmode
\textbf{event} (an instance of \code{watchdog.events.FileSystemEvent}) -- The event.

\end{description}\end{quote}

\end{fulllineitems}

\index{on\_modified() (controller.PROJECT\_FILE\_EVENT\_HANDLER method)}

\begin{fulllineitems}
\phantomsection\label{api:controller.PROJECT_FILE_EVENT_HANDLER.on_modified}\pysiglinewithargsret{\bfcode{on\_modified}}{\emph{event}}{}
On\_modified event handler. Logs to database.
\begin{quote}\begin{description}
\item[{Parameters}] \leavevmode
\textbf{event} (an instance of \code{watchdog.events.FileSystemEvent}) -- The event.

\end{description}\end{quote}

\end{fulllineitems}


\end{fulllineitems}

\index{SCAN\_HANDLER (class in controller)}

\begin{fulllineitems}
\phantomsection\label{api:controller.SCAN_HANDLER}\pysiglinewithargsret{\strong{class }\code{controller.}\bfcode{SCAN\_HANDLER}}{\emph{project\_id}}{}
Handler for FileSystem events on SCANNING folder.
\begin{quote}\begin{description}
\item[{Parameters}] \leavevmode
\textbf{project\_id} (\emph{Integer.}) -- Project id from database.

\end{description}\end{quote}
\index{on\_created() (controller.SCAN\_HANDLER method)}

\begin{fulllineitems}
\phantomsection\label{api:controller.SCAN_HANDLER.on_created}\pysiglinewithargsret{\bfcode{on\_created}}{\emph{event}}{}
On\_created event handler. Logs to database.
\begin{quote}\begin{description}
\item[{Parameters}] \leavevmode
\textbf{event} (an instance of \code{watchdog.events.FileSystemEvent}) -- The event.

\end{description}\end{quote}

\end{fulllineitems}


\end{fulllineitems}

\index{StartNewSession() (in module controller)}

\begin{fulllineitems}
\phantomsection\label{api:controller.StartNewSession}\pysiglinewithargsret{\code{controller.}\bfcode{StartNewSession}}{\emph{project\_id}, \emph{session\_id=None}, \emph{old\_session\_id=None}}{}
Creates a session to the database and return a session object.
\begin{quote}\begin{description}
\item[{Parameters}] \leavevmode\begin{itemize}
\item {} 
\textbf{project\_id} (\emph{Integer.}) -- Project id from database.

\item {} 
\textbf{session\_id} (\emph{Integer.}) -- an existing session id from database.

\item {} 
\textbf{old\_session\_id} (\emph{Integer.}) -- A session id of a session which will be continued.

\end{itemize}

\end{description}\end{quote}

\end{fulllineitems}



\section{Models module}
\label{api:module-models}\label{api:models-module}\index{models (module)}
Created on 23.5.2012

@author: neriksso

@requires: \code{sqlalchemy}

@requires: \code{pywin32}
\begin{quote}\begin{description}
\item[{synopsis}] \leavevmode
Used to represent the different database structures on DiWa.

\end{description}\end{quote}
\index{Action (class in models)}

\begin{fulllineitems}
\phantomsection\label{api:models.Action}\pysiglinewithargsret{\strong{class }\code{models.}\bfcode{Action}}{\emph{name}}{}
A class representation of a action. A file action uses this to describe
the action.
\begin{description}
\item[{Field:}] \leavevmode\begin{itemize}
\item {} 
\code{id} (\code{sqlalchemy.schema.Column(sqlalchemy.types.Integer)}) - ID of the action, used as primary key in database table.

\item {} 
\code{name} (\code{sqlalchemy.schema.Column(sqlalchemy.types.String)}) - Name of the action (Max 50 characters).

\end{itemize}

\end{description}
\begin{quote}\begin{description}
\item[{Parameters}] \leavevmode
\textbf{name} (\code{String}) -- Name of the action.

\end{description}\end{quote}

\end{fulllineitems}

\index{Activity (class in models)}

\begin{fulllineitems}
\phantomsection\label{api:models.Activity}\pysiglinewithargsret{\strong{class }\code{models.}\bfcode{Activity}}{\emph{project}, \emph{session=None}}{}
A class representation of an activity.
\begin{description}
\item[{Fields:}] \leavevmode\begin{itemize}
\item {} 
\code{id} (\code{sqlalchemy.schema.Column(sqlalchemy.types.Integer)}) - ID of activity, used as primary key in database table.

\item {} 
\code{session\_id} (\code{sqlalchemy.schema.Column(sqlalchemy.types.Integer)}) - ID of the session activity belongs to.

\item {} 
\code{session} (\code{sqlalchemy.orm.relationship}) - Session relationship.

\item {} 
\code{project\_id} (\code{sqlalchemy.schema.Column(sqlalchemy.types.Integer)}) - ID of the project activity belongs to.

\item {} 
\code{project} (\code{sqlalchemy.orm.relationship}) - Project relationship.

\item {} 
\code{active} (\code{sqlalchemy.schema.Column(sqlalchemy.types.Boolean)}) - Boolean flag indicating that the project is active.

\end{itemize}

\end{description}
\begin{quote}\begin{description}
\item[{Parameters}] \leavevmode\begin{itemize}
\item {} 
\textbf{project} ({\hyperref[api:models.Project]{\code{models.Project}}}) -- Project activity belongs to.

\item {} 
\textbf{session} ({\hyperref[api:models.Session]{\code{models.Session}}}) -- Optional session activity belongs to.

\end{itemize}

\end{description}\end{quote}

\end{fulllineitems}

\index{Company (class in models)}

\begin{fulllineitems}
\phantomsection\label{api:models.Company}\pysiglinewithargsret{\strong{class }\code{models.}\bfcode{Company}}{\emph{name}}{}
A class representation of a company.
\begin{description}
\item[{Fields:}] \leavevmode\begin{itemize}
\item {} 
\code{id} (\code{sqlalchemy.schema.Column(sqlalchemy.types.Integer)}) - ID of the company, used as primary key in database table.

\item {} 
\code{name} (\code{sqlalchemy.schema.Column(sqlalchemy.types.String)}) - Name of the company (Max 50 characters).

\end{itemize}

\end{description}
\begin{quote}\begin{description}
\item[{Parameters}] \leavevmode
\textbf{name} (\code{String}) -- The name of the company.

\end{description}\end{quote}

\end{fulllineitems}

\index{Computer (class in models)}

\begin{fulllineitems}
\phantomsection\label{api:models.Computer}\pysiglinewithargsret{\strong{class }\code{models.}\bfcode{Computer}}{\emph{**kwargs}}{}
A class representation of a computer.
\begin{description}
\item[{Fields:}] \leavevmode\begin{itemize}
\item {} 
\code{id} (\code{sqlalchemy.schema.Column(sqlalchemy.types.Integer)}) - ID of computer, used as primary key in database table.

\item {} 
\code{name} (\code{sqlalchemy.schema.Column(sqlalchemy.types.String)}) - Name of the computer.

\item {} 
\code{ip} (\code{sqlalchemy.schema.Column(sqlalchemy.dialects.mysql.INTEGER)}) - Internet Protocol address of the computer (Defined as unsigned).

\item {} 
\code{mac} (\code{sqlalchemy.schema.Column(sqlalchemy.types.String)}) - Media Access Control address of the computer.

\item {} 
\code{time} (\code{sqlalchemy.schema.Column(sqlalchemy.dialects.mysql.DATETIME)}) - Time of the last network activity from the computer.

\item {} 
\code{screens} (\code{sqlalchemy.schema.Column(sqlalchemy.dialects.mysql.SMALLINT)}) - Number of screens on the computer.

\item {} 
\code{responsive} (\code{sqlalchemy.schema.Column(sqlalchemy.dialects.mysql.TINYINT)}) - The responsive value of the computer.

\item {} 
\code{user\_id} (\code{sqlalchemy.schema.Column(sqlalchemy.types.Integer)}) - ID of the user currently using the computer.

\item {} 
\code{user} (\code{sqlalchemy.orm.relationship}) - The current user.

\item {} 
\code{wos\_id} (\code{sqlalchemy.schema.Column(sqlalchemy.types.Integer)}) - \textbf{WOS} ID.

\end{itemize}

\end{description}

\end{fulllineitems}

\index{Event (class in models)}

\begin{fulllineitems}
\phantomsection\label{api:models.Event}\pysiglinewithargsret{\strong{class }\code{models.}\bfcode{Event}}{\emph{**kwargs}}{}
A class representation of Event. A simple note with timestamp during a
session.
\begin{description}
\item[{Fields:}] \leavevmode\begin{itemize}
\item {} 
\code{id} (\code{sqlalchemy.schema.Column(sqlalchemy.types.Integer)}) - ID of the event, used as primary key in database table.

\item {} 
\code{title} (\code{sqlalchemy.schema.Column(sqlalchemy.types.String)}) - Title of the event (Max 40 characters).

\item {} 
\code{desc} (\code{sqlalchemy.schema.Column(sqlalchemy.types.String)}) - More in-depth description of the event (Max 500 characters).

\item {} 
\code{time} (\code{sqlalchemy.schema.Column(sqlalchemy.dialects.mysql.DATETIME)}) - Time the event took place.

\item {} 
\code{session\_id} (\code{sqlalchemy.schema.Column(sqlalchemy.types.Integer)}) - ID of the session this event belongs to.

\item {} 
\code{session} (\code{sqlalchemy.orm.relationship}) - Session this event belongs to.

\end{itemize}

\end{description}

\end{fulllineitems}

\index{File (class in models)}

\begin{fulllineitems}
\phantomsection\label{api:models.File}\pysiglinewithargsret{\strong{class }\code{models.}\bfcode{File}}{\emph{**kwargs}}{}
A class representation of a file.
\begin{description}
\item[{Fields:}] \leavevmode\begin{itemize}
\item {} 
\code{id} (\code{sqlalchemy.schema.Column(sqlalchemy.types.Integer)}) - ID of the file, used as primary key in database table.

\item {} 
\code{path} (\code{sqlalchemy.schema.Column(sqlalchemy.types.String)}) - Path of the file on DiWa (max 255 chars).

\item {} 
\code{project\_id} (\code{sqlalchemy.schema.Column(sqlalchemy.types.Integer)}) - ID of the project this file belongs to.

\item {} 
\code{project} (\code{sqlalchemy.orm.relationship}) - Project this file belongs to.

\end{itemize}

\end{description}

\end{fulllineitems}

\index{FileAction (class in models)}

\begin{fulllineitems}
\phantomsection\label{api:models.FileAction}\pysiglinewithargsret{\strong{class }\code{models.}\bfcode{FileAction}}{\emph{file}, \emph{action}, \emph{session=None}, \emph{computer=None}, \emph{user=None}}{}
A class representation of a fileaction.
\begin{description}
\item[{Fields:}] \leavevmode\begin{itemize}
\item {} 
\code{id} (\code{sqlalchemy.schema.Column(sqlalchemy.types.Integer)}) - ID of the FileAction, used as primary key in the database table.

\item {} 
\code{file\_id} (\code{sqlalchemy.schema.Column(sqlaclhemy.types.Integer)}) - ID of the file this FileAction affects.

\item {} 
\code{file} (\code{sqlalchemy.orm.relationship)}) - The file this FileAction affects.

\item {} 
\code{action\_id} (\code{sqlalchemy.schema.Column(sqlalchemy.types.Integer)}) - ID of the action affecting the file.

\item {} 
\code{action} (\code{sqlalchemy.orm.relationship)}) - Action affecting the file.

\item {} 
\code{action\_time} (\code{sqlalchemy.schema.Column(sqlalchemy.dialects.mysql.DATETIME)}) - Time the action took place on.

\item {} 
\code{user\_id} (\code{sqlalchemy.schema.Column(sqlalchemy.types.Integer)}) - ID of the user performing the action.

\item {} 
\code{user} (\code{sqlalchemy.orm.relationship}) - User peforming the action.

\item {} 
\code{computer\_id} (\code{sqlalchemy.schema.Column(sqlalchemy.types.Integer)}) - ID of the computer user performed the action on.

\item {} 
\code{computer} (\code{sqlalchemy.orm.relationship}) - Computer user performed the action on.

\item {} 
\code{session\_id} (\code{sqlalchemy.schema.Column(sqlalchemy.types.Integer)}) - ID of the session user performed the action in.

\item {} 
\code{session} (\code{sqlalchemy.orm.relationship}) - Session user performed the action in.

\end{itemize}

\end{description}
\begin{quote}\begin{description}
\item[{Parameters}] \leavevmode\begin{itemize}
\item {} 
\textbf{file} ({\hyperref[api:models.File]{\code{models.File}}}) -- The file which is subjected to the action.

\item {} 
\textbf{action} ({\hyperref[api:models.Action]{\code{models.Action}}}) -- The action which is applied to the file.

\item {} 
\textbf{session} ({\hyperref[api:models.Session]{\code{models.Session}}}) -- The session in which the FileAction took place on.

\item {} 
\textbf{computer} ({\hyperref[api:models.Computer]{\code{models.Computer}}}) -- The computer from which the user performed the action.

\item {} 
\textbf{user} ({\hyperref[api:models.User]{\code{models.User}}}) -- The user performing the action.

\end{itemize}

\end{description}\end{quote}

\end{fulllineitems}

\index{Project (class in models)}

\begin{fulllineitems}
\phantomsection\label{api:models.Project}\pysiglinewithargsret{\strong{class }\code{models.}\bfcode{Project}}{\emph{name}, \emph{company}, \emph{password}}{}
A class representation of a project.
\begin{description}
\item[{Fields:}] \leavevmode\begin{itemize}
\item {} 
\code{id} (\code{sqlalchemy.schema.Column(sqlalchemy.types.Integer)}) - ID of project, used as primary key in database table.

\item {} 
\code{name} (\code{sqlalchemy.schema.Column(sqlalchemy.types.String)}) - Name of the project (Max 50 characters).

\item {} 
\code{company\_id} (\code{sqlalchemy.schema.Column(sqlalchemy.types.Integer)}) - ID of the company that owns the project.

\item {} 
\code{company} (\code{sqlalchemy.orm.relationship}) - The company that owns the project.

\item {} 
\code{dir} (\code{sqlalchemy.schema.Column(sqlalchemy.types.String)}) - Directory path for the project files (Max 255 characters).

\item {} 
\code{password} (\code{sqlalchemy.schema.Column(sqlalchemy.types.String)}) - Password for the project (Max 40 characters).

\item {} 
\code{members} (\code{sqlalchemy.orm.relationship}) - The users that work on the project.

\end{itemize}

\end{description}
\begin{quote}\begin{description}
\item[{Parameters}] \leavevmode\begin{itemize}
\item {} 
\textbf{name} (\code{String}) -- Name of the project.

\item {} 
\textbf{company} ({\hyperref[api:models.Company]{\code{models.Company}}}) -- The owner of the project.

\end{itemize}

\end{description}\end{quote}

\end{fulllineitems}

\index{Session (class in models)}

\begin{fulllineitems}
\phantomsection\label{api:models.Session}\pysiglinewithargsret{\strong{class }\code{models.}\bfcode{Session}}{\emph{project}}{}
A class representation of a session.
\begin{description}
\item[{Fields:}] \leavevmode\begin{itemize}
\item {} 
\code{id} (\code{sqlalchemy.schema.Column(sqlalchemy.types.Integer)}) - ID of session, used as primary key in database table.

\item {} 
\code{name} (\code{sqlalchemy.schema.Column(sqlalchemy.types.String)}) - Name of session (Max 50 characters).

\item {} 
\code{project\_id} (\code{sqlalchemy.schema.Column(sqlalchemy.types.Integer)}) - ID of the project the session belongs to.

\item {} 
\code{project} (\code{sqlalchemy.orm.relationship}) - The project the session belongs to.

\item {} 
\code{starttime} (\code{sqlalchemy.schema.Column(sqlalchemy.dialects.mysql.DATETIME)}) - Time the session began, defaults to \emph{now()}.

\item {} 
\code{endtime} (\code{sqlalchemy.schema.Column(sqlalchemy.dialects.mysql.DATETIME)}) - The time session ended.

\item {} 
\code{previous\_session\_id} (\code{sqlalchemy.schema.Column(sqlalchemy.types.Integer)}) - ID of the previous session.

\item {} 
\code{previous\_session} (\code{sqlalchemy.orm.relationship}) - The previous session.

\item {} 
\code{participants} (\code{sqlalchemy.orm.relationship}) - Users that belong to this session.

\item {} 
\code{computers} (\code{sqlalchemy.orm.relationship}) - Computers that belong to this session.

\end{itemize}

\end{description}
\begin{quote}\begin{description}
\item[{Parameters}] \leavevmode
\textbf{project} ({\hyperref[api:models.Project]{\code{models.Project}}}) -- The project for the session.

\end{description}\end{quote}
\index{addUser() (models.Session method)}

\begin{fulllineitems}
\phantomsection\label{api:models.Session.addUser}\pysiglinewithargsret{\bfcode{addUser}}{\emph{user}}{}
Add users to a session.
\begin{quote}\begin{description}
\item[{Parameters}] \leavevmode
\textbf{user} ({\hyperref[api:models.User]{\code{models.User}}}) -- User to be added into the session.

\end{description}\end{quote}

\end{fulllineitems}

\index{fileRoutine() (models.Session method)}

\begin{fulllineitems}
\phantomsection\label{api:models.Session.fileRoutine}\pysiglinewithargsret{\bfcode{fileRoutine}}{}{}
File checking routine for logging.
\begin{quote}\begin{description}
\item[{Throws IOError}] \leavevmode
When log.txt is not available for write access.

\end{description}\end{quote}

\end{fulllineitems}

\index{get\_last\_checked() (models.Session method)}

\begin{fulllineitems}
\phantomsection\label{api:models.Session.get_last_checked}\pysiglinewithargsret{\bfcode{get\_last\_checked}}{}{}
Fetch \code{last\_checked} field.
\begin{quote}\begin{description}
\item[{Returns}] \leavevmode
\code{last\_checked} field (None before
{\hyperref[api:models.Session.start]{\code{models.Session.start()}}} is called).

\item[{Return type}] \leavevmode
\code{datetime.datetime} or \code{None}

\end{description}\end{quote}

\end{fulllineitems}

\index{start() (models.Session method)}

\begin{fulllineitems}
\phantomsection\label{api:models.Session.start}\pysiglinewithargsret{\bfcode{start}}{}{}
Start a session. Set the \code{last\_checked} field to current
DateTime.

\end{fulllineitems}


\end{fulllineitems}

\index{User (class in models)}

\begin{fulllineitems}
\phantomsection\label{api:models.User}\pysiglinewithargsret{\strong{class }\code{models.}\bfcode{User}}{\emph{name}, \emph{company}}{}
A class representation of a user.
\begin{description}
\item[{Fields:}] \leavevmode\begin{itemize}
\item {} 
\code{id} (\code{sqlalchemy.schema.Column(sqlalchemy.types.Integer)}) - ID of the user, used as primary key in database table.

\item {} 
\code{name} (\code{sqlalchemy.schema.Column(sqlalchemy.types.String)}) - Name of the user (Max 50 characters).

\item {} 
\code{email} (\code{sqlalchemy.schema.Column(sqlalchemy.types.String)}) - Email address of the user (Max 100 characters).

\item {} 
\code{title} (\code{sqlalchemy.schema.Column(sqlalchemy.types.String)}) - Title of the user in the company (Max 50 characters).

\item {} 
\code{department} (\code{sqlalchemy.schema.Column(sqlalchemy.types.String)}) - Department of the user in the company (Max 100 characters).

\item {} 
\code{company\_id} (\code{sqlalchemy.schema.Column(sqlalchemy.types.Integer)}) - Company id of the employing company.

\item {} 
\code{company} (\code{sqlalchemy.orm.relationship}) - Company relationship.

\end{itemize}

\end{description}
\begin{quote}\begin{description}
\item[{Parameters}] \leavevmode\begin{itemize}
\item {} 
\textbf{name} (\code{String}) -- Name of the user.

\item {} 
\textbf{company} ({\hyperref[api:models.Company]{\code{models.Company}}}) -- The employer.

\end{itemize}

\end{description}\end{quote}

\end{fulllineitems}



\section{SWNP module}
\label{api:swnp-module}\label{api:module-swnp}\index{swnp (module)}
Created on 30.4.2012

@author: neriksso
\index{Message (class in swnp)}

\begin{fulllineitems}
\phantomsection\label{api:swnp.Message}\pysiglinewithargsret{\strong{class }\code{swnp.}\bfcode{Message}}{\emph{TAG}, \emph{PREFIX}, \emph{PAYLOAD}}{}
A class representation of a Message.

Messages are divided into three parts: TAG, PREFIX, PAYLOAD. Messages are encoded to json for transmission.
\begin{quote}\begin{description}
\item[{Parameters}] \leavevmode\begin{itemize}
\item {} 
\textbf{TAG} (\emph{String.}) -- TAG of the message.

\item {} 
\textbf{PREFIX} (\emph{String.}) -- PREFIX of the message.

\item {} 
\textbf{PAYLOAD} (\emph{String.}) -- PAYLOAD of the message.

\end{itemize}

\end{description}\end{quote}
\index{from\_json() (swnp.Message static method)}

\begin{fulllineitems}
\phantomsection\label{api:swnp.Message.from_json}\pysiglinewithargsret{\strong{static }\bfcode{from\_json}}{\emph{json\_dict}}{}
Return a message from json.
\begin{quote}\begin{description}
\item[{Parameters}] \leavevmode
\textbf{json\_dict} (\emph{json.}) -- The json.

\item[{Return type}] \leavevmode
{\hyperref[api:swnp.Message]{\code{swnp.Message}}}.

\end{description}\end{quote}

\end{fulllineitems}

\index{to\_dict() (swnp.Message static method)}

\begin{fulllineitems}
\phantomsection\label{api:swnp.Message.to_dict}\pysiglinewithargsret{\strong{static }\bfcode{to\_dict}}{\emph{msg}}{}
Return a message in a dict.
\begin{quote}\begin{description}
\item[{Parameters}] \leavevmode
\textbf{msg} ({\hyperref[api:swnp.Message]{\code{swnp.Message}}}) -- The message.

\item[{Return type}] \leavevmode
Dict.

\end{description}\end{quote}

\end{fulllineitems}


\end{fulllineitems}

\index{Node (class in swnp)}

\begin{fulllineitems}
\phantomsection\label{api:swnp.Node}\pysiglinewithargsret{\strong{class }\code{swnp.}\bfcode{Node}}{\emph{id}, \emph{screens}, \emph{name=None}, \emph{data=None}}{}
A class representation of a node in the network.
\begin{quote}\begin{description}
\item[{Parameters}] \leavevmode\begin{itemize}
\item {} 
\textbf{id} (\emph{Integer.}) -- Node id

\item {} 
\textbf{screens} (\emph{Integer.}) -- Amount of visible screens.

\item {} 
\textbf{name} (\emph{String.}) -- The name of the node.

\end{itemize}

\end{description}\end{quote}
\index{refresh() (swnp.Node method)}

\begin{fulllineitems}
\phantomsection\label{api:swnp.Node.refresh}\pysiglinewithargsret{\bfcode{refresh}}{}{}
Updates the timestamp.

\end{fulllineitems}


\end{fulllineitems}

\index{SWNP (class in swnp)}

\begin{fulllineitems}
\phantomsection\label{api:swnp.SWNP}\pysiglinewithargsret{\strong{class }\code{swnp.}\bfcode{SWNP}}{\emph{pgm\_group}, \emph{screens=0}, \emph{name=None}, \emph{id=None}, \emph{context=None}, \emph{error\_handler=None}}{}
The main class of swnp.

This class has the required ZeroMQ bindings and is responsible for communicating with other instances.

\begin{notice}{warning}{Warning:}
Only one instance per computer
\end{notice}
\begin{quote}\begin{description}
\item[{Parameters}] \leavevmode\begin{itemize}
\item {} 
\textbf{screens} (\emph{Integer.}) -- The number of visible screens. Defaults to 0.

\item {} 
\textbf{name} (\emph{String.}) -- The name of the instance. Optional.

\end{itemize}

\end{description}\end{quote}
\index{close() (swnp.SWNP method)}

\begin{fulllineitems}
\phantomsection\label{api:swnp.SWNP.close}\pysiglinewithargsret{\bfcode{close}}{}{}
Closes all connections and exits.

\end{fulllineitems}

\index{do\_ping() (swnp.SWNP method)}

\begin{fulllineitems}
\phantomsection\label{api:swnp.SWNP.do_ping}\pysiglinewithargsret{\bfcode{do\_ping}}{}{}
Send a PING message to the network.

\end{fulllineitems}

\index{find\_node() (swnp.SWNP method)}

\begin{fulllineitems}
\phantomsection\label{api:swnp.SWNP.find_node}\pysiglinewithargsret{\bfcode{find\_node}}{\emph{node\_id}}{}
Search the node list for a specific node.
\begin{quote}\begin{description}
\item[{Parameters}] \leavevmode
\textbf{node\_id} (\emph{Integer.}) -- The id of the searched node.

\item[{Return type}] \leavevmode
{\hyperref[api:swnp.Node]{\code{swnp.Node}}}

\end{description}\end{quote}

\end{fulllineitems}

\index{get\_buffer() (swnp.SWNP method)}

\begin{fulllineitems}
\phantomsection\label{api:swnp.SWNP.get_buffer}\pysiglinewithargsret{\bfcode{get\_buffer}}{}{}
Gets the buffered messages and returns them
\begin{quote}\begin{description}
\item[{Return type}] \leavevmode
json.

\end{description}\end{quote}

\end{fulllineitems}

\index{get\_list() (swnp.SWNP method)}

\begin{fulllineitems}
\phantomsection\label{api:swnp.SWNP.get_list}\pysiglinewithargsret{\bfcode{get\_list}}{}{}
Returns a list of all nodes
\begin{quote}\begin{description}
\item[{Return type}] \leavevmode
list.

\end{description}\end{quote}

\end{fulllineitems}

\index{get\_screen\_list() (swnp.SWNP method)}

\begin{fulllineitems}
\phantomsection\label{api:swnp.SWNP.get_screen_list}\pysiglinewithargsret{\bfcode{get\_screen\_list}}{}{}
Returns a list of screens nodes.
\begin{quote}\begin{description}
\item[{Return type}] \leavevmode
list.

\end{description}\end{quote}

\end{fulllineitems}

\index{ping\_handler() (swnp.SWNP method)}

\begin{fulllineitems}
\phantomsection\label{api:swnp.SWNP.ping_handler}\pysiglinewithargsret{\bfcode{ping\_handler}}{\emph{payload}}{}
A handler for PING messages. Sends update\_screens, if necessary.
\begin{quote}\begin{description}
\item[{Parameters}] \leavevmode
\textbf{payload} (\emph{String.}) -- The payload of a PING message.

\end{description}\end{quote}

\end{fulllineitems}

\index{ping\_routine() (swnp.SWNP method)}

\begin{fulllineitems}
\phantomsection\label{api:swnp.SWNP.ping_routine}\pysiglinewithargsret{\bfcode{ping\_routine}}{\emph{error\_handler}}{}
A routine for sending PING messages at regular intervals.

\end{fulllineitems}

\index{send() (swnp.SWNP method)}

\begin{fulllineitems}
\phantomsection\label{api:swnp.SWNP.send}\pysiglinewithargsret{\bfcode{send}}{\emph{tag}, \emph{prefix}, \emph{message}}{}
Send a message to the network.
\begin{quote}\begin{description}
\item[{Parameters}] \leavevmode\begin{itemize}
\item {} 
\textbf{tag} (\emph{String.}) -- The tag of the message; recipient.

\item {} 
\textbf{prefix} (\emph{String.}) -- The prefix of the message.

\item {} 
\textbf{message} (\emph{String.}) -- The payload of the message.

\end{itemize}

\end{description}\end{quote}

\end{fulllineitems}

\index{set\_screens() (swnp.SWNP method)}

\begin{fulllineitems}
\phantomsection\label{api:swnp.SWNP.set_screens}\pysiglinewithargsret{\bfcode{set\_screens}}{\emph{screens}}{}
Sets the number of screens for the instance.
\begin{quote}\begin{description}
\item[{Parameters}] \leavevmode
\textbf{screens} (\emph{Integer.}) -- New number of screens.

\end{description}\end{quote}

\end{fulllineitems}

\index{shutdown() (swnp.SWNP method)}

\begin{fulllineitems}
\phantomsection\label{api:swnp.SWNP.shutdown}\pysiglinewithargsret{\bfcode{shutdown}}{}{}
shuts down all connections, no exit.

\end{fulllineitems}

\index{sub\_routine() (swnp.SWNP method)}

\begin{fulllineitems}
\phantomsection\label{api:swnp.SWNP.sub_routine}\pysiglinewithargsret{\bfcode{sub\_routine}}{\emph{sub\_url}, \emph{context}}{}
Subscriber routine for the node ID.
\begin{quote}\begin{description}
\item[{Parameters}] \leavevmode\begin{itemize}
\item {} 
\textbf{sub\_url} (\emph{String}) -- Subscribing URL.

\item {} 
\textbf{context} (\code{zmq.core.context.Context}) -- ZeroMQ context for message sending

\end{itemize}

\end{description}\end{quote}

\end{fulllineitems}

\index{sub\_routine\_sys() (swnp.SWNP method)}

\begin{fulllineitems}
\phantomsection\label{api:swnp.SWNP.sub_routine_sys}\pysiglinewithargsret{\bfcode{sub\_routine\_sys}}{\emph{sub\_url}, \emph{context}}{}
Subscriber routine for the node ID.
\begin{quote}\begin{description}
\item[{Parameters}] \leavevmode\begin{itemize}
\item {} 
\textbf{sub\_url} (\emph{String}) -- Subscribing URL.

\item {} 
\textbf{context} (\code{zmq.core.context.Context}) -- ZeroMQ context for message sending

\end{itemize}

\end{description}\end{quote}

\end{fulllineitems}

\index{sync\_handler() (swnp.SWNP method)}

\begin{fulllineitems}
\phantomsection\label{api:swnp.SWNP.sync_handler}\pysiglinewithargsret{\bfcode{sync\_handler}}{\emph{msg}}{}
Handler for sync messages.

\DUspan{}{Deprecated since version 0.2.}
\begin{quote}\begin{description}
\item[{Parameters}] \leavevmode
\textbf{msg} ({\hyperref[api:swnp.Message]{\code{swnp.Message}}}) -- The message.

\end{description}\end{quote}

\end{fulllineitems}

\index{sys\_handler() (swnp.SWNP method)}

\begin{fulllineitems}
\phantomsection\label{api:swnp.SWNP.sys_handler}\pysiglinewithargsret{\bfcode{sys\_handler}}{\emph{msg}}{}
Handler for ``SYS'' messages.
\begin{quote}\begin{description}
\item[{Parameters}] \leavevmode
\textbf{msg} ({\hyperref[api:swnp.Message]{\code{swnp.Message}}}) -- The received message.

\end{description}\end{quote}

\end{fulllineitems}

\index{timeout\_routine() (swnp.SWNP method)}

\begin{fulllineitems}
\phantomsection\label{api:swnp.SWNP.timeout_routine}\pysiglinewithargsret{\bfcode{timeout\_routine}}{}{}
Routine for checking node list and removing nodes with timeout.

\end{fulllineitems}


\end{fulllineitems}



\section{Filesystem module}
\label{api:module-filesystem}\label{api:filesystem-module}\index{filesystem (module)}
Created on 17.5.2013
\index{CopyFileToProject() (in module filesystem)}

\begin{fulllineitems}
\phantomsection\label{api:filesystem.CopyFileToProject}\pysiglinewithargsret{\code{filesystem.}\bfcode{CopyFileToProject}}{\emph{filepath}, \emph{project\_id}}{}
Copy file to project dir and return new filepath in project dir.
\begin{quote}\begin{description}
\item[{Parameters}] \leavevmode\begin{itemize}
\item {} 
\textbf{filepath} (\emph{String.}) -- The file path.

\item {} 
\textbf{project\_id} (\emph{Integer.}) -- Project id from database.

\end{itemize}

\end{description}\end{quote}

\end{fulllineitems}

\index{CopyToTemp() (in module filesystem)}

\begin{fulllineitems}
\phantomsection\label{api:filesystem.CopyToTemp}\pysiglinewithargsret{\code{filesystem.}\bfcode{CopyToTemp}}{\emph{filepath}}{}
Copy a file to temporary folder.
\begin{quote}\begin{description}
\item[{Parameters}] \leavevmode
\textbf{filepath} (\emph{String.}) -- The file path.

\end{description}\end{quote}

\end{fulllineitems}

\index{CreateProjectDir() (in module filesystem)}

\begin{fulllineitems}
\phantomsection\label{api:filesystem.CreateProjectDir}\pysiglinewithargsret{\code{filesystem.}\bfcode{CreateProjectDir}}{\emph{dir\_name}}{}
Creates a project directory, if one does not exist in the file system
\begin{quote}\begin{description}
\item[{Parameters}] \leavevmode
\textbf{dir\_name} (\emph{String.}) -- Name of the directory

\end{description}\end{quote}

\end{fulllineitems}

\index{FileToBase64() (in module filesystem)}

\begin{fulllineitems}
\phantomsection\label{api:filesystem.FileToBase64}\pysiglinewithargsret{\code{filesystem.}\bfcode{FileToBase64}}{\emph{filepath}}{}
Transform a file to a binary object.
\begin{quote}\begin{description}
\item[{Parameters}] \leavevmode
\textbf{filepath} (\emph{String.}) -- The file path.

\end{description}\end{quote}

\end{fulllineitems}

\index{GetFileExtension() (in module filesystem)}

\begin{fulllineitems}
\phantomsection\label{api:filesystem.GetFileExtension}\pysiglinewithargsret{\code{filesystem.}\bfcode{GetFileExtension}}{\emph{path}}{}
Returns the file extension of a file
\begin{quote}\begin{description}
\item[{Parameters}] \leavevmode
\textbf{path} (\emph{String}) -- The file path.

\item[{Return type}] \leavevmode
String.

\end{description}\end{quote}

\end{fulllineitems}

\index{GetNodeImg() (in module filesystem)}

\begin{fulllineitems}
\phantomsection\label{api:filesystem.GetNodeImg}\pysiglinewithargsret{\code{filesystem.}\bfcode{GetNodeImg}}{\emph{node}}{}
Searches for a node's image in STORAGE.
\begin{quote}\begin{description}
\item[{Parameters}] \leavevmode
\textbf{node} (\emph{Integer.}) -- The node id.

\end{description}\end{quote}

\end{fulllineitems}

\index{IsSubtree() (in module filesystem)}

\begin{fulllineitems}
\phantomsection\label{api:filesystem.IsSubtree}\pysiglinewithargsret{\code{filesystem.}\bfcode{IsSubtree}}{\emph{filename}, \emph{parent}}{}
Determines, if filename is inside the parent folder.
\begin{quote}\begin{description}
\item[{Parameters}] \leavevmode\begin{itemize}
\item {} 
\textbf{filename} (\emph{String.}) -- The file path.

\item {} 
\textbf{parent} (\emph{String.}) -- The parent file path.

\end{itemize}

\end{description}\end{quote}

\end{fulllineitems}

\index{OpenFile() (in module filesystem)}

\begin{fulllineitems}
\phantomsection\label{api:filesystem.OpenFile}\pysiglinewithargsret{\code{filesystem.}\bfcode{OpenFile}}{\emph{filepath}}{}
Opens a file path.
\begin{quote}\begin{description}
\item[{Parameters}] \leavevmode
\textbf{filepath} (\emph{String.}) -- The file path.

\end{description}\end{quote}

\end{fulllineitems}

\index{OpenedFilesQuery() (in module filesystem)}

\begin{fulllineitems}
\phantomsection\label{api:filesystem.OpenedFilesQuery}\pysiglinewithargsret{\code{filesystem.}\bfcode{OpenedFilesQuery}}{}{}
Calls the openedfilesview.

\end{fulllineitems}

\index{RecentFilesQuery() (in module filesystem)}

\begin{fulllineitems}
\phantomsection\label{api:filesystem.RecentFilesQuery}\pysiglinewithargsret{\code{filesystem.}\bfcode{RecentFilesQuery}}{}{}
Calls the recentfilesview.

\end{fulllineitems}

\index{SaveScreen() (in module filesystem)}

\begin{fulllineitems}
\phantomsection\label{api:filesystem.SaveScreen}\pysiglinewithargsret{\code{filesystem.}\bfcode{SaveScreen}}{\emph{win}, \emph{filepath}}{}
Saves the background image of the desktop.
\begin{quote}\begin{description}
\item[{Parameters}] \leavevmode
\textbf{filename} (\emph{String.}) -- The filename for the saved image.

\end{description}\end{quote}

\end{fulllineitems}

\index{SearchFile() (in module filesystem)}

\begin{fulllineitems}
\phantomsection\label{api:filesystem.SearchFile}\pysiglinewithargsret{\code{filesystem.}\bfcode{SearchFile}}{\emph{filename}, \emph{search\_path}}{}
Search file in a given path.
\begin{quote}\begin{description}
\item[{Parameters}] \leavevmode\begin{itemize}
\item {} 
\textbf{filename} (\emph{String.}) -- The file name.

\item {} 
\textbf{search\_path} (\emph{String.}) -- The search path.

\end{itemize}

\end{description}\end{quote}

\end{fulllineitems}



\section{Utils module}
\label{api:utils-module}\label{api:module-utils}\index{utils (module)}
Created on 17.5.2013
\index{DottedIPToInt() (in module utils)}

\begin{fulllineitems}
\phantomsection\label{api:utils.DottedIPToInt}\pysiglinewithargsret{\code{utils.}\bfcode{DottedIPToInt}}{\emph{dotted\_ip}}{}
Transforms a dotted IP address to Integer.
\begin{quote}\begin{description}
\item[{Parameters}] \leavevmode
\textbf{dotted\_ip} (\emph{String}) -- The IP address.

\item[{Returns}] \leavevmode
The IP address.

\item[{Return type}] \leavevmode
Integer

\end{description}\end{quote}

\end{fulllineitems}

\index{GetLANMachines() (in module utils)}

\begin{fulllineitems}
\phantomsection\label{api:utils.GetLANMachines}\pysiglinewithargsret{\code{utils.}\bfcode{GetLANMachines}}{\emph{lan\_ip}}{}~\begin{quote}\begin{description}
\item[{Parameters}] \leavevmode
\textbf{lan\_ip} (\emph{string}) -- Local Area Network IP.

\item[{Returns}] \leavevmode
lan machines

\item[{Return type}] \leavevmode
string{[}{]}

\end{description}\end{quote}

\end{fulllineitems}

\index{GetLocalIPAddress() (in module utils)}

\begin{fulllineitems}
\phantomsection\label{api:utils.GetLocalIPAddress}\pysiglinewithargsret{\code{utils.}\bfcode{GetLocalIPAddress}}{\emph{target}}{}
Used to get local Internet Protocol address.
\begin{quote}\begin{description}
\item[{Returns}] \leavevmode
The current IP address.

\item[{Return type}] \leavevmode
string

\end{description}\end{quote}

\end{fulllineitems}

\index{GetMacForIp() (in module utils)}

\begin{fulllineitems}
\phantomsection\label{api:utils.GetMacForIp}\pysiglinewithargsret{\code{utils.}\bfcode{GetMacForIp}}{\emph{ip}}{}
Returns the mac address for an local IP address.
\begin{quote}\begin{description}
\item[{Parameters}] \leavevmode
\textbf{ip} (\emph{String.}) -- IP address

\end{description}\end{quote}

\end{fulllineitems}

\index{IntToDottedIP() (in module utils)}

\begin{fulllineitems}
\phantomsection\label{api:utils.IntToDottedIP}\pysiglinewithargsret{\code{utils.}\bfcode{IntToDottedIP}}{\emph{intip}}{}
Transforms an Integer IP address to dotted representation.
\begin{quote}\begin{description}
\item[{Parameters}] \leavevmode
\textbf{intip} (\emph{Integer}) -- The IP

\item[{Returns}] \leavevmode
The IP

\item[{Return type}] \leavevmode
string

\end{description}\end{quote}

\end{fulllineitems}

\index{IterIsLast() (in module utils)}

\begin{fulllineitems}
\phantomsection\label{api:utils.IterIsLast}\pysiglinewithargsret{\code{utils.}\bfcode{IterIsLast}}{\emph{iterable}}{{ $\rightarrow$ generates (item, islast) pairs}}
Generates pairs where the first element is an item from the iterable
source and the second element is a boolean flag indicating if it is the
last item in the sequence.
\begin{quote}\begin{description}
\item[{Parameters}] \leavevmode
\textbf{iterable} (\emph{iterable}) -- The iterable element.

\end{description}\end{quote}

\end{fulllineitems}

\index{MapNetworkShare() (in module utils)}

\begin{fulllineitems}
\phantomsection\label{api:utils.MapNetworkShare}\pysiglinewithargsret{\code{utils.}\bfcode{MapNetworkShare}}{\emph{letter}, \emph{share}}{}
Maps the network share to a letter.
\begin{quote}\begin{description}
\item[{Parameters}] \leavevmode\begin{itemize}
\item {} 
\textbf{letter} (\emph{String.}) -- The letter for which to map.

\item {} 
\textbf{share} (\emph{String.}) -- The network share.

\end{itemize}

\end{description}\end{quote}

\end{fulllineitems}



\section{Wos module}
\label{api:wos-module}\label{api:module-wos}\index{wos (module)}
Created on 8.5.2012

@author: neriksso
\index{AddProjectDialog (class in wos)}

\begin{fulllineitems}
\phantomsection\label{api:wos.AddProjectDialog}\pysiglinewithargsret{\strong{class }\code{wos.}\bfcode{AddProjectDialog}}{\emph{parent}, \emph{title}, \emph{project\_id=None}}{}
A dialog for adding a new project
\begin{quote}\begin{description}
\item[{Parameters}] \leavevmode\begin{itemize}
\item {} 
\textbf{parent} (\code{wx.Frame}) -- Parent frame.

\item {} 
\textbf{title} (\emph{String.}) -- A title for the dialog.

\end{itemize}

\end{description}\end{quote}
\index{OnAdd() (wos.AddProjectDialog method)}

\begin{fulllineitems}
\phantomsection\label{api:wos.AddProjectDialog.OnAdd}\pysiglinewithargsret{\bfcode{OnAdd}}{\emph{e}}{}
Handles the addition of a project to database, when ``Add'' button is pressed.
\begin{quote}\begin{description}
\item[{Parameters}] \leavevmode
\textbf{e} (\emph{Event.}) -- GUI Event.

\end{description}\end{quote}

\end{fulllineitems}

\index{OnClose() (wos.AddProjectDialog method)}

\begin{fulllineitems}
\phantomsection\label{api:wos.AddProjectDialog.OnClose}\pysiglinewithargsret{\bfcode{OnClose}}{\emph{e}}{}
Handles ``Close'' button presses
\begin{quote}\begin{description}
\item[{Parameters}] \leavevmode
\textbf{e} (\emph{Event.}) -- GUI Event.

\end{description}\end{quote}

\end{fulllineitems}


\end{fulllineitems}

\index{AudioRecorder (class in wos)}

\begin{fulllineitems}
\phantomsection\label{api:wos.AudioRecorder}\pysiglinewithargsret{\strong{class }\code{wos.}\bfcode{AudioRecorder}}{\emph{parent}}{}
A thread for capturing audio continuously. It keeps a buffer that can be saved to a file.
\index{run() (wos.AudioRecorder method)}

\begin{fulllineitems}
\phantomsection\label{api:wos.AudioRecorder.run}\pysiglinewithargsret{\bfcode{run}}{}{}
Continuously record from the microphone to the buffer. If the buffer is full, the first frame will be removed and the new block appended.

\end{fulllineitems}

\index{save() (wos.AudioRecorder method)}

\begin{fulllineitems}
\phantomsection\label{api:wos.AudioRecorder.save}\pysiglinewithargsret{\bfcode{save}}{\emph{ide}, \emph{path}}{}
Save the buffer to a file.

\end{fulllineitems}


\end{fulllineitems}

\index{CHECK\_UPDATE (class in wos)}

\begin{fulllineitems}
\phantomsection\label{api:wos.CHECK_UPDATE}\pysigline{\strong{class }\code{wos.}\bfcode{CHECK\_UPDATE}}
Thread for checking version updates.

\end{fulllineitems}

\index{CONN\_ERR\_TH (class in wos)}

\begin{fulllineitems}
\phantomsection\label{api:wos.CONN_ERR_TH}\pysiglinewithargsret{\strong{class }\code{wos.}\bfcode{CONN\_ERR\_TH}}{\emph{parent}}{}
Thread for checking connection errors.
\begin{quote}\begin{description}
\item[{Parameters}] \leavevmode
\textbf{parent} (\emph{wx.Frame.}) -- Parent object.

\end{description}\end{quote}
\index{run() (wos.CONN\_ERR\_TH method)}

\begin{fulllineitems}
\phantomsection\label{api:wos.CONN_ERR_TH.run}\pysiglinewithargsret{\bfcode{run}}{}{}
Starts the thread.

\end{fulllineitems}


\end{fulllineitems}

\index{CURRENT\_PROJECT (class in wos)}

\begin{fulllineitems}
\phantomsection\label{api:wos.CURRENT_PROJECT}\pysiglinewithargsret{\strong{class }\code{wos.}\bfcode{CURRENT\_PROJECT}}{\emph{project\_id}, \emph{swnp}}{}
Thread for transmitting current project selection. When user selects a project, an instance is started. When a new selection is made, by any Chimaira instance, the old instance is terminated.
\begin{quote}\begin{description}
\item[{Parameters}] \leavevmode\begin{itemize}
\item {} 
\textbf{project\_id} (\emph{Integer.}) -- Project id from the database.

\item {} 
\textbf{swnp} ({\hyperref[api:swnp.SWNP]{\code{swnp.SWNP}}}) -- SWNP instance for sending data to the network.

\end{itemize}

\end{description}\end{quote}
\index{run() (wos.CURRENT\_PROJECT method)}

\begin{fulllineitems}
\phantomsection\label{api:wos.CURRENT_PROJECT.run}\pysiglinewithargsret{\bfcode{run}}{}{}
Starts the thread.

\end{fulllineitems}

\index{stop() (wos.CURRENT\_PROJECT method)}

\begin{fulllineitems}
\phantomsection\label{api:wos.CURRENT_PROJECT.stop}\pysiglinewithargsret{\bfcode{stop}}{}{}
Stops the thread.

\end{fulllineitems}


\end{fulllineitems}

\index{CURRENT\_SESSION (class in wos)}

\begin{fulllineitems}
\phantomsection\label{api:wos.CURRENT_SESSION}\pysiglinewithargsret{\strong{class }\code{wos.}\bfcode{CURRENT\_SESSION}}{\emph{parent}, \emph{swnp}}{}
Thread for transmitting current session id, when one is started by the user.  When the session is ended, by any Chimaira instance, the instance is terminated.
\begin{quote}\begin{description}
\item[{Parameters}] \leavevmode\begin{itemize}
\item {} 
\textbf{session\_id} (\emph{Integer.}) -- Session id from the database.

\item {} 
\textbf{swnp} ({\hyperref[api:swnp.SWNP]{\code{swnp.SWNP}}}) -- SWNP instance for sending data to the network.

\end{itemize}

\end{description}\end{quote}
\index{run() (wos.CURRENT\_SESSION method)}

\begin{fulllineitems}
\phantomsection\label{api:wos.CURRENT_SESSION.run}\pysiglinewithargsret{\bfcode{run}}{}{}
Starts the thread.

\end{fulllineitems}

\index{stop() (wos.CURRENT\_SESSION method)}

\begin{fulllineitems}
\phantomsection\label{api:wos.CURRENT_SESSION.stop}\pysiglinewithargsret{\bfcode{stop}}{}{}
Stops the thread

\end{fulllineitems}


\end{fulllineitems}

\index{DropTarget (class in wos)}

\begin{fulllineitems}
\phantomsection\label{api:wos.DropTarget}\pysiglinewithargsret{\strong{class }\code{wos.}\bfcode{DropTarget}}{\emph{window}, \emph{parent}, \emph{i}}{}
Implements drop target functionality to receive files, bitmaps and text
\index{OnData() (wos.DropTarget method)}

\begin{fulllineitems}
\phantomsection\label{api:wos.DropTarget.OnData}\pysiglinewithargsret{\bfcode{OnData}}{\emph{x}, \emph{y}, \emph{d}}{}
Handles drag/dropping files/text or a bitmap

\end{fulllineitems}


\end{fulllineitems}

\index{EventList (class in wos)}

\begin{fulllineitems}
\phantomsection\label{api:wos.EventList}\pysiglinewithargsret{\strong{class }\code{wos.}\bfcode{EventList}}{\emph{parent}, \emph{*args}, \emph{**kwargs}}{}
A Frame which displays the possible event titles and handles the event creation.
\index{GetIcon() (wos.EventList method)}

\begin{fulllineitems}
\phantomsection\label{api:wos.EventList.GetIcon}\pysiglinewithargsret{\bfcode{GetIcon}}{\emph{icon}}{}
Fetches gui icons.
\begin{quote}\begin{description}
\item[{Parameters}] \leavevmode
\textbf{icon} (\emph{String.}) -- The icon file name.

\item[{Return type}] \leavevmode
\code{wx.Image}

\end{description}\end{quote}

\end{fulllineitems}


\end{fulllineitems}

\index{GUI (class in wos)}

\begin{fulllineitems}
\phantomsection\label{api:wos.GUI}\pysiglinewithargsret{\strong{class }\code{wos.}\bfcode{GUI}}{\emph{parent}, \emph{title}}{}
WOS Application Frame
\begin{quote}\begin{description}
\item[{Parameters}] \leavevmode\begin{itemize}
\item {} 
\textbf{parent} (\code{wx.Frame}) -- Parent frame.

\item {} 
\textbf{title} (\emph{String.}) -- Title for the frame

\end{itemize}

\end{description}\end{quote}
\index{AlignCenterTop() (wos.GUI method)}

\begin{fulllineitems}
\phantomsection\label{api:wos.GUI.AlignCenterTop}\pysiglinewithargsret{\bfcode{AlignCenterTop}}{}{}
Aligns frame to Horizontal center and vertical top

\end{fulllineitems}

\index{CreateConfig() (wos.GUI method)}

\begin{fulllineitems}
\phantomsection\label{api:wos.GUI.CreateConfig}\pysiglinewithargsret{\bfcode{CreateConfig}}{}{}
Creates a config file

\end{fulllineitems}

\index{GetIcon() (wos.GUI method)}

\begin{fulllineitems}
\phantomsection\label{api:wos.GUI.GetIcon}\pysiglinewithargsret{\bfcode{GetIcon}}{\emph{icon}}{}
Fetches gui icons.
\begin{quote}\begin{description}
\item[{Parameters}] \leavevmode
\textbf{icon} (\emph{String.}) -- The icon file name.

\item[{Return type}] \leavevmode
\code{wx.Image}

\end{description}\end{quote}

\end{fulllineitems}

\index{HandleFileSend() (wos.GUI method)}

\begin{fulllineitems}
\phantomsection\label{api:wos.GUI.HandleFileSend}\pysiglinewithargsret{\bfcode{HandleFileSend}}{\emph{file}}{}
Sends a file link to another node

\end{fulllineitems}

\index{HideScreens() (wos.GUI method)}

\begin{fulllineitems}
\phantomsection\label{api:wos.GUI.HideScreens}\pysiglinewithargsret{\bfcode{HideScreens}}{}{}
Hides all screens

\end{fulllineitems}

\index{InitScreens() (wos.GUI method)}

\begin{fulllineitems}
\phantomsection\label{api:wos.GUI.InitScreens}\pysiglinewithargsret{\bfcode{InitScreens}}{}{}
Inits Screens

\end{fulllineitems}

\index{InitUI() (wos.GUI method)}

\begin{fulllineitems}
\phantomsection\label{api:wos.GUI.InitUI}\pysiglinewithargsret{\bfcode{InitUI}}{}{}
UI initing

\end{fulllineitems}

\index{LoadConfig() (wos.GUI method)}

\begin{fulllineitems}
\phantomsection\label{api:wos.GUI.LoadConfig}\pysiglinewithargsret{\bfcode{LoadConfig}}{}{}
Loads a config file or creates one

\end{fulllineitems}

\index{MessageHandler() (wos.GUI method)}

\begin{fulllineitems}
\phantomsection\label{api:wos.GUI.MessageHandler}\pysiglinewithargsret{\bfcode{MessageHandler}}{\emph{message}}{}
Message handler for received messages
\begin{quote}\begin{description}
\item[{Parameters}] \leavevmode
\textbf{message} (an instance of {\hyperref[api:swnp.Message]{\code{swnp.Message}}}) -- Received message.

\end{description}\end{quote}

\end{fulllineitems}

\index{OnAboutBox() (wos.GUI method)}

\begin{fulllineitems}
\phantomsection\label{api:wos.GUI.OnAboutBox}\pysiglinewithargsret{\bfcode{OnAboutBox}}{\emph{unused\_event}}{}
About dialog
\begin{quote}\begin{description}
\item[{Parameters}] \leavevmode
\textbf{e} (\emph{Event.}) -- GUI Event.

\end{description}\end{quote}

\end{fulllineitems}

\index{OnCreateTables() (wos.GUI method)}

\begin{fulllineitems}
\phantomsection\label{api:wos.GUI.OnCreateTables}\pysiglinewithargsret{\bfcode{OnCreateTables}}{\emph{evt}}{}
Create necessary db tables
\begin{quote}\begin{description}
\item[{Parameters}] \leavevmode
\textbf{evt} (\emph{Event}) -- GUI event.

\end{description}\end{quote}

\end{fulllineitems}

\index{OnEvtBtn() (wos.GUI method)}

\begin{fulllineitems}
\phantomsection\label{api:wos.GUI.OnEvtBtn}\pysiglinewithargsret{\bfcode{OnEvtBtn}}{\emph{evt}}{}
Event Button handler
\begin{quote}\begin{description}
\item[{Parameters}] \leavevmode
\textbf{evt} (\emph{Event.}) -- GUI Event.

\end{description}\end{quote}

\end{fulllineitems}

\index{OnExit() (wos.GUI method)}

\begin{fulllineitems}
\phantomsection\label{api:wos.GUI.OnExit}\pysiglinewithargsret{\bfcode{OnExit}}{\emph{event}}{}
Exits program.
\begin{quote}\begin{description}
\item[{Parameters}] \leavevmode
\textbf{event} (\emph{Event.}) -- GUI Event

\end{description}\end{quote}

\end{fulllineitems}

\index{OnIconify() (wos.GUI method)}

\begin{fulllineitems}
\phantomsection\label{api:wos.GUI.OnIconify}\pysiglinewithargsret{\bfcode{OnIconify}}{\emph{unused\_event}}{}
Window minimize event handler
\begin{quote}\begin{description}
\item[{Parameters}] \leavevmode
\textbf{evt} (\emph{Event.}) -- GUI Event.

\end{description}\end{quote}

\end{fulllineitems}

\index{OnProjectSelected() (wos.GUI method)}

\begin{fulllineitems}
\phantomsection\label{api:wos.GUI.OnProjectSelected}\pysiglinewithargsret{\bfcode{OnProjectSelected}}{}{}
Project selected event handler

\end{fulllineitems}

\index{OnSession() (wos.GUI method)}

\begin{fulllineitems}
\phantomsection\label{api:wos.GUI.OnSession}\pysiglinewithargsret{\bfcode{OnSession}}{\emph{evt}}{}
Session button pressed
\begin{quote}\begin{description}
\item[{Parameters}] \leavevmode
\textbf{evt} (\emph{Event}) -- GUI Event.

\end{description}\end{quote}

\end{fulllineitems}

\index{OnTaskBarActivate() (wos.GUI method)}

\begin{fulllineitems}
\phantomsection\label{api:wos.GUI.OnTaskBarActivate}\pysiglinewithargsret{\bfcode{OnTaskBarActivate}}{\emph{evt}}{}
Taskbar activate event handler
\begin{quote}\begin{description}
\item[{Parameters}] \leavevmode
\textbf{evt} (\emph{Event.}) -- GUI Event.

\end{description}\end{quote}

\end{fulllineitems}

\index{OnTaskBarClose() (wos.GUI method)}

\begin{fulllineitems}
\phantomsection\label{api:wos.GUI.OnTaskBarClose}\pysiglinewithargsret{\bfcode{OnTaskBarClose}}{\emph{unused\_event}}{}
Taskbar close event handler.
\begin{quote}\begin{description}
\item[{Parameters}] \leavevmode
\textbf{evt} (\emph{Event.}) -- GUI Event.

\end{description}\end{quote}

\end{fulllineitems}

\index{OpenProjectDir() (wos.GUI method)}

\begin{fulllineitems}
\phantomsection\label{api:wos.GUI.OpenProjectDir}\pysiglinewithargsret{\bfcode{OpenProjectDir}}{\emph{evt}}{}
Opens project directory in windows explorer
\begin{quote}\begin{description}
\item[{Parameters}] \leavevmode
\textbf{evt} (\emph{event.}) -- The GUI event.

\end{description}\end{quote}

\end{fulllineitems}

\index{PaintSelect() (wos.GUI method)}

\begin{fulllineitems}
\phantomsection\label{api:wos.GUI.PaintSelect}\pysiglinewithargsret{\bfcode{PaintSelect}}{\emph{evt}}{}
Paints the selection of a node.

\begin{notice}{note}{Note:}
For future use.
\end{notice}
\begin{quote}\begin{description}
\item[{Parameters}] \leavevmode
\textbf{evt} (\emph{Event.}) -- GUI Event

\end{description}\end{quote}

\end{fulllineitems}

\index{SelectNode() (wos.GUI method)}

\begin{fulllineitems}
\phantomsection\label{api:wos.GUI.SelectNode}\pysiglinewithargsret{\bfcode{SelectNode}}{\emph{evt}}{}
Handles the selection of a node, start remote control.

\begin{notice}{note}{Note:}
For future use.
\end{notice}
\begin{quote}\begin{description}
\item[{Parameters}] \leavevmode
\textbf{evt} (\emph{Event.}) -- GUI Event

\end{description}\end{quote}

\end{fulllineitems}

\index{SelectProjectDialog() (wos.GUI method)}

\begin{fulllineitems}
\phantomsection\label{api:wos.GUI.SelectProjectDialog}\pysiglinewithargsret{\bfcode{SelectProjectDialog}}{\emph{evt}}{}
Select project event handler
\begin{quote}\begin{description}
\item[{Parameters}] \leavevmode
\textbf{evt} (\emph{Event.}) -- GUI Event.

\end{description}\end{quote}

\end{fulllineitems}

\index{SetCurrentProject() (wos.GUI method)}

\begin{fulllineitems}
\phantomsection\label{api:wos.GUI.SetCurrentProject}\pysiglinewithargsret{\bfcode{SetCurrentProject}}{\emph{project\_id}}{}
Start current project loop 
:param project\_id: The project id from database.
:type project\_id: Integer.

\end{fulllineitems}

\index{SetCurrentSession() (wos.GUI method)}

\begin{fulllineitems}
\phantomsection\label{api:wos.GUI.SetCurrentSession}\pysiglinewithargsret{\bfcode{SetCurrentSession}}{\emph{session\_id}}{}
Set current session
\begin{quote}\begin{description}
\item[{Parameters}] \leavevmode
\textbf{session\_id} (\emph{Integer.}) -- a session id from database.

\end{description}\end{quote}

\end{fulllineitems}

\index{SetProjectObserver() (wos.GUI method)}

\begin{fulllineitems}
\phantomsection\label{api:wos.GUI.SetProjectObserver}\pysiglinewithargsret{\bfcode{SetProjectObserver}}{}{}
Observer for filechanges in project dir

\end{fulllineitems}

\index{SetScanObserver() (wos.GUI method)}

\begin{fulllineitems}
\phantomsection\label{api:wos.GUI.SetScanObserver}\pysiglinewithargsret{\bfcode{SetScanObserver}}{}{}
Observer for created files in scanned or taken with camera

\end{fulllineitems}

\index{Shift() (wos.GUI method)}

\begin{fulllineitems}
\phantomsection\label{api:wos.GUI.Shift}\pysiglinewithargsret{\bfcode{Shift}}{\emph{evt}}{}
Caroussel Shift function
\begin{quote}\begin{description}
\item[{Parameters}] \leavevmode
\textbf{evt} (\emph{Event.}) -- GUI Event.

\end{description}\end{quote}

\end{fulllineitems}

\index{ShowPreferences() (wos.GUI method)}

\begin{fulllineitems}
\phantomsection\label{api:wos.GUI.ShowPreferences}\pysiglinewithargsret{\bfcode{ShowPreferences}}{\emph{evt}}{}
Preferences dialog event handler
\begin{quote}\begin{description}
\item[{Parameters}] \leavevmode
\textbf{evt} (\emph{Event.}) -- GUI Event.

\end{description}\end{quote}

\end{fulllineitems}

\index{StartCurrentProject() (wos.GUI method)}

\begin{fulllineitems}
\phantomsection\label{api:wos.GUI.StartCurrentProject}\pysiglinewithargsret{\bfcode{StartCurrentProject}}{}{}
Start current project loop

\end{fulllineitems}

\index{StartCurrentSession() (wos.GUI method)}

\begin{fulllineitems}
\phantomsection\label{api:wos.GUI.StartCurrentSession}\pysiglinewithargsret{\bfcode{StartCurrentSession}}{}{}
Start current project loop

\end{fulllineitems}

\index{SwnpSend() (wos.GUI method)}

\begin{fulllineitems}
\phantomsection\label{api:wos.GUI.SwnpSend}\pysiglinewithargsret{\bfcode{SwnpSend}}{\emph{node}, \emph{message}}{}
Sends a message to the node.
\begin{quote}\begin{description}
\item[{Parameters}] \leavevmode\begin{itemize}
\item {} 
\textbf{node} (\emph{String.}) -- The node for which to send a message.

\item {} 
\textbf{message} (\emph{String.}) -- The message.

\end{itemize}

\end{description}\end{quote}

\end{fulllineitems}

\index{UpdateScreens() (wos.GUI method)}

\begin{fulllineitems}
\phantomsection\label{api:wos.GUI.UpdateScreens}\pysiglinewithargsret{\bfcode{UpdateScreens}}{\emph{update}}{}
Called when screens need to be updated and redrawn
\begin{quote}\begin{description}
\item[{Parameters}] \leavevmode
\textbf{update} (\emph{Boolean.}) -- Pubsub needs one param, therefore it is called update.

\end{description}\end{quote}

\end{fulllineitems}


\end{fulllineitems}

\index{INPUT\_CAPTURE (class in wos)}

\begin{fulllineitems}
\phantomsection\label{api:wos.INPUT_CAPTURE}\pysiglinewithargsret{\strong{class }\code{wos.}\bfcode{INPUT\_CAPTURE}}{\emph{parent}, \emph{swnp}}{}
Thread for capturing input from mouse/keyboard.
\begin{quote}\begin{description}
\item[{Parameters}] \leavevmode\begin{itemize}
\item {} 
\textbf{parent} ({\hyperref[api:wos.GUI]{\code{GUI}}}.) -- Parent instance.

\item {} 
\textbf{swnp} ({\hyperref[api:swnp.SWNP]{\code{swnp.SWNP}}}) -- SWNP instance for sending data to the network.

\end{itemize}

\end{description}\end{quote}
\index{OnMouseEvent() (wos.INPUT\_CAPTURE method)}

\begin{fulllineitems}
\phantomsection\label{api:wos.INPUT_CAPTURE.OnMouseEvent}\pysiglinewithargsret{\bfcode{OnMouseEvent}}{\emph{event}}{}
WM\_MOUSEFIRST = 0x200

WM\_MOUSEMOVE = 0x200

WM\_LBUTTONDOWN = 0x201

WM\_LBUTTONUP = 0x202

WM\_LBUTTONDBLCLK = 0x203

WM\_RBUTTONDOWN = 0x204

WM\_RBUTTONUP = 0x205

WM\_RBUTTONDBLCLK = 0x206

WM\_MBUTTONDOWN = 0x207

WM\_MBUTTONUP = 0x208

WM\_MBUTTONDBLCLK = 0x209

WM\_MOUSEWHEEL = 0x20A

WM\_MOUSEHWHEEL = 0x20E

\end{fulllineitems}

\index{run() (wos.INPUT\_CAPTURE method)}

\begin{fulllineitems}
\phantomsection\label{api:wos.INPUT_CAPTURE.run}\pysiglinewithargsret{\bfcode{run}}{}{}
Starts the thread.

\end{fulllineitems}

\index{stop() (wos.INPUT\_CAPTURE method)}

\begin{fulllineitems}
\phantomsection\label{api:wos.INPUT_CAPTURE.stop}\pysiglinewithargsret{\bfcode{stop}}{}{}
Stops the thread.

\end{fulllineitems}


\end{fulllineitems}

\index{MySplashScreen (class in wos)}

\begin{fulllineitems}
\phantomsection\label{api:wos.MySplashScreen}\pysiglinewithargsret{\strong{class }\code{wos.}\bfcode{MySplashScreen}}{\emph{parent=None}}{}
Create a splash screen widget.

\end{fulllineitems}

\index{PreferencesDialog (class in wos)}

\begin{fulllineitems}
\phantomsection\label{api:wos.PreferencesDialog}\pysiglinewithargsret{\strong{class }\code{wos.}\bfcode{PreferencesDialog}}{\emph{config}, \emph{evtlist}}{}
Creates and displays a preferences dialog that allows the user to change some settings.
\begin{quote}\begin{description}
\item[{Parameters}] \leavevmode
\textbf{config} -- a Config object

\end{description}\end{quote}
\index{loadPreferences() (wos.PreferencesDialog method)}

\begin{fulllineitems}
\phantomsection\label{api:wos.PreferencesDialog.loadPreferences}\pysiglinewithargsret{\bfcode{loadPreferences}}{}{}
Load the current preferences and fills the text controls

\end{fulllineitems}

\index{onCancel() (wos.PreferencesDialog method)}

\begin{fulllineitems}
\phantomsection\label{api:wos.PreferencesDialog.onCancel}\pysiglinewithargsret{\bfcode{onCancel}}{\emph{event}}{}
Closes the dialog without modifications.
\begin{quote}\begin{description}
\item[{Parameters}] \leavevmode
\textbf{event} (\emph{Event.}) -- GUI event.

\end{description}\end{quote}

\end{fulllineitems}

\index{openConfig() (wos.PreferencesDialog method)}

\begin{fulllineitems}
\phantomsection\label{api:wos.PreferencesDialog.openConfig}\pysiglinewithargsret{\bfcode{openConfig}}{\emph{event}}{}
Closes the dialog without modifications.
\begin{quote}\begin{description}
\item[{Parameters}] \leavevmode
\textbf{event} (\emph{Event.}) -- GUI event.

\end{description}\end{quote}

\end{fulllineitems}

\index{savePreferences() (wos.PreferencesDialog method)}

\begin{fulllineitems}
\phantomsection\label{api:wos.PreferencesDialog.savePreferences}\pysiglinewithargsret{\bfcode{savePreferences}}{\emph{event}}{}
Save the preferences.
\begin{quote}\begin{description}
\item[{Parameters}] \leavevmode
\textbf{event} (\emph{Event.}) -- GUI Event.

\end{description}\end{quote}

\end{fulllineitems}


\end{fulllineitems}

\index{ProjectSelectDialog (class in wos)}

\begin{fulllineitems}
\phantomsection\label{api:wos.ProjectSelectDialog}\pysiglinewithargsret{\strong{class }\code{wos.}\bfcode{ProjectSelectDialog}}{\emph{parent}}{}
A dialog for selecting a project.
\begin{quote}\begin{description}
\item[{Parameters}] \leavevmode
\textbf{parent} (\code{wx.Frame}) -- Parent frame.

\end{description}\end{quote}
\index{AddEvent() (wos.ProjectSelectDialog method)}

\begin{fulllineitems}
\phantomsection\label{api:wos.ProjectSelectDialog.AddEvent}\pysiglinewithargsret{\bfcode{AddEvent}}{\emph{unused\_event}}{}
Shows a modal dialog for adding a new project.
\begin{quote}\begin{description}
\item[{Parameters}] \leavevmode
\textbf{event} (\emph{Event.}) -- GUI Event.

\end{description}\end{quote}

\end{fulllineitems}

\index{DelEvent() (wos.ProjectSelectDialog method)}

\begin{fulllineitems}
\phantomsection\label{api:wos.ProjectSelectDialog.DelEvent}\pysiglinewithargsret{\bfcode{DelEvent}}{\emph{unused\_event}}{}
Handles the selection of a project.
Starts a {\hyperref[api:wos.CURRENT_PROJECT]{\code{wos.CURRENT\_PROJECT}}}, if necessary.
Shows a dialog of the selected project.
\begin{quote}\begin{description}
\item[{Parameters}] \leavevmode
\textbf{evt} (\emph{Event.}) -- GUI Event.

\end{description}\end{quote}

\end{fulllineitems}

\index{EditEvent() (wos.ProjectSelectDialog method)}

\begin{fulllineitems}
\phantomsection\label{api:wos.ProjectSelectDialog.EditEvent}\pysiglinewithargsret{\bfcode{EditEvent}}{\emph{unused\_event}}{}
Shows a modal dialog for adding a new project.
\begin{quote}\begin{description}
\item[{Parameters}] \leavevmode
\textbf{event} (\emph{Event.}) -- GUI Event.

\end{description}\end{quote}

\end{fulllineitems}

\index{GetProjects() (wos.ProjectSelectDialog method)}

\begin{fulllineitems}
\phantomsection\label{api:wos.ProjectSelectDialog.GetProjects}\pysiglinewithargsret{\bfcode{GetProjects}}{\emph{company\_id=1}}{}
Fetches all projects from the database, based on the company.
\begin{quote}\begin{description}
\item[{Parameters}] \leavevmode
\textbf{company\_id} (\emph{Integer.}) -- A company id, the owner of the projects. Defaults to 1.

\end{description}\end{quote}

\end{fulllineitems}

\index{SelEvent() (wos.ProjectSelectDialog method)}

\begin{fulllineitems}
\phantomsection\label{api:wos.ProjectSelectDialog.SelEvent}\pysiglinewithargsret{\bfcode{SelEvent}}{\emph{unused\_event}}{}
Handles the selection of a project.
Starts a {\hyperref[api:wos.CURRENT_PROJECT]{\code{wos.CURRENT\_PROJECT}}}, if necessary.
Shows a dialog of the selected project.
\begin{quote}\begin{description}
\item[{Parameters}] \leavevmode
\textbf{evt} (\emph{Event.}) -- GUI Event.

\end{description}\end{quote}

\end{fulllineitems}

\index{onCancel() (wos.ProjectSelectDialog method)}

\begin{fulllineitems}
\phantomsection\label{api:wos.ProjectSelectDialog.onCancel}\pysiglinewithargsret{\bfcode{onCancel}}{\emph{unused\_event}}{}
Handles ``Cancel'' button presses.
\begin{quote}\begin{description}
\item[{Parameters}] \leavevmode
\textbf{event} (\emph{Event.}) -- GUI Event.

\end{description}\end{quote}

\end{fulllineitems}


\end{fulllineitems}

\index{SEND\_FILE\_CONTEX\_MENU\_HANDLER (class in wos)}

\begin{fulllineitems}
\phantomsection\label{api:wos.SEND_FILE_CONTEX_MENU_HANDLER}\pysiglinewithargsret{\strong{class }\code{wos.}\bfcode{SEND\_FILE\_CONTEX\_MENU\_HANDLER}}{\emph{parent}, \emph{context}, \emph{send\_file}, \emph{handle\_file}}{}
Thread for OS contex menu actions like file sending to other node.
\begin{quote}\begin{description}
\item[{Parameters}] \leavevmode\begin{itemize}
\item {} 
\textbf{context} (\emph{ZeroMQ context.}) -- Context for creating sockets.

\item {} 
\textbf{send\_file} (\emph{Function.}) -- Sends files.

\item {} 
\textbf{handle\_file} (\emph{Function.}) -- Handles files

\end{itemize}

\end{description}\end{quote}
\index{run() (wos.SEND\_FILE\_CONTEX\_MENU\_HANDLER method)}

\begin{fulllineitems}
\phantomsection\label{api:wos.SEND_FILE_CONTEX_MENU_HANDLER.run}\pysiglinewithargsret{\bfcode{run}}{}{}
Starts the thread

\end{fulllineitems}

\index{stop() (wos.SEND\_FILE\_CONTEX\_MENU\_HANDLER method)}

\begin{fulllineitems}
\phantomsection\label{api:wos.SEND_FILE_CONTEX_MENU_HANDLER.stop}\pysiglinewithargsret{\bfcode{stop}}{}{}
Stops the thread

\end{fulllineitems}


\end{fulllineitems}

\index{SysTray (class in wos)}

\begin{fulllineitems}
\phantomsection\label{api:wos.SysTray}\pysiglinewithargsret{\strong{class }\code{wos.}\bfcode{SysTray}}{\emph{parent}}{}
Taskbar Icon class
\begin{quote}\begin{description}
\item[{Parameters}] \leavevmode
\textbf{parent} (\code{wx.Frame}) -- Parent frame

\end{description}\end{quote}
\index{CreateMenu() (wos.SysTray method)}

\begin{fulllineitems}
\phantomsection\label{api:wos.SysTray.CreateMenu}\pysiglinewithargsret{\bfcode{CreateMenu}}{}{}
Create systray menu

\end{fulllineitems}

\index{ShowMenu() (wos.SysTray method)}

\begin{fulllineitems}
\phantomsection\label{api:wos.SysTray.ShowMenu}\pysiglinewithargsret{\bfcode{ShowMenu}}{\emph{event}}{}
Show popup menu
\begin{quote}\begin{description}
\item[{Parameters}] \leavevmode
\textbf{event} (\emph{Event.}) -- GUI event.

\end{description}\end{quote}

\end{fulllineitems}


\end{fulllineitems}

\index{UpdateDialog (class in wos)}

\begin{fulllineitems}
\phantomsection\label{api:wos.UpdateDialog}\pysiglinewithargsret{\strong{class }\code{wos.}\bfcode{UpdateDialog}}{\emph{title}, \emph{url}, \emph{*args}, \emph{**kwargs}}{}
A Dialog which notifies about a software update

\end{fulllineitems}

\index{WORKER\_THREAD (class in wos)}

\begin{fulllineitems}
\phantomsection\label{api:wos.WORKER_THREAD}\pysiglinewithargsret{\strong{class }\code{wos.}\bfcode{WORKER\_THREAD}}{\emph{parent}}{}
Worker thread for non-UI jobs.
\begin{quote}\begin{description}
\item[{Parameters}] \leavevmode\begin{itemize}
\item {} 
\textbf{context} (\emph{ZeroMQ context.}) -- Context for creating sockets.

\item {} 
\textbf{send\_file} (\emph{Function.}) -- Sends files.

\item {} 
\textbf{handle\_file} (\emph{Function.}) -- Handles files

\end{itemize}

\end{description}\end{quote}
\index{AddProjectReg() (wos.WORKER\_THREAD method)}

\begin{fulllineitems}
\phantomsection\label{api:wos.WORKER_THREAD.AddProjectReg}\pysiglinewithargsret{\bfcode{AddProjectReg}}{}{}
Adds project folder to registry

\end{fulllineitems}

\index{AddRegEntry() (wos.WORKER\_THREAD method)}

\begin{fulllineitems}
\phantomsection\label{api:wos.WORKER_THREAD.AddRegEntry}\pysiglinewithargsret{\bfcode{AddRegEntry}}{\emph{name}, \emph{id}}{}
Adds a node to registry
\begin{quote}\begin{description}
\item[{Parameters}] \leavevmode\begin{itemize}
\item {} 
\textbf{name} (\emph{String}) -- Node name.

\item {} 
\textbf{id} (\emph{Integer.}) -- Node id.

\end{itemize}

\end{description}\end{quote}

\end{fulllineitems}

\index{RemoveAllRegEntries() (wos.WORKER\_THREAD method)}

\begin{fulllineitems}
\phantomsection\label{api:wos.WORKER_THREAD.RemoveAllRegEntries}\pysiglinewithargsret{\bfcode{RemoveAllRegEntries}}{}{}
Removes all related registry entries

\end{fulllineitems}

\index{parseConfig() (wos.WORKER\_THREAD method)}

\begin{fulllineitems}
\phantomsection\label{api:wos.WORKER_THREAD.parseConfig}\pysiglinewithargsret{\bfcode{parseConfig}}{\emph{config}}{}
Handles config file settings

\end{fulllineitems}


\end{fulllineitems}



\chapter{Bugs}
\label{bugs::doc}\label{bugs:bugs}
\begin{tabulary}{\linewidth}{|L|L|L|}
\hline
\textbf{\relax 
Bug
} & \textbf{\relax 
Description
} & \textbf{\relax 
Status
}\\\hline

Sample bug
 & 
Description for sample
 & 
Open / Closed / Will not be fixed
\\\hline
\end{tabulary}



\chapter{Features}
\label{features::doc}\label{features:features}
\begin{tabulary}{\linewidth}{|L|L|}
\hline
\textbf{\relax 
Feature
} & \textbf{\relax 
Description
}\\\hline

Project
 & 
User can add, edit and select a project
\\\hline

Session
 & 
User can start, end and continue sessions
\\\hline

Event
 & 
User can tag an interesting event during  a session
\\\hline

File Monitoring
 & 
Users' file actions are monitored during a session. It includes opening files.
\\\hline
\end{tabulary}



\chapter{License}
\label{license::doc}\label{license:license}
European Union Public Licence
\begin{enumerate}
\setcounter{enumi}{21}
\item {} 
1.1

\end{enumerate}

EUPL © the European Community 2007

This European Union Public Licence (the “EUPL”) applies to the Work or Software
(as defined below) which is provided under the terms of this Licence. Any use of the
Work, other than as authorised under this Licence is prohibited (to the extent such use
is covered by a right of the copyright holder of the Work).

The Original Work is provided under the terms of this Licence when the Licensor (as
defined below) has placed the following notice immediately following the copyright
notice for the Original Work:
\begin{quote}

Licensed under the EUPL V.1.1
\end{quote}

or has expressed by any other mean his willingness to license under the EUPL.


\section{1. Definitions}
\label{license:definitions}
In this Licence, the following terms have the following meaning:
\begin{itemize}
\item {} \begin{description}
\item[{The Licence:}] \leavevmode
This Licence.

\end{description}

\item {} \begin{description}
\item[{The Original Work or the Software:}] \leavevmode
The software distributed and/or communicated by the Licensor under this Licence,
available as Source Code and also as Executable Code as the case may be.

\end{description}

\item {} \begin{description}
\item[{Derivative Works:}] \leavevmode
The works or software that could be created by the Licensee,
based upon the Original Work or modifications thereof. This Licence does not define
the extent of modification or dependence on the Original Work required in order to
classify a work as a Derivative Work; this extent is determined by copyright law
applicable in the country mentioned in Article 15.

\end{description}

\item {} \begin{description}
\item[{The Work:}] \leavevmode
The Original Work and/or its Derivative Works.

\end{description}

\item {} \begin{description}
\item[{The Source Code:}] \leavevmode
The human-readable form of the Work which is the most convenient for people to study and modify.

\end{description}

\item {} \begin{description}
\item[{The Executable Code:}] \leavevmode
Any code which has generally been compiled and which is meant to be interpreted by a computer as a program.

\end{description}

\item {} \begin{description}
\item[{The Licensor:}] \leavevmode
The natural or legal person that distributes and/or communicates the Work under the Licence.

\end{description}

\item {} \begin{description}
\item[{Contributor(s):}] \leavevmode
Any natural or legal person who modifies the Work under the Licence, or otherwise contributes to the creation of a Derivative Work.

\end{description}

\item {} \begin{description}
\item[{The Licensee or “You”:}] \leavevmode
Any natural or legal person who makes any usage of the Software under the terms of the Licence.

\end{description}

\item {} \begin{description}
\item[{Distribution and/or Communication:}] \leavevmode
Any act of selling, giving, lending, renting,
distributing, communicating, transmitting, or otherwise making available, on-line or
off-line, copies of the Work or providing access to its essential functionalities at the
disposal of any other natural or legal person.

\end{description}

\end{itemize}


\section{2. Scope of the rights granted by the Licence}
\label{license:scope-of-the-rights-granted-by-the-licence}
The Licensor hereby grants You a world-wide, royalty-free, non-exclusive, sublicensable
licence to do the following, for the duration of copyright vested in the
Original Work:
\begin{itemize}
\item {} 
use the Work in any circumstance and for all usage,

\item {} 
reproduce the Work,

\item {} 
modify the Original Work, and make Derivative Works based upon the Work,

\item {} 
communicate to the public, including the right to make available or display the
Work or copies thereof to the public and perform publicly, as the case may be, the Work,

\item {} 
distribute the Work or copies thereof,

\item {} 
lend and rent the Work or copies thereof,

\item {} 
sub-license rights in the Work or copies thereof.

\end{itemize}

Those rights can be exercised on any media, supports and formats, whether now
known or later invented, as far as the applicable law permits so.

In the countries where moral rights apply, the Licensor waives his right to exercise his
moral right to the extent allowed by law in order to make effective the licence of the
economic rights here above listed.

The Licensor grants to the Licensee royalty-free, non exclusive usage rights to any
patents held by the Licensor, to the extent necessary to make use of the rights granted
on the Work under this Licence.


\section{3. Communication of the Source Code}
\label{license:communication-of-the-source-code}
The Licensor may provide the Work either in its Source Code form, or as Executable
Code. If the Work is provided as Executable Code, the Licensor provides in addition a
machine-readable copy of the Source Code of the Work along with each copy of the
Work that the Licensor distributes or indicates, in a notice following the copyright
notice attached to the Work, a repository where the Source Code is easily and freely
accessible for as long as the Licensor continues to distribute and/or communicate the
Work.


\section{4. Limitations on copyright}
\label{license:limitations-on-copyright}
Nothing in this Licence is intended to deprive the Licensee of the benefits from any
exception or limitation to the exclusive rights of the rights owners in the Original
Work or Software, of the exhaustion of those rights or of other applicable limitations
thereto.


\section{5. Obligations of the Licensee}
\label{license:obligations-of-the-licensee}
The grant of the rights mentioned above is subject to some restrictions and obligations
imposed on the Licensee. Those obligations are the following:

Attribution right: the Licensee shall keep intact all copyright, patent or trademarks
notices and all notices that refer to the Licence and to the disclaimer of warranties.
The Licensee must include a copy of such notices and a copy of the Licence with
every copy of the Work he/she distributes and/or communicates. The Licensee must
cause any Derivative Work to carry prominent notices stating that the Work has been
modified and the date of modification.
\begin{description}
\item[{Copyleft clause:}] \leavevmode
If the Licensee distributes and/or communicates copies of the
Original Works or Derivative Works based upon the Original Work, this Distribution
and/or Communication will be done under the terms of this Licence or of a later
version of this Licence unless the Original Work is expressly distributed only under
this version of the Licence. The Licensee (becoming Licensor) cannot offer or impose
any additional terms or conditions on the Work or Derivative Work that alter or
restrict the terms of the Licence.

\item[{Compatibility clause:}] \leavevmode
If the Licensee Distributes and/or Communicates Derivative
Works or copies thereof based upon both the Original Work and another work
licensed under a Compatible Licence, this Distribution and/or Communication can be
done under the terms of this Compatible Licence. For the sake of this clause,
“Compatible Licence” refers to the licences listed in the appendix attached to this
Licence. Should the Licensee’s obligations under the Compatible Licence conflict
with his/her obligations under this Licence, the obligations of the Compatible Licence
shall prevail.

\item[{Provision of Source Code:}] \leavevmode
When distributing and/or communicating copies of the
Work, the Licensee will provide a machine-readable copy of the Source Code or
indicate a repository where this Source will be easily and freely available for as long
as the Licensee continues to distribute and/or communicate the Work.

\item[{Legal Protection:}] \leavevmode
This Licence does not grant permission to use the trade names,
trademarks, service marks, or names of the Licensor, except as required for
reasonable and customary use in describing the origin of the Work and reproducing
the content of the copyright notice.

\end{description}


\section{6. Chain of Authorship}
\label{license:chain-of-authorship}
The original Licensor warrants that the copyright in the Original Work granted
hereunder is owned by him/her or licensed to him/her and that he/she has the power
and authority to grant the Licence.

Each Contributor warrants that the copyright in the modifications he/she brings to the
Work are owned by him/her or licensed to him/her and that he/she has the power and
authority to grant the Licence.

Each time You accept the Licence, the original Licensor and subsequent Contributors
grant You a licence to their contributions to the Work, under the terms of this
Licence.


\section{7. Disclaimer of Warranty}
\label{license:disclaimer-of-warranty}
The Work is a work in progress, which is continuously improved by numerous
contributors. It is not a finished work and may therefore contain defects or “bugs”
inherent to this type of software development.

For the above reason, the Work is provided under the Licence on an “as is” basis and
without warranties of any kind concerning the Work, including without limitation
merchantability, fitness for a particular purpose, absence of defects or errors,
accuracy, non-infringement of intellectual property rights other than copyright as
stated in Article 6 of this Licence.

This disclaimer of warranty is an essential part of the Licence and a condition for the
grant of any rights to the Work.


\section{8. Disclaimer of Liability}
\label{license:disclaimer-of-liability}
Except in the cases of wilful misconduct or damages directly caused to natural
persons, the Licensor will in no event be liable for any direct or indirect, material or
moral, damages of any kind, arising out of the Licence or of the use of the Work,
including without limitation, damages for loss of goodwill, work stoppage, computer
failure or malfunction, loss of data or any commercial damage, even if the Licensor
has been advised of the possibility of such damage. However, the Licensor will be
liable under statutory product liability laws as far such laws apply to the Work.


\section{9. Additional agreements}
\label{license:additional-agreements}
While distributing the Original Work or Derivative Works, You may choose to
conclude an additional agreement to offer, and charge a fee for, acceptance of support,
warranty, indemnity, or other liability obligations and/or services consistent with this
Licence. However, in accepting such obligations, You may act only on your own
behalf and on your sole responsibility, not on behalf of the original Licensor or any
other Contributor, and only if You agree to indemnify, defend, and hold each
Contributor harmless for any liability incurred by, or claims asserted against such
Contributor by the fact You have accepted any such warranty or additional liability.


\section{10. Acceptance of the Licence}
\label{license:acceptance-of-the-licence}
The provisions of this Licence can be accepted by clicking on an icon “I agree”
placed under the bottom of a window displaying the text of this Licence or by
affirming consent in any other similar way, in accordance with the rules of applicable
law. Clicking on that icon indicates your clear and irrevocable acceptance of this
Licence and all of its terms and conditions.

Similarly, you irrevocably accept this Licence and all of its terms and conditions by
exercising any rights granted to You by Article 2 of this Licence, such as the use of
the Work, the creation by You of a Derivative Work or the Distribution and/or
Communication by You of the Work or copies thereof.


\section{11. Information to the public}
\label{license:information-to-the-public}
In case of any Distribution and/or Communication of the Work by means of electronic
communication by You (for example, by offering to download the Work from a
remote location) the distribution channel or media (for example, a website) must at
least provide to the public the information requested by the applicable law regarding
the Licensor, the Licence and the way it may be accessible, concluded, stored and
reproduced by the Licensee.


\section{12. Termination of the Licence}
\label{license:termination-of-the-licence}
The Licence and the rights granted hereunder will terminate automatically upon any
breach by the Licensee of the terms of the Licence.

Such a termination will not terminate the licences of any person who has received the
Work from the Licensee under the Licence, provided such persons remain in full
compliance with the Licence.


\section{13. Miscellaneous}
\label{license:miscellaneous}
Without prejudice of Article 9 above, the Licence represents the complete agreement
between the Parties as to the Work licensed hereunder.

If any provision of the Licence is invalid or unenforceable under applicable law, this
will not affect the validity or enforceability of the Licence as a whole. Such provision
will be construed and/or reformed so as necessary to make it valid and enforceable.

The European Commission may publish other linguistic versions and/or new versions
of this Licence, so far this is required and reasonable, without reducing the scope of
the rights granted by the Licence. New versions of the Licence will be published with
a unique version number.

All linguistic versions of this Licence, approved by the European Commission, have
identical value. Parties can take advantage of the linguistic version of their choice.


\section{14. Jurisdiction}
\label{license:jurisdiction}
Any litigation resulting from the interpretation of this License, arising between the
European Commission, as a Licensor, and any Licensee, will be subject to the
jurisdiction of the Court of Justice of the European Communities, as laid down in
article 238 of the Treaty establishing the European Community.

Any litigation arising between Parties, other than the European Commission, and
resulting from the interpretation of this License, will be subject to the exclusive
jurisdiction of the competent court where the Licensor resides or conducts its primary
business.


\section{15. Applicable Law}
\label{license:applicable-law}
This Licence shall be governed by the law of the European Union country where the
Licensor resides or has his registered office.
This licence shall be governed by the Belgian law if:
- a litigation arises between the European Commission, as a Licensor, and any Licensee;
- the Licensor, other than the European Commission, has no residence or registered office inside a European Union country.


\section{Appendix}
\label{license:appendix}\begin{description}
\item[{“Compatible Licences” according to article 5 EUPL are:}] \leavevmode\begin{itemize}
\item {} 
GNU General Public License (GNU GPL) v. 2

\item {} 
Open Software License (OSL) v. 2.1, v. 3.0

\item {} 
Common Public License v. 1.0

\item {} 
Eclipse Public License v. 1.0

\item {} 
Cecill v. 2.0

\end{itemize}

\end{description}


\chapter{User Interface}
\label{ui::doc}\label{ui:user-interface}\begin{figure}[htbp]
\centering
\capstart

\includegraphics{screencap.png}
\caption{The UI of Chimaira; Several icons for different functions.}\end{figure}

The screen icons identify different screens nodes. The user can drop files on to these icons, causing the dropped files to be opened in the specific node. The arrows control the carousel of nodes, and are visble only if more than three nodes are connected. The drop-down list in holds recently viewed files in the selected project.


\begin{threeparttable}
\capstart\caption{Icons explained}

\begin{tabulary}{\linewidth}{|L|L|}
\hline
\textbf{\relax 
Icon
} & \textbf{\relax 
Description
}\\\hline

Briefcase
 & 
Select a project
\\\hline

Clock
 & 
Start / End a session
\\\hline

Folder
 & 
Open project directory
\\\hline

Note
 & 
Create an Event note
\\\hline

Circle
 & 
Hide the application
\\\hline

Cross
 & 
Exit the application
\\\hline
\end{tabulary}

\end{threeparttable}



\chapter{Indices and tables}
\label{index:indices-and-tables}\begin{itemize}
\item {} 
\emph{genindex}

\item {} 
\emph{modindex}

\item {} 
\emph{search}

\end{itemize}


\renewcommand{\indexname}{Python Module Index}
\begin{theindex}
\def\bigletter#1{{\Large\sffamily#1}\nopagebreak\vspace{1mm}}
\bigletter{a}
\item {\texttt{add\_file}}, \pageref{api:module-add_file}
\indexspace
\bigletter{c}
\item {\texttt{controller}}, \pageref{api:module-controller}
\indexspace
\bigletter{f}
\item {\texttt{filesystem}}, \pageref{api:module-filesystem}
\indexspace
\bigletter{m}
\item {\texttt{models}}, \pageref{api:module-models}
\indexspace
\bigletter{s}
\item {\texttt{send\_file\_to}}, \pageref{api:module-send_file_to}
\item {\texttt{swnp}}, \pageref{api:module-swnp}
\indexspace
\bigletter{u}
\item {\texttt{utils}}, \pageref{api:module-utils}
\indexspace
\bigletter{w}
\item {\texttt{wos}}, \pageref{api:module-wos}
\end{theindex}

\renewcommand{\indexname}{Index}
\printindex
\end{document}
